\documentclass[12pt]{article}

\usepackage{tikz} % картинки в tikz
\usepackage{microtype} % свешивание пунктуации

\usepackage{array} % для столбцов фиксированной ширины

\usepackage{indentfirst} % отступ в первом параграфе

\usepackage{sectsty} % для центрирования названий частей
\allsectionsfont{\centering}

\usepackage{amsmath} % куча стандартных математических плюшек

\usepackage{comment}
\usepackage{amsfonts}

\usepackage[top=2cm, left=1.2cm, right=1.2cm, bottom=2cm]{geometry} % размер текста на странице

\usepackage{lastpage} % чтобы узнать номер последней страницы

\usepackage{enumitem} % дополнительные плюшки для списков
%  например \begin{enumerate}[resume] позволяет продолжить нумерацию в новом списке
\usepackage{caption}

\usepackage{longtable}
\usepackage{multicol}
\usepackage{multirow}


\usepackage{fancyhdr} % весёлые колонтитулы
\pagestyle{fancy}
\lhead{Теория вероятностей}
\chead{}
\rhead{Минимум к контрольной \textnumero 3 по ТВ и МС}
\lfoot{}
\cfoot{}
\rfoot{\thepage/\pageref{LastPage}}
\renewcommand{\headrulewidth}{0.4pt}
\renewcommand{\footrulewidth}{0.4pt}



\usepackage{todonotes} % для вставки в документ заметок о том, что осталось сделать
% \todo{Здесь надо коэффициенты исправить}
% \missingfigure{Здесь будет Последний день Помпеи}
% \listoftodos --- печатает все поставленные \todo'шки


% более красивые таблицы
\usepackage{booktabs}
% заповеди из документации:
% 1. Не используйте вертикальные линии
% 2. Не используйте двойные линии
% 3. Единицы измерения - в шапку таблицы
% 4. Не сокращайте .1 вместо 0.1
% 5. Повторяющееся значение повторяйте, а не говорите "то же"



\usepackage{fontspec}
\usepackage{polyglossia}

\setmainlanguage{russian}
\setotherlanguages{english}

% download "Linux Libertine" fonts:
% http://www.linuxlibertine.org/index.php?id=91&L=1
\setmainfont{Linux Libertine O} % or Helvetica, Arial, Cambria
% why do we need \newfontfamily:
% http://tex.stackexchange.com/questions/91507/
\newfontfamily{\cyrillicfonttt}{Linux Libertine O}

\AddEnumerateCounter{\asbuk}{\russian@alph}{щ} % для списков с русскими буквами
\setlist[enumerate, 2]{label=\asbuk*),ref=\asbuk*}

%% эконометрические сокращения
\DeclareMathOperator{\Cov}{Cov}
\DeclareMathOperator{\Corr}{Corr}
\DeclareMathOperator{\Var}{Var}
\DeclareMathOperator{\E}{E}
\def \hb{\hat{\beta}}
\def \hs{\hat{\sigma}}
\def \htheta{\hat{\theta}}
\def \s{\sigma}
\def \hy{\hat{y}}
\def \hY{\hat{Y}}
\def \v1{\vec{1}}
\def \e{\varepsilon}
\def \he{\hat{\e}}
\def \z{z}
\def \hVar{\widehat{\Var}}
\def \hCorr{\widehat{\Corr}}
\def \hCov{\widehat{\Cov}}
\def \cN{\mathcal{N}}
\def \P{\mathbb{P}}


\begin{document}

\section{Теоретический минимум}

\begin{enumerate}
  \item Дайте определение нормально распределённой случайной величины. Укажите диапазон возможных значений, функцию плотности, ожидание, дисперсию. Нарисуйте функцию плотности.
  \item Дайте определение хи-квадрат распределения. Укажите диапазон возможных значений, выражение через нормальные распределения, математическое ожидание. Нарисуйте функцию плотности при разных степенях свободы.
  \item Дайте определение распределения Стьюдента. Укажите диапазон возможных значений, выражение через нормальные распределения. Нарисуйте функцию плотности распределения Стьюдента при разных степенях свободы на фоне нормальной стандартной функции плотности.
  \item Дайте определение распределения Фишера. Укажите диапазон возможных значений, выражение через нормальные распределенеия. Нарисуйте возможную функцию плотности.
\end{enumerate}

Для следующего блока вопросов предполагается, что
имеется случайная выборка $X_1$, $X_2$, \ldots, $X_n$ из распределения
с функцией плотности $f(x, \theta)$, зависящей от от параметра $\theta$. Дайте определение каждого понятия из списка или сформулируйте соответствующую теорему:

\begin{enumerate}[resume]
  \item Выборочное среднее и выборочная дисперсия;
  \item Формула несмещённой оценки дисперсии;
  \item Выборочный начальный момент порядка $k$;
  \item Выборочный центральный момент порядка $k$;
  \item Выборочная функция распределения;
  \item Несмещённая оценка $\hat \theta$ параметра $\theta$;
  \item Состоятельная последовательность оценок $\hat \theta_n$;
  \item Эффективность оценки $\hat \theta$ среди множества оценок $\hat \Theta$;
  \item Неравенство Крамера–Рао для несмещённых оценок;
  \item Функция правдоподобия и логарифмическая функция правдоподобия;
  \item Информация Фишера о параметре $\theta$, содержащаяся в одном наблюдении;
  \item Оценка метода моментов параметра $\theta$ при использовании первого момента, если $\E(X_i)=g(\theta)$ и существует обратная функция $g^{-1}$;
  \item Оценка метода максимального правдоподобия параметра $\theta$;
\end{enumerate}

Для следующего блока вопросов предполагается, что величины $X_1$, $X_2$, \ldots, $X_n$ независимы и нормальны $\cN(\mu;\sigma^2)$.

\begin{enumerate}[resume]
  \item Укажите закон распределения выборочного среднего, величины $\frac{\bar X - \mu}{\sigma/\sqrt{n}}$, величины $\frac{\bar X - \mu}{\hat\sigma/\sqrt{n}}$, величины $\frac{\hat\sigma^2(n-1)}{\sigma^2}$;
  \item Укажите формулу доверительного интервала с уровнем доверия $(1-\alpha)$ для $\mu$ при известной дисперсии, для $\mu$ при неизвестной дисперсии, для $\sigma^2$;
\end{enumerate}

\newpage
\section{Задачный минимум}

\begin{enumerate}

\item Рост в сантиметрах (случайная величина $X$) и вес в килограммах (случайная величина $Y$) взрослого мужчины является нормальным случайным вектором $Z = (X, Y)$ с математическим ожиданием $\E(Z) = (175, 74)$ и ковариационной матрицей

\[
\Var(Z) =
\begin{pmatrix}
 49 & 28 \\
28 & 36
\end{pmatrix}
\]

Лишний вес характеризуется случайной величиной $U = X - Y$. Считается, что человек страдает избыточным весом, если $U < 90$.

\begin{enumerate}
\item Определите вероятность того, что рост мужчины отклоняется от среднего более, чем на $10$ см.
\item Укажите распределение случайной величины $U$. Выпишите её плотность распределения.
\item Найдите вероятность того, что случайно выбранный мужчина страдает избыточным весом.
\end{enumerate}

\item Рост в сантиметрах, случайная величина $X$, и вес в килограммах, случайная величина $Y$, взрослого мужчины является нормальным случайным вектором $Z = (X, Y)$ с математическим ожиданием $\E(Z) = (175, 74)$ и ковариационной матрицей

\[
\Var(Z) =
\begin{pmatrix}
 49 & 28 \\
28 & 36
\end{pmatrix}
\]

\begin{enumerate}
\item Найдите средний вес мужчины при условии, что его рост составляет $170$ см.
\item Выпишите условную плотность распределения веса мужчины при условии, что его рост составляет $170$ см.
\item Найдите условную вероятность того, что человек будет иметь вес, больший $90$ кг, при условии, что его рост составляет $170$ см.
\end{enumerate}

\item Для реализации случайной выборки $x=(1,0,-1,1)$ найдите:

\begin{enumerate}
\item выборочное среднее,
\item неисправленную выборочную дисперсию,
\item исправленную выборочную дисперсию,
\item выборочный второй начальный момент,
\item выборочный третий центральный момент,
\end{enumerate}

\item Для реализации случайной выборки $x=(1,0,-1,1)$ найдите:

\begin{enumerate}
\item вариационный ряд,
\item первый член вариационного ряда,
\item последний член вариационного ряда,
\item график выборочной функции распределения.
\end{enumerate}

\item Пусть $X=(X_1, \ldots,X_n)$ — случайная выборка из дискретного распределения, заданного с помощью таблицы

\begin{center}
\begin{tabular}{cccc}
\toprule
 $x$ & $-3$  &$ 0 $  & $2 $  \\
 \midrule
 $\P(X_i = x)$ & $2/3 - \theta$ & $1/3$ & $\theta$ \\
 \bottomrule
\end{tabular}
\end{center}

Рассмотрите оценку $\hat{\theta} = \dfrac{\bar{X}+2}{5}$.

\begin{enumerate}
    \item Найдите $\E[\hat{\theta}]$.
    \item Является ли оценка $\hat{\theta}$ несмещенной оценкой неизвестного параметра $\theta$?
\end{enumerate}

\item Пусть $X=(X_1, \ldots ,X_n)$ — случайная выборка из распределения с плотностью распределения

\[
f(x,\theta) = \begin{cases}
\dfrac{6x(\theta - x)}{\theta^3} & \text{при } x \in [0;\theta], \\
0 & \text{при } x \not\in [0;\theta],
\end{cases}
\]


где $\theta > 0$ — неизвестный параметр распределения и $\hat{\theta} = \bar{X}$.

\begin{enumerate}
\item Является ли оценка $\hat{\theta} = \bar{X}$ несмещенной оценкой неизвестного параметра $\theta$?
\item Подберите константу $c$ так, чтобы оценка $\tilde{\theta} = c\bar{X}$ оказалась несмещенной оценкой неизвестного параметра $\theta$.
\end{enumerate}

\item Пусть $X = (X_1,X_2,X_3)$ — случайная выборка из распределения Бернулли с неизвестным параметром $p \in (0,1)$. Какие из следующих ниже оценкой являются несмещенными? Среди перечисленных ниже оценок найдите наиболее эффективную оценку:

\begin{itemize}
  \item $\hat{p}_1 = \dfrac{X_1+X_3}{2}$,
  \item $\hat{p}_2 = \frac{1}{4}X_1+\frac{1}{2}X_2+\frac{1}{4}X_3$,
  \item $\hat{p}_3 = \frac{1}{3}X_1+\frac{1}{3}X_2+\frac{1}{3}X_3$.
\end{itemize}

\item Пусть $X=(X_1, \ldots,X_n)$ — случайная выборка из распределения с плотностью

\[
f(x,\theta) =
\begin{cases}
\frac{1}{\theta} \ e^{-\frac{x}{\theta}} & \text{при } x \geq 0, \\
0 & \text{при } x < 0,
\end{cases}
\]
где $\theta > 0$ — неизвестный параметр.
Является ли оценка  $\hat{\theta}_n = \dfrac{X_1+...+X_n}{n+1}$ состоятельной?

\item Пусть $X=(X_1, \ldots ,X_n)$ — случайная выборка из распределения с плотностью распределения

\[
f(x,\theta) = \begin{cases}
\dfrac{6x(\theta-x)}{\theta^3} & \text{при } x \in [0;\theta], \\
0 & \text{при } x \not\in [0;\theta], \end{cases}
\]


где $\theta > 0$ — неизвестный параметр распределения. Является ли оценка \ $\hat{\theta}_n = \frac{2n+1}{n}\bar{X}_n$ состоятельной оценкой неизвестного параметра $\theta$?

\item Пусть $X=(X_1, \ldots ,X_n)$ — случайная выборка из распределения с плотностью распределения

\[
f(x,\theta) =
\begin{cases}
\dfrac{6x(\theta-x)}{\theta^3} & \text{при } x \in [0;\theta], \\
0 & \text{при } x \not\in [0;\theta],
\end{cases}
\]


где $\theta > 0$ — неизвестный параметр распределения. Используя центральный момент 2-го порядка, при помощи метода моментов найдите оценку для неизвестного параметра $\theta$.

\item Пусть $X=(X_1, \ldots,X_n)$ — случайная выборка. Случайные величины $X_1, \ldots, X_n$ имеют дискретное распределение, которое задано при помощи таблицы

\begin{center}
\begin{tabular}{cccc}
\toprule
 $x$ & $-3$  &$ 0 $  & $2 $  \\
 \midrule
 $\P(X_i = x)$ & $2/3 - \theta$ & $1/3$ & $\theta$ \\
 \bottomrule
\end{tabular}
\end{center}

Используя второй начальный момент, при помощи метода моментов найдите оценку неизвестного параметра $\theta$. Для реализации случайной выборки $x=(0,0,-3,0,2)$ найдите числовое значение найденной оценки параметра $\theta$.

\item Пусть $X=(X_1, \ldots,X_n)$ — случайная выборка из распределения с плотностью распределения

\[
f(x,\theta) =
\begin{cases}
\frac{2x}{\theta} \ e^{-\frac{x^2}{\theta}} & \text{при } x>0, \\
0 & \text{при } x \leq 0,
\end{cases}
\]

где $\theta > 0$. При помощи метода максимального правдоподобия найдите оценку неизвестного параметра $\theta$.

\item Пусть $X=(X_1, \ldots, X_n)$ – случайная выборка из распределения Бернулли с параметром $\P \in (0;1)$. При помощи метода максимального правдоподобия найдите оценку неизвестного параметра $\P$.

\item Пусть $X=(X_1, \ldots, X_n)$ — случайная выборка из распределения с плотностью

\[
f(x,\theta) =
\begin{cases}
\frac{1}{\theta} \ e^{-\frac{x}{\theta}} & \text{при } x \geq 0, \\
0 & \text{при } x < 0, \end{cases}
\]

где $\theta > 0$ — неизвестный параметр. Является ли оценка  $\hat{\theta} = \bar{X}$ эффективной?

\item Стоимость выборочного исследования генеральной совокупности, состоящей из трех страт, определяется по формуле $TC = c_1n_1 + c_2n_2 + c_3n_3$, где $c_i$ — цена одного наблюдения в $i$-ой страте, a $n_i$ — число наблюдений, которые приходятся на $i$-ую страту. Найдите $n_1$, $n_2$ и $n_3$, при которых дисперсия стратифицированного среднего достигает наименьшего значения, если бюджет исследования 8000 и имеется следующая информация:

\begin{center}
\begin{tabular}{cccc}
\toprule
 Страта & $1$ & $2$ & $3$  \\
 \midrule
 Среднее значение & $30$ & $40$ & $50$ \\
 Стандартная ошибка  & $5$ & $10$ & $20$ \\
 Вес & $25\%$ & $25\%$ & $50\%$ \\
 Цена наблюдения & $1$ & $5$ & $10$ \\
 \bottomrule
\end{tabular}
\end{center}

\end{enumerate}

Ответы:

\begin{enumerate}
\item
\begin{enumerate}
\item $\approx 0.15$
\item $U \sim \cN(101,29)$, $f(u) = \frac{1}{\sqrt{2\pi\cdot 29}}e^{-\frac{1}{2}\frac{(u-101)^2}{29}}$
\item $\approx 0.02$
\end{enumerate}
\item
\begin{enumerate}
\item $71.14$
\item $f(y|x=170) = \frac{1}{\sqrt{2\pi\cdot20}}e^{-\frac{1}{2}\frac{(y-71.14)^2}{20}}$
\item $\approx 0$
\end{enumerate}
\item
\begin{enumerate}
\item $0.25$
\item $0.6875$
\item $0.91(6)$
\item $0.75$
\item $-0.28125$
\end{enumerate}
\item
\begin{enumerate}
\item $-1, 0, 1, 1$
\item $-1$
\item $1$
\item $f(x) = \begin{cases}
0, & x < -1 \\
0.25, & -1 \leq x < 0 \\
0.5, & 0 \leq x < 1 \\
1, & x \geq 1
\end{cases}$
\end{enumerate}
\item
\begin{enumerate}
\item $\theta$
\item да
\end{enumerate}

\item
\begin{enumerate}
\item нет, оценка смещена
\item $c = 2$
\end{enumerate}
\item
\begin{enumerate}
\item все оценки несмещенные
\item $\hat{p}_3$ наиболее эффективная
\end{enumerate}
\item да
\item да
\item $\hat{\theta}_{MM} = \sqrt{\frac{\sum_{i=1}^n(X_i-\overline{X})^2\cdot20}{n}}$

\item $\hat{\theta}_{MM} = \frac{1}{5}\left(6 - \frac{1}{n}\sum_{i=1}^n X_i^2 \right)$, $\hat{\theta}_{MM} = 0.68$
\item $\hat{\theta}_{ML} = \frac{\sum_{i=1}^n x_i^2}{n}$
\item $\hat{p}_{ML} = \frac{\sum_{i=1}^n x_i}{n}$
\item да
\item $n_1 \approx 260$, $n_2 \approx 232$, $n_3 \approx 658$

\end{enumerate}


\end{document}
