\documentclass[12pt]{article}

\usepackage{tikz} % картинки в tikz
\usepackage{microtype} % свешивание пунктуации

\usepackage{array} % для столбцов фиксированной ширины

\usepackage{indentfirst} % отступ в первом параграфе

\usepackage{sectsty} % для центрирования названий частей
\allsectionsfont{\centering}

\usepackage{amsmath} % куча стандартных математических плюшек

\usepackage{comment}
\usepackage{amsfonts}

\usepackage[top=2cm, left=1.2cm, right=1.2cm, bottom=2cm]{geometry} % размер текста на странице

\usepackage{lastpage} % чтобы узнать номер последней страницы

\usepackage{enumitem} % дополнительные плюшки для списков
%  например \begin{enumerate}[resume] позволяет продолжить нумерацию в новом списке
\usepackage{caption}


\usepackage{fancyhdr} % весёлые колонтитулы
\pagestyle{fancy}
\lhead{Теория вероятностей}
\chead{}
\rhead{Минимум к контрольной \textnumero 2 по ТВ и МС}
\lfoot{}
\cfoot{}
\rfoot{\thepage/\pageref{LastPage}}
\renewcommand{\headrulewidth}{0.4pt}
\renewcommand{\footrulewidth}{0.4pt}



\usepackage{todonotes} % для вставки в документ заметок о том, что осталось сделать
% \todo{Здесь надо коэффициенты исправить}
% \missingfigure{Здесь будет Последний день Помпеи}
% \listoftodos --- печатает все поставленные \todo'шки


% более красивые таблицы
\usepackage{booktabs}
% заповеди из докупентации:
% 1. Не используйте вертикальные линни
% 2. Не используйте двойные линии
% 3. Единицы измерения - в шапку таблицы
% 4. Не сокращайте .1 вместо 0.1
% 5. Повторяющееся значение повторяйте, а не говорите "то же"



\usepackage{fontspec}
\usepackage{polyglossia}

\setmainlanguage{russian}
\setotherlanguages{english}

% download "Linux Libertine" fonts:
% http://www.linuxlibertine.org/index.php?id=91&L=1
\setmainfont{Linux Libertine O} % or Helvetica, Arial, Cambria
% why do we need \newfontfamily:
% http://tex.stackexchange.com/questions/91507/
\newfontfamily{\cyrillicfonttt}{Linux Libertine O}

\AddEnumerateCounter{\asbuk}{\russian@alph}{щ} % для списков с русскими буквами
\setlist[enumerate, 2]{label=\asbuk*),ref=\asbuk*}

%% эконометрические сокращения
\DeclareMathOperator{\Cov}{Cov}
\DeclareMathOperator{\Corr}{Corr}
\DeclareMathOperator{\Var}{Var}
\DeclareMathOperator{\E}{E}
\def \hb{\hat{\beta}}
\def \hs{\hat{\sigma}}
\def \htheta{\hat{\theta}}
\def \s{\sigma}
\def \hy{\hat{y}}
\def \hY{\hat{Y}}
\def \v1{\vec{1}}
\def \e{\varepsilon}
\def \he{\hat{\e}}
\def \z{z}
\def \hVar{\widehat{\Var}}
\def \hCorr{\widehat{\Corr}}
\def \hCov{\widehat{\Cov}}
\def \cN{\mathcal{N}}
\def \P{\mathbb{P}}


\begin{document}

\section{Теоретический минимум}

\begin{enumerate}
\item Сформулируйте определение независимости событий, формулу полной вероятности.
\item Приведите определение условной вероятности случайного события, формулу Байеса.
\item Сформулируйте определение и свойства функции распределения случайной величины.
\item Сформулируйте определение и свойства функции плотности случайной величины.
\item Сформулируйте определение и свойства математического ожидания для абсолютно непрерывной случайной величины.
\item Сформулируйте определение и свойства математического ожидания для дискретной случайной величины.
\item Сформулируйте определение и свойства дисперсии случайной величины.
\item Сформулируйте определения следующих законов распределений: биномиального, Пуассона, шеометрического, равномерного, экспоненциального, нормального. Укажите математическое ожидание и дисперсию.
\item Сформулируйте определение функции совместного распределения двух случайных величин, независимости случайных величин. Укажите, как связаны совместное распределение и частные распределения компонент случайного вектора.
\item	Сформулируйте определение и свойства совместной функции плотности двух случайных величин, сформулируйте определение независимости случайных величин.
\item Сформулируйте определение и свойства ковариации случайных величин.
\item Сформулируйте определение и свойства корреляции случайных величин.
\item Сформулируйте определение и свойства условной функции плотности.
\item Сформулируйте определение  условного математического ожидания $\E(Y|X=x)$ для совместного дискретного и совместного абсолютно непрерывного распределений.
\item Сформулируйте определение математического ожидания и ковариационной матрицы случайного вектора и их свойства.
\item Сформулируйте неравенство Чебышёва и неравенство Маркова.
\item Сформулируйте закон больших чисел в слабой форме.
\item Сформулируйте центральную предельную теорему.
\item Сформулируйте теорему Муавра—Лапласа.
\item Сформулируйте определение сходимости по вероятности для последовательности случайных величин.

\end{enumerate}


\section{Задачный минимум}

\begin{enumerate}

\item Пусть задана таблица совместного распределения случайных величин $X$ и $Y$.

\begin{center}\begin{tabular}{r|rrr}
\toprule
 $X$ \textbackslash $Y$    & $-1$  & $0$   & $1$   \\ \midrule
$-1$                 & $0.2$ & $0.1$ & $0.2$ \\
 $1$                 & $0.1$ & $0.3$ & $0.1$ \\ \bottomrule
\end{tabular}\end{center}


Найдите
\begin{enumerate}
\item $\P(\{X = -1\})$
\item $\P(\{Y = -1\})$
\item $\P(\{X = -1\}\cap\{Y = -1\})$
\item Являются ли случайные величины $X$ и $Y$ независимыми?
\item $F_{X,Y}(-1,0)$
\item Таблицу распределения случайной величины $X$
\item Функцию $F_{X}(x)$ распределения случайной величины $X$.
\item Постройте график функции $F_{X}(x)$ распределения случайной величины $X$.
\end{enumerate}

\item Пусть задана таблица совместного распределения случайных величин $X$ и $Y$.


\begin{center}\begin{tabular}{l|rrr}
\toprule
 $X$ \textbackslash $Y$    & $-1$  & $0$  & $1$   \\ \midrule
$-1$                 & $0.2$ & $0.1$ & $0.2$ \\
 $1$                 & $0.2$ & $0.1$ & $0.2$ \\ \bottomrule
\end{tabular}\end{center}

Найдите
\begin{enumerate}
\item $\P(X = 1)$,
\item $\P(\{Y = 1\})$,
\item $\P(\{X = 1\}\cap\{Y = 1\})$
\item Являются ли случайные величины $X$ и $Y$ независимыми?
\item $F_{X,Y}$(1,0)
\item Таблицу распределения случайной величины $Y$
\item Функцию $F_{Y}(y)$ распределения случайной величины $Y$
\item Постройте график функции $F_{Y}(y)$ распределения случайной величины $Y$.
\end{enumerate}

\item Пусть задана таблица совместного распределения случайных величин $X$ и $Y$.

\begin{center}\begin{tabular}{l|rrr}
\toprule
 $X$ \textbackslash $Y$    & $-1$  & $0$   & $1$   \\ \midrule
$-1$                 & $0.2$ & $0.1$ & $0.2$ \\
 $1$                 & $0.1$ & $0.3$ & $0.1$ \\ \bottomrule
\end{tabular}\end{center}

Найдите
\begin{enumerate}
    \item $\E(X),$
    \item $\E(X^{2}),$
	\item $\Var(X),$
    \item $\E(Y),$
    \item $\E(Y^{2}),$
    \item $\Var(Y),$
    \item $\E(XY),$
	\item $\Cov(X,Y)$
    \item $\Corr(X,Y)$
    \item Являются ли случайные величины $X$ и $Y$ некоррелированными?
\end{enumerate}

\item Пусть задана таблица совместного распределения случайных величин $X$ и $Y$.

\begin{center}\begin{tabular}{l|rrr}
\toprule
 $X$ \textbackslash $Y$    & $-1$  &$ 0 $  & $1 $  \\ \midrule
$-1$                 & $0.2$ & $0.1$ & $0.2$ \\
 $1$                 & $0.2$ & $0.1$ & $0.2$ \\ \bottomrule
\end{tabular}\end{center}

Найдите
\begin{enumerate}
    \item $\E(X),$
    \item $\E(X^{2}),$
	\item $\Var(X),$
    \item $\E(Y),$
    \item $\E(Y^{2}),$
    \item $\Var(Y),$
    \item $\E(XY),$
	\item $\Cov(X,Y)$
    \item $\Corr(X,Y)$
    \item Являются ли случайные величины $X$ и $Y$ некоррелированными?
\end{enumerate}

\item
Пусть задана таблица совместного распределения случайных величин $X$ и $Y$.

\begin{center}\begin{tabular}{l|rrr}
\toprule
 $X$\textbackslash $Y$    & $-1$  & $0$   & $1$   \\ \midrule
$-1$                 & $0.2$ & $0.1$ & $0.2$ \\
 $1$                 & $0.1$ & $0.3$ & $0.1$ \\ \bottomrule
\end{tabular}\end{center}

Найдите
\begin{enumerate}
\item $\P(\{X = -1\} | \{Y = 0\})$
\item $\P(\{Y = 0\} | \{X = -1\})$
\item таблицу условного распределения случайной величины $Y$ при условии $\{X = -1\}$
\item условное математическое ожидание случайной величины $Y$ при $\{X = -1\}$
\item условную дисперсию случайной величины $Y$
при условии $\{X = -1\}$
\end{enumerate}

\item Пусть задана таблица совместного распределения случайных величин $X$ и $Y$.

\begin{center}\begin{tabular}{l|rrr}
\toprule
 $X$ \textbackslash $Y$    & $-1$  & $0$   & $1$   \\ \midrule
$-1$                 & $0.2$ & $0.1$ & $0.2$ \\
 $1$                 & $0.2$ & $0.1$ & $0.2$ \\ \bottomrule
\end{tabular}\end{center}

Найдите
\begin{enumerate}
\item $\P(\{X = 1\} | \{Y = 0\})$
\item $\P(\{Y = 0\} | \{X = 1\})$
\item таблицу условного распределения случайной величины $Y$ при условии $\{X = 1\}$
\item условное математическое ожидание случайной величины $Y$ при $\{X = 1\}$
\item условную дисперсию случайной величины $Y$
при условии $\{X = 1\}$
\end{enumerate}

\item Пусть $\E(X)=1$, $\E(Y)=2$, $\Var(X) = 3$, $\Var(Y) = 4$, $\Cov(X,Y) = -1$. Найдите
\begin{enumerate}
\item $\E(2X + Y - 4)$
\item $\Var(3Y + 3)$
\item $\Var(X - Y)$
\item $\Var(2X - 3Y +1)$
\item $\Cov(X+ 2Y + 1,3X - Y -1)$
\item $\Corr(X + Y, X - Y)$
\item Ковариационную матрицу случайного вектора $Z = (X\hspace*{0.4cm} Y)$ \end{enumerate}


\item Пусть $\E(X)=-1$, $\E(Y)=2$, $\Var(X) = 1$, $\Var(Y) = 2$, $\Cov(X,Y) = 1$. Найдите
\begin{enumerate}
\item $\E(2X + Y - 4)$
\item $\Var(2Y + 3)$
\item $\Var(X - Y)$
\item $\Var(2X - 3Y +1)$
\item $\Cov(3X+ Y + 1,X - 2Y -1)$
\item $\Corr(X + Y, X - Y)$
\item Ковариационную матрицу случайного вектора $Z = (X\hspace*{0.4cm}Y)$
\end{enumerate}

\item Пусть случайная величина $X$ имеет стандартное нормальное распределение.

Найдите
\begin{enumerate}
\item $\P(\{0 < X < 1\})$
\item $\P(\{X > 2\})$
\item $\P(\{0 < 1 - 2X \leq 1\})$
\end{enumerate}

\item Пусть случайная величина $X$ имеет стандартное нормальное распределение.

Найдите
\begin{enumerate}
\item $\P(\{-1 < X < 1\})$
\item $\P(\{X < -2\})$
\item $\P(\{-2 < -X + 1 \leq 0\})$
\end{enumerate}

\item Пусть случайная величина $X \sim \cN(1,4)$. Найдите $\P(\{1<X<4\})$

\item Пусть случайная величина $X \sim \cN(2,4)$. Найдите $\P(\{-2<X<4\})$

\item Случайные величины $X$ и $Y$ независимы и  имеют нормальное распределение, $\E(X) = 0 $, $\Var(X) = 1$, $\E(Y) = 2$, $\Var(Y) = 6$. Найдите $\P(\{1 < X + 2Y < 7\})$.

\item Случайные величины $X$ и $Y$ независимы и  имеют нормальное распределение, $\E(X) = 0 $, $\Var(X) = 1$, $\E(Y) = 3$, $\Var(Y) = 7$. Найдите $\P(\{1 < 3X + Y < 7\})$.

\item Игральная кость подбрасывается $420$ раз. При помощи центральной предельной теоремы приближенно найти вероятность того, что суммарное число очков будет находиться в пределах от $1400$ до $1505$?

\item При выстреле по мишени стрелок попадает в десятку с вероятностью $0.5$, в девятку – $0.3$, в восьмерку – $0.1$, в семерку – $0.05$, в шестерку – $0.05$.
Стрелок сделал $100$ выстрелов. При помощи центральной предельной теоремы приближенно найти вероятность того, что он набрал не менее 900 очков?

\item Предположим, что на станцию скорой помощи поступают вызовы, число которых распределено по закону Пуассона с параметром $\lambda = 73$, и в разные сутки их количество не зависит друг от друга. При помощи центральной предельной теоремы приближенно найти вероятность того, что в течение года (365 дней) общее число вызовов будет в пределах от $26500$ до $26800$.

\item Число посетителей магазина (в день) имеет распределение Пуассона с математическим ожиданием $289$. При помощи центральной предельной теоремы приближенно найти вероятность того, что за $100$ рабочих дней суммарное число посетителей составит от $28550$ до $29250$ человек.

\item Пусть плотность распределения случайного вектора $(X,Y)$ имеет вид
\begin{center} $f_{X,Y}(x,y) = \begin{cases} x+y, & \text{при } (x,y) \in [0;1] \times [0;1] \\ 0 , & \text{при } (x,y) \not\in [0;1] \times [0;1] \end{cases}$  \end{center}

Найдите
\begin{enumerate}
\item $\P(\{X \leq \frac{1}{2}\} \cap\{Y \leq \frac{1}{2}\})$,
\item $\P(\{X\leq Y\})$,
\item $f_{X}(x)$,
\item $f_{Y}(y)$,
\item Являются ли случайные величины $X$ и $Y$ независимыми?
\end{enumerate}

\item Пусть плотность распределения случайного вектора $(X,Y)$ имеет вид
\begin{center} $f_{X,Y}(x,y) = \begin{cases} 4xy, & \text{при } (x,y) \in [0;1] \times [0;1] \\ 0 , & \text{при } (x,y) \not\in [0;1] \times [0;1] \end{cases}$  \end{center}

Найдите
\begin{enumerate}
\item $\P(\{X \leq \frac{1}{2}\} \cap \{Y \leq \frac{1}{2}\})$,
\item $\P(\{X\leq Y\})$,
\item $f_{X}(x)$,
\item $f_{Y}(y)$,
\item Являются ли случайные величины $X$ и $Y$ независимыми?
\end{enumerate}

\item Пусть плотность распределения случайного вектора $(X,Y)$ имеет вид
\begin{center} $f_{X,Y}(x,y) = \begin{cases} x+y, & \text{при } (x,y) \in [0;1] \times [0;1] \\ 0 , & \text{при } (x,y) \not\in [0;1] \times [0;1] \end{cases}$  \end{center}

Найдите
\begin{enumerate}
\item $\E(X)$,
\item $\E(Y)$,
\item $\E(XY)$,
\item $\Cov(X,Y)$,
\item $\Corr(X,Y)$.
\end{enumerate}

\item Пусть плотность распределения случайного вектора $(X,Y)$ имеет вид
\begin{center} $f_{X,Y}(x,y) = \begin{cases} 4xy, & \text{при } (x,y) \in [0;1] \times [0;1] \\ 0 , & \text{при } (x,y) \not\in [0;1] \times [0;1] \end{cases}$  \end{center}

Найдите
\begin{enumerate}
\item $\E(X)$,
\item $\E(Y)$,
\item $\E(XY)$,
\item $\Cov(X,Y)$,
\item $\Corr(X,Y)$.
\end{enumerate}

\item Пусть плотность распределения случайного вектора $(X,Y)$ имеет вид
\begin{center} $f_{X,Y}(x,y) = \begin{cases} x+y, & \text{при } (x,y) \in [0;1] \times [0;1] \\ 0 , & \text{при } (x,y) \not\in [0;1] \times [0;1] \end{cases}$  \end{center}

Найдите
\begin{enumerate}
\item $f_{Y}(y)$,
\item $f_{X|Y}\left(x|\frac{1}{2}\right)$
\item $\E\left(X|Y = \frac{1}{2}\right)$
\item $\Var\left(X|Y = \frac{1}{2}\right)$
\end{enumerate}

\item Пусть плотность распределения случайного вектора $(X,Y)$ имеет вид
\begin{center} $f_{X,Y}(x,y) = \begin{cases} 4xy, & \text{при } (x,y) \in [0;1] \times [0;1] \\ 0 , & \text{при } (x,y) \not\in [0;1] \times [0;1] \end{cases}$  \end{center}

Найдите
\begin{enumerate}
\item $f_{Y}(y)$,
\item $f_{X|Y}\left(x|\frac{1}{2}\right)$
\item $\E\left(X|Y = \frac{1}{2}\right)$
\item $\Var\left(X|Y = \frac{1}{2}\right)$
\end{enumerate}

\end{enumerate}

\subsection*{Ответы}

\begin{enumerate}

\item
\begin{enumerate}
\item   $0.5 $
\item   $0.3$
\item   $0.2$
\item   нет
\item   $0.3$
\item
\begin{tabular}{lrr}
\toprule
$X$ & $-1$  & $1$   \\ \midrule
$\P(\cdot)$ & $0.5$ & $0.5$ \\ \bottomrule
\end{tabular}
\item  $F_{X}(x) = \begin{cases} 0, & \text{при } x < -1 \\ 0.5 , & \text{при } x \in [-1;1) \\ 1, & \text{при }  x > 1 \end{cases}$
\end{enumerate}
\item
\begin{enumerate}
\item   $0.5$
\item   $0.4$
\item   $0.2$
\item   да
\item   $0.6$
\item
\begin{tabular}{lrrr}
\toprule
$Y$ & $-1$  & $0$   & $1$   \\ \midrule
$\P(\cdot)$ & $0.4$ & $0.2$ & $0.4$ \\ \bottomrule
\end{tabular}
\item   $F_{Y}(y) = \begin{cases} 0, & \text{при } y < -1 \\ 0.4 , & \text{при } y \in [-1;0) \\ 0.6, & \text{при }  y \in [0;1)\\ 1, & \text{при } y > 1 \end{cases}$
\end{enumerate}

\item
\begin{enumerate}
\item   $0$
\item   $1$
\item  $1$
\item   $0$
\item   $0.6$
\item   $0.6$
\item   $0$
\item   $0$
\item   $0$
\item   да, являются некоррелированными, но нельзя утверждать, что являются независимыми
\end{enumerate}

\item
\begin{enumerate}
\item   $0$
\item   $1$
\item   $1$
\item   $0$
\item   $0.8$
\item   $0.8$
\item   $0$
\item   $0$
\item   $0$
\item   да, являются некоррелированными, но нельзя утверждать, что являются независимыми
\end{enumerate}

\item
\begin{enumerate}
\item   $0.25$
\item   $0.2$
\item   \begin{tabular}{lrrr}
\toprule
$Y$ | $\{X = -1\}$ & $-1$  & $0$   & $1$   \\ \midrule
$\P(\cdot)$              & $0.4$ & $0.2$ & $0.4$ \\ \bottomrule
\end{tabular}
\item   $0$
\item   $0.8 $
\end{enumerate}
\item
\begin{enumerate}
\item   $0.5$
\item   $0.2$
\item   \begin{tabular}{lrrr}
\toprule
$Y$ | $\{X = 1\}$ & $-1$  & $0$   & $1$   \\ \midrule
$\P(\cdot)$             & $0.4$ & $0.2$ & $0.4$ \\ \bottomrule
\end{tabular}
\item   $0$
\item   $0.8$
\end{enumerate}

\item
\begin{enumerate}
\item   $0 $
\item   $36$
\item  $9 $
\item   $60 $
\item  $-4$
\item   $\frac{-1}{3\sqrt{5}}$
\item  $\begin{pmatrix}
 3 & -1 \\
-1 & 4
\end{pmatrix}$
\end{enumerate}

\item
\begin{enumerate}
\item $-4$
\item $8 $
\item $1 $
\item $10 $
\item $-6$
\item$ \frac{-1}{\sqrt{5}}$

\item $\begin{pmatrix}
 1 & 1 \\
 1 & 2
\end{pmatrix}$
\end{enumerate}
\item
\begin{enumerate}
\item $0.3413$
\item $0.0228$
\item $0.1915$
\end{enumerate}

\item
\begin{enumerate}
\item $0.6826$
\item $0.0228  $
\item $0.4987  $
\end{enumerate}

\item $0.4332 $
\item $0.1815  $
\item $0.4514 $
\item $0.383  $
\item $0.0558 $
\item $0.5517$
\item $0.0359 $
\item $0.1586$

\item
\begin{enumerate}
\item $0.125   $
\item $0.5 $
\item $f_{X}(x) = \begin{cases} x+\frac{1}{2}, & \text{при } x \in [0;1] \\ 0 , & \text{при } x \not\in [0;1] \end{cases}$
\item $f_{Y}(y) = \begin{cases} y+\frac{1}{2}, & \text{при } y \in [0;1] \\ 0 , & \text{при } y \not\in [0;1] \end{cases}$
\item нет
\end{enumerate}

\item
\begin{enumerate}
\item $\frac{1}{16}$

\item $\frac{1}{2}$

\item$f_{X}(x) = \begin{cases} 2x, & \text{при } x \in [0;1] \\ 0 , & \text{при } x \not\in [0;1] \end{cases}$

\item$f_{Y}(y) = \begin{cases} 2y, & \text{при } y \in [0;1] \\ 0 , & \text{при } y \not\in [0;1] \end{cases}$

\item да
\end{enumerate}

\item
\begin{enumerate}
\item $\frac{7}{12}$

\item $\frac{7}{12}$

\item $\frac{1}{3}$

\item $-\frac{1}{144}$

\item $-\frac{1}{11}$
\end{enumerate}

\item
\begin{enumerate}
\item $\frac{2}{3}$

\item $\frac{2}{3}$

\item $\frac{4}{9}$

\item 0

\item 0
\end{enumerate}

\item
\begin{enumerate}
\item $f_{Y}(y) = \begin{cases} y+\frac{1}{2}, & \text{при } y \in [0;1] \\ 0 , & \text{при } y \not\in [0;1] \end{cases}$

\item $f_{X|Y}(x|\frac{1}{2}) = \begin{cases} x+\frac{1}{2}, & \text{при } x \in [0;1] \\ 0 , & \text{при } x \not\in [0;1] \end{cases}$

\item $\frac{7}{12}$

\item $\frac{11}{144}$
\end{enumerate}

\item
\begin{enumerate}
\item $f_{Y}(y) = \begin{cases} 2y, & \text{при } y \in [0;1] \\ 0 , & \text{при } y \not\in [0;1] \end{cases}$

\item $f_{X|Y}(x|\frac{1}{2}) = \begin{cases} 2x, & \text{при } x \in [0;1] \\ 0 , & \text{при } x \not\in [0;1] \end{cases}$

\item $\frac{2}{3}$

\item $\frac{1}{18}$
\end{enumerate}
\end{enumerate}

\end{document}
