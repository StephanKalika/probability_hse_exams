\documentclass[11pt]{article} % размер шрифта

\usepackage{tikz} % картинки в tikz
\usepackage{microtype} % свешивание пунктуации

\usepackage{array} % для столбцов фиксированной ширины
\usepackage{verbatim} % для вставки комментариев

\usepackage{indentfirst} % отступ в первом параграфе

\usepackage{sectsty} % для центрирования названий частей
\allsectionsfont{\centering} % приказываем центрировать все sections

\usepackage{amsmath, amsfonts} % куча стандартных математических плюшек

\usepackage[top=1.5cm, left=1.5cm, right=1.5cm, bottom=1.5cm]{geometry} % размер текста на странице

\usepackage{lastpage} % чтобы узнать номер последней страницы

\usepackage{enumitem} % дополнительные плюшки для списков
%  например \begin{enumerate}[resume] позволяет продолжить нумерацию в новом списке
\usepackage{caption} % подписи к картинкам без плавающего окружения figure


\usepackage{fancyhdr} % весёлые колонтитулы
\pagestyle{fancy}
\lhead{Теория вероятностей}
\chead{}
\rhead{2017-12-09, контрольная работа 2}
\lfoot{вариант 19}
\cfoot{}
\rfoot{\thepage/\pageref{LastPage}}
\renewcommand{\headrulewidth}{0.4pt}
\renewcommand{\footrulewidth}{0.4pt}



\usepackage{todonotes} % для вставки в документ заметок о том, что осталось сделать
% \todo{Здесь надо коэффициенты исправить}
% \missingfigure{Здесь будет картина Последний день Помпеи}
% команда \listoftodos — печатает все поставленные \todo'шки

\usepackage{booktabs} % красивые таблицы
% заповеди из документации:
% 1. Не используйте вертикальные линии
% 2. Не используйте двойные линии
% 3. Единицы измерения помещайте в шапку таблицы
% 4. Не сокращайте .1 вместо 0.1
% 5. Повторяющееся значение повторяйте, а не говорите "то же"

\usepackage{fontspec} % поддержка разных шрифтов
\usepackage{polyglossia} % поддержка разных языков

\setmainlanguage{russian}
\setotherlanguages{english}

\setmainfont{Linux Libertine O} % выбираем шрифт
% можно также попробовать Helvetica, Arial, Cambria и т.Д.

% чтобы использовать шрифт Linux Libertine на личном компе,
% его надо предварительно скачать по ссылке
% http://www.linuxlibertine.org/index.php?id=91&L=1

\newfontfamily{\cyrillicfonttt}{Linux Libertine O}
% пояснение зачем нужно шаманство с \newfontfamily
% http://tex.stackexchange.com/questions/91507/

\AddEnumerateCounter{\asbuk}{\russian@alph}{щ} % для списков с русскими буквами
\setlist[enumerate, 2]{label=\asbuk*),ref=\asbuk*} % списки уровня 2 будут буквами а) б) ...

%% эконометрические и вероятностные сокращения
\DeclareMathOperator{\Cov}{Cov}
\DeclareMathOperator{\Corr}{Corr}
\DeclareMathOperator{\Var}{Var}
\DeclareMathOperator{\E}{E}
\def \hb{\hat{\beta}}
\def \hs{\hat{\sigma}}
\def \htheta{\hat{\theta}}
\def \s{\sigma}
\def \hy{\hat{y}}
\def \hY{\hat{Y}}
\def \v1{\vec{1}}
\def \e{\varepsilon}
\def \he{\hat{\e}}
\def \z{z}
\def \hVar{\widehat{\Var}}
\def \hCorr{\widehat{\Corr}}
\def \hCov{\widehat{\Cov}}
\def \cN{\mathcal{N}}

\let\P\relax
\DeclareMathOperator{\P}{\mathbb{P}}

 
\begin{document}

\begin{comment}
\section{Это минимумы из которых можно копировать задачи :)}

\subsection{Теоретический минимум}

\begin{enumerate}
\item Сформулируйте определение независимости событий, формулу полной вероятности.
\item Приведите определение условной вероятности случайного события, формулу Байеса.
\item Сформулируйте определение и свойства функции распределения случайной величины.
\item Сформулируйте определение и свойства функции плотности случайной величины.
\item Сформулируйте определение и свойства математического ожидания для абсолютно непрерывной случайной величины.
\item Сформулируйте определение и свойства математического ожидания для дискретной случайной величины.
\item Сформулируйте определение и свойства дисперсии случайной величины.
\item Сформулируйте определения следующих законов распределений: биномиального, Пуассона, геометрического, равномерного, экспоненциального, нормального. Укажите математическое ожидание и дисперсию.
\item Сформулируйте определение функции совместного распределения двух случайных величин, независимости случайных величин. Укажите, как связаны совместное распределение и частные распределения компонент случайного вектора.
\item	Сформулируйте определение и свойства совместной функции плотности двух случайных величин, сформулируйте определение независимости случайных величин.
\item Сформулируйте определение и свойства ковариации случайных величин.
\item Сформулируйте определение и свойства корреляции случайных величин.
\item Сформулируйте определение и свойства условной функции плотности.
\item Сформулируйте определение  условного математического ожидания $\E(Y|X=x)$ для совместного дискретного и совместного абсолютно непрерывного распределений.
\item Сформулируйте определение математического ожидания и ковариационной матрицы случайного вектора и их свойства.
\item Сформулируйте неравенство Чебышёва и неравенство Маркова.
\item Сформулируйте закон больших чисел в слабой форме.
\item Сформулируйте центральную предельную теорему.
\item Сформулируйте теорему Муавра—Лапласа.
\item Сформулируйте определение сходимости по вероятности для последовательности случайных величин.

\end{enumerate}


\subsection{Задачный минимум}

\begin{enumerate}

\item Пусть задана таблица совместного распределения случайных величин $X$ и $Y$.

\begin{center}\begin{tabular}{lccc}
\toprule
 $X$ \textbackslash $Y$    & $-1$  & $0$   & $1$   \\ \midrule
$-1$                 & $0.2$ & $0.1$ & $0.2$ \\
 $1$                 & $0.1$ & $0.3$ & $0.1$ \\ \bottomrule
\end{tabular}\end{center}


Найдите
\begin{enumerate}
\item $\P(X = -1)$
\item $\P(Y = -1)$
\item $\P(X = -1 \cap Y = -1 )$
\item Являются ли случайные величины $X$ и $Y$ независимыми?
\item $F_{X,Y}(-1,0)$
\item Таблицу распределения случайной величины $X$
\item Функцию $F_{X}(x)$ распределения случайной величины $X$.
\item Постройте график функции $F_{X}(x)$ распределения случайной величины $X$.
\end{enumerate}

\item Пусть задана таблица совместного распределения случайных величин $X$ и $Y$.


\begin{center}\begin{tabular}{lccc}
\toprule
 $X$ \textbackslash $Y$    & $-1$  & $0$  & $1$   \\ \midrule
$-1$                 & $0.2$ & $0.1$ & $0.2$ \\
 $1$                 & $0.2$ & $0.1$ & $0.2$ \\ \bottomrule
\end{tabular}\end{center}

Найдите
\begin{enumerate}
\item $\P(X = 1)$,
\item $\P(Y = 1)$,
\item $\P(X = 1 \cap Y = 1)$
\item Являются ли случайные величины $X$ и $Y$ независимыми?
\item $F_{X,Y}(1,0)$
\item Таблицу распределения случайной величины $Y$
\item Функцию $F_{Y}(y)$ распределения случайной величины $Y$
\item Постройте график функции $F_{Y}(y)$ распределения случайной величины $Y$.
\end{enumerate}

\item Пусть задана таблица совместного распределения случайных величин $X$ и $Y$.

\begin{center}
\begin{tabular}{lccc}
\toprule
 $X$ \textbackslash $Y$    & $-1$  & $0$   & $1$   \\ 
 \midrule
$-1$                 & $0.2$ & $0.1$ & $0.2$ \\
 $1$                 & $0.1$ & $0.3$ & $0.1$ \\ 
 \bottomrule
\end{tabular}
\end{center}

Найдите
\begin{enumerate}
    \item $\E(X),$
    \item $\E(X^{2}),$
	\item $\Var(X),$
    \item $\E(Y),$
    \item $\E(Y^{2}),$
    \item $\Var(Y),$
    \item $\E(XY),$
	\item $\Cov(X,Y)$
    \item $\Corr(X,Y)$
    \item Являются ли случайные величины $X$ и $Y$ некоррелированными?
\end{enumerate}

\item Пусть задана таблица совместного распределения случайных величин $X$ и $Y$.

\begin{center}\begin{tabular}{lccc}
\toprule
 $X$ \textbackslash $Y$    & $-1$  &$ 0 $  & $1 $  \\ \midrule
$-1$                 & $0.2$ & $0.1$ & $0.2$ \\
 $1$                 & $0.2$ & $0.1$ & $0.2$ \\ \bottomrule
\end{tabular}\end{center}

Найдите
\begin{enumerate}
    \item $\E(X),$
    \item $\E(X^{2}),$
	\item $\Var(X),$
    \item $\E(Y),$
    \item $\E(Y^{2}),$
    \item $\Var(Y),$
    \item $\E(XY),$
	\item $\Cov(X,Y)$
    \item $\Corr(X,Y)$
    \item Являются ли случайные величины $X$ и $Y$ некоррелированными?
\end{enumerate}

\item
Пусть задана таблица совместного распределения случайных величин $X$ и $Y$.

\begin{center}\begin{tabular}{lccc}
\toprule
 $X$\textbackslash $Y$    & $-1$  & $0$   & $1$   \\ \midrule
$-1$                 & $0.2$ & $0.1$ & $0.2$ \\
 $1$                 & $0.1$ & $0.3$ & $0.1$ \\ \bottomrule
\end{tabular}\end{center}

Найдите
\begin{enumerate}
\item $\P(X = -1 | Y = 0)$
\item $\P(Y = 0 | X = -1)$
\item таблицу условного распределения случайной величины $Y$ при условии $X = -1$
\item условное математическое ожидание случайной величины $Y$ при $X = -1$
\item условную дисперсию случайной величины $Y$
при условии $X = -1$
\end{enumerate}

\item Пусть задана таблица совместного распределения случайных величин $X$ и $Y$.

\begin{center}\begin{tabular}{lccc}
\toprule
 $X$ \textbackslash $Y$    & $-1$  & $0$   & $1$   \\ \midrule
$-1$                 & $0.2$ & $0.1$ & $0.2$ \\
 $1$                 & $0.2$ & $0.1$ & $0.2$ \\ \bottomrule
\end{tabular}\end{center}

Найдите
\begin{enumerate}
\item $\P(X = 1 | Y = 0)$
\item $\P(Y = 0 | X = 1)$
\item таблицу условного распределения случайной величины $Y$ при условии $X = 1$
\item условное математическое ожидание случайной величины $Y$ при $X = 1$
\item условную дисперсию случайной величины $Y$
при условии $X = 1$
\end{enumerate}

\item Пусть $\E(X)=1$, $\E(Y)=2$, $\Var(X) = 3$, $\Var(Y) = 4$, $\Cov(X,Y) = -1$. Найдите
\begin{enumerate}
\item $\E(2X + Y - 4)$
\item $\Var(3Y + 3)$
\item $\Var(X - Y)$
\item $\Var(2X - 3Y +1)$
\item $\Cov(X+ 2Y + 1,3X - Y -1)$
\item $\Corr(X + Y, X - Y)$
\item Ковариационную матрицу случайного вектора $Z = (X\hspace*{0.4cm} Y)$ \end{enumerate}


\item Пусть $\E(X)=-1$, $\E(Y)=2$, $\Var(X) = 1$, $\Var(Y) = 2$, $\Cov(X,Y) = 1$. Найдите
\begin{enumerate}
\item $\E(2X + Y - 4)$
\item $\Var(2Y + 3)$
\item $\Var(X - Y)$
\item $\Var(2X - 3Y +1)$
\item $\Cov(3X+ Y + 1,X - 2Y -1)$
\item $\Corr(X + Y, X - Y)$
\item Ковариационную матрицу случайного вектора $Z = (X\hspace*{0.4cm}Y)$
\end{enumerate}

\item Пусть случайная величина $X$ имеет стандартное нормальное распределение.

Найдите
\begin{enumerate}
\item $\P(0 < X < 1)$
\item $\P(X > 2)$
\item $\P(0 < 1 - 2X \leq 1)$
\end{enumerate}

\item Пусть случайная величина $X$ имеет стандартное нормальное распределение.

Найдите
\begin{enumerate}
\item $\P(-1 < X < 1)$
\item $\P(X < -2)$
\item $\P(-2 < -X + 1 \leq 0)$
\end{enumerate}

\item Пусть случайная величина $X \sim \cN(1,4)$. Найдите $\P(1<X<4)$

\item Пусть случайная величина $X \sim \cN(2,4)$. Найдите $\P(-2<X<4)$

\item Случайные величины $X$ и $Y$ независимы и  имеют нормальное распределение, $\E(X) = 0 $, $\Var(X) = 1$, $\E(Y) = 2$, $\Var(Y) = 6$. Найдите $\P(1 < X + 2Y < 7)$.

\item Случайные величины $X$ и $Y$ независимы и  имеют нормальное распределение, $\E(X) = 0 $, $\Var(X) = 1$, $\E(Y) = 3$, $\Var(Y) = 7$. Найдите $\P(1 < 3X + Y < 7)$.

\item Игральная кость подбрасывается $420$ раз. При помощи центральной предельной теоремы приближенно найти вероятность того, что суммарное число очков будет находиться в пределах от $1400$ до $1505$?

\item При выстреле по мишени стрелок попадает в десятку с вероятностью $0.5$, в девятку – $0.3$, в восьмерку – $0.1$, в семерку – $0.05$, в шестерку – $0.05$.
Стрелок сделал $100$ выстрелов. При помощи центральной предельной теоремы приближенно найти вероятность того, что он набрал не менее 900 очков?

\item Предположим, что на станцию скорой помощи поступают вызовы, число которых распределено по закону Пуассона с параметром $\lambda = 73$, и в разные сутки их количество не зависит друг от друга. При помощи центральной предельной теоремы приближенно найти вероятность того, что в течение года (365 дней) общее число вызовов будет в пределах от $26500$ до $26800$.

\item Число посетителей магазина (в день) имеет распределение Пуассона с математическим ожиданием $289$. При помощи центральной предельной теоремы приближенно найти вероятность того, что за $100$ рабочих дней суммарное число посетителей составит от $28550$ до $29250$ человек.

\item Пусть плотность распределения случайного вектора $(X,Y)$ имеет вид
\begin{center} $f_{X,Y}(x,y) = \begin{cases} x+y, & \text{при } (x,y) \in [0;1] \times [0;1] \\ 0 , & \text{при } (x,y) \not\in [0;1] \times [0;1] \end{cases}$  \end{center}

Найдите
\begin{enumerate}
\item $\P(X \leq \frac{1}{2} \cap Y \leq \frac{1}{2})$,
\item $\P(X\leq Y)$,
\item $f_{X}(x)$,
\item $f_{Y}(y)$,
\item Являются ли случайные величины $X$ и $Y$ независимыми?
\end{enumerate}

\item Пусть плотность распределения случайного вектора $(X,Y)$ имеет вид
\begin{center} $f_{X,Y}(x,y) = \begin{cases} 4xy, & \text{при } (x,y) \in [0;1] \times [0;1] \\ 0 , & \text{при } (x,y) \not\in [0;1] \times [0;1] \end{cases}$  \end{center}

Найдите
\begin{enumerate}
\item $\P(X \leq \frac{1}{2} \cap Y \leq \frac{1}{2})$,
\item $\P(X\leq Y)$,
\item $f_{X}(x)$,
\item $f_{Y}(y)$,
\item Являются ли случайные величины $X$ и $Y$ независимыми?
\end{enumerate}

\item Пусть плотность распределения случайного вектора $(X,Y)$ имеет вид
\begin{center} $f_{X,Y}(x,y) = \begin{cases} x+y, & \text{при } (x,y) \in [0;1] \times [0;1] \\ 0 , & \text{при } (x,y) \not\in [0;1] \times [0;1] \end{cases}$  \end{center}

Найдите
\begin{enumerate}
\item $\E(X)$,
\item $\E(Y)$,
\item $\E(XY)$,
\item $\Cov(X,Y)$,
\item $\Corr(X,Y)$.
\end{enumerate}

\item Пусть плотность распределения случайного вектора $(X,Y)$ имеет вид
\begin{center} $f_{X,Y}(x,y) = \begin{cases} 4xy, & \text{при } (x,y) \in [0;1] \times [0;1] \\ 0 , & \text{при } (x,y) \not\in [0;1] \times [0;1] \end{cases}$  \end{center}

Найдите
\begin{enumerate}
\item $\E(X)$,
\item $\E(Y)$,
\item $\E(XY)$,
\item $\Cov(X,Y)$,
\item $\Corr(X,Y)$.
\end{enumerate}

\item Пусть плотность распределения случайного вектора $(X,Y)$ имеет вид
\begin{center} $f_{X,Y}(x,y) = \begin{cases} x+y, & \text{при } (x,y) \in [0;1] \times [0;1] \\ 0 , & \text{при } (x,y) \not\in [0;1] \times [0;1] \end{cases}$  \end{center}

Найдите
\begin{enumerate}
\item $f_{Y}(y)$,
\item $f_{X|Y}\left(x|\frac{1}{2}\right)$
\item $\E\left(X|Y = \frac{1}{2}\right)$
\item $\Var\left(X|Y = \frac{1}{2}\right)$
\end{enumerate}

\item Пусть плотность распределения случайного вектора $(X,Y)$ имеет вид
\begin{center} $f_{X,Y}(x,y) = \begin{cases} 4xy, & \text{при } (x,y) \in [0;1] \times [0;1] \\ 0 , & \text{при } (x,y) \not\in [0;1] \times [0;1] \end{cases}$  \end{center}

Найдите
\begin{enumerate}
\item $f_{Y}(y)$,
\item $f_{X|Y}\left(x|\frac{1}{2}\right)$
\item $\E\left(X|Y = \frac{1}{2}\right)$
\item $\Var\left(X|Y = \frac{1}{2}\right)$
\end{enumerate}

\end{enumerate}



\newpage
\end{comment}


\section{Минимум}


% 1 + 7 + 12 + 20

\begin{enumerate}
\item Сформулируйте определение независимости событий, формулу полной вероятности.
\item Сформулируйте определение и свойства дисперсии случайной величины.
\item Сформулируйте определение и свойства корреляции случайных величин.
\item Сформулируйте определение сходимости по вероятности для последовательности случайных величин.
\item Задана таблица совместного распределения случайных величин $X$ и $Y$.
\begin{center}
\begin{tabular}{lccc}
\toprule
                       & $Y=-1$  & $Y=0$   & $Y=1$   \\ 
 \midrule
$X=0$                 & $0.1$ & $0.2$ & $0.3$ \\
 $X=1$                 & $0.1$ & $0.2$ & $0.1$ \\ 
 \bottomrule
\end{tabular}
\end{center}


\begin{enumerate}
    \item Найдите $F_{X,Y}(0, 0)$;
    \item Найдите $\E(X)$, $\E(X^2)$, $\E(Y)$, $\E(Y^2)$;
    \item Найдите $\Var(X)$, $\Var(Y)$;
    \item Найдите $\Cov(X, Y)$, $\Corr(X, Y)$
\end{enumerate}    
\item Плотность распределения случайного вектора $(X,Y)$ имеет вид
\[
f_{X,Y}(x,y) = 
\begin{cases} 
\frac{10x+4y}{7}, & \text{при } (x,y) \in [0;1] \times [0;1] \\ 
0 , & \text{при } (x,y) \not\in [0;1] \times [0;1] \\
\end{cases}
\]

\begin{enumerate}
\item Найдите $\P(X \leq Y)$;
\item Найдите функцию плотности $f_X(x)$;
\item Найдите $\E(X)$, $\E(Y)$ и $\Cov(X, Y)$;
\item Являются ли случайные величины $X$ и $Y$ независимыми?
\end{enumerate} 


\end{enumerate}

\section{Задачи}

\begin{enumerate}[resume]

\item Статистика авиакомпании «А» за много лет свидетельствует о том, что 10\% людей, купивших билет на самолет, не являются на рейс. Авиакомпания продала 330 билетов на 300 мест.
\begin{enumerate}
\item Какова вероятность, что всем явившимся на рейс пассажирам хватит места?
\item Укажите наибольшее число билетов, которое можно продавать на 300 мест, чтобы случаи переполнения случались не чаще, чем на одном из десяти рейсов.
\end{enumerate}

\item Сегодня акция компании «Ух» стоит 1 рубль. Каждый день акция может с вероятностью 0.7 вырасти на 1\%, с вероятностью 0.2999 упасть на 1\% и с вероятностью 0.0001 обесцениться (упасть на 100\%).
\begin{enumerate}
\item Считая изменение цены акции независимыми, найдите математическое ожидание её стоимости через 20 торговых дней.
\item Найдите предел по вероятности среднего изменения цены акции в процентах на бесконечном промежутке времени (Ответ обоснуйте).
\item Найдите математическое ожидание цены акции на бесконечном промежутке времени.
\item Инвестор вложил все свои средства в акции компании «Ух». Найдите вероятность его разорения на бесконечном промежутке времени.
\end{enumerate}


\end{enumerate}




\newpage
\setcounter{page}{1}
\setcounter{section}{0}
\lhead{Теория вероятностей}
\chead{}
\rhead{2017-12-09, контрольная работа 2}
\lfoot{вариант 25}
\cfoot{}
\rfoot{\thepage/\pageref{LastPage}}


\section{Минимум}
% 2 + 4 + 14 + 16

\begin{enumerate}
\item Приведите определение условной вероятности случайного события, формулу Байеса.
\item Сформулируйте определение и свойства функции плотности случайной величины. 
\item Сформулируйте определение  условного математического ожидания $\E(Y|X=x)$ для совместного дискретного и совместного абсолютно непрерывного распределений.
\item Сформулируйте неравенство Чебышёва и неравенство Маркова.

\item Задана таблица совместного распределения случайных величин $X$ и $Y$.
\begin{center}
\begin{tabular}{lccc}
\toprule
                       & $Y=-1$  & $Y=0$   & $Y=1$   \\ 
 \midrule
$X=0$                 & $0.2$ & $0.1$ & $0.3$ \\
 $X=1$                 & $0.2$ & $0.1$ & $0.1$ \\ 
 \bottomrule
\end{tabular}
\end{center}


\begin{enumerate}
    \item Найдите $F_{X,Y}(0, 0)$;
    \item Найдите $\E(X)$, $\E(X^2)$, $\E(Y)$, $\E(Y^2)$;
    \item Найдите $\Var(X)$, $\Var(Y)$;
    \item Найдите $\Cov(X, Y)$, $\Corr(X, Y)$
\end{enumerate}    
\item Плотность распределения случайного вектора $(X,Y)$ имеет вид
\[
f_{X,Y}(x,y) = 
\begin{cases} 
\frac{4x+10y}{7}, & \text{при } (x,y) \in [0;1] \times [0;1] \\ 
0 , & \text{при } (x,y) \not\in [0;1] \times [0;1] \\
\end{cases}
\]

\begin{enumerate}
\item Найдите $\P(X \leq Y)$;
\item Найдите функцию плотности $f_X(x)$;
\item Найдите $\E(X)$, $\E(Y)$ и $\Cov(X, Y)$;
\item Являются ли случайные величины $X$ и $Y$ независимыми?
\end{enumerate} 


\end{enumerate}

\section{Задачи}

\begin{enumerate}[resume]

\item Статистика авиакомпании «А» за много лет свидетельствует о том, что 10\% людей, купивших билет на самолет, не являются на рейс. Авиакомпания продала 330 билетов на 300 мест.
\begin{enumerate}
\item Какова вероятность, что всем явившимся на рейс пассажирам хватит места?
\item Укажите наибольшее число билетов, которое можно продавать на 300 мест, чтобы случаи переполнения случались не чаще, чем на одном из десяти рейсов.
\end{enumerate}

\item Сегодня акция компании «Ух» стоит 1 рубль. Каждый день акция может с вероятностью 0.7 вырасти на 1\%, с вероятностью 0.2999 упасть на 1\% и с вероятностью 0.0001 обесцениться (упасть на 100\%).
\begin{enumerate}
\item Считая изменение цены акции независимыми, найдите математическое ожидание её стоимости через 20 торговых дней.
\item Найдите предел по вероятности среднего изменения цены акции в процентах на бесконечном промежутке времени (Ответ обоснуйте).
\item Найдите математическое ожидание цены акции на бесконечном промежутке времени.
\item Инвестор вложил все свои средства в акции компании «Ух». Найдите вероятность его разорения на бесконечном промежутке времени.
\end{enumerate}


\end{enumerate}



\newpage
\setcounter{page}{1}
\setcounter{section}{0}
\lhead{Теория вероятностей}
\chead{}
\rhead{2017-12-09, контрольная работа 2}
\lfoot{вариант 37}
\cfoot{}
\rfoot{\thepage/\pageref{LastPage}}

\section{Минимум}
% 3 + 8(пуасс + биномиальный) + 11 + 19

\begin{enumerate}
\item Сформулируйте определение и свойства функции распределения случайной величины.
\item Сформулируйте определения следующих законов распределений: биномиального и Пуассона. Укажите математическое ожидание и дисперсию.
\item Сформулируйте определение и свойства ковариации случайных величин.
\item Сформулируйте теорему Муавра—Лапласа.

\item Задана таблица совместного распределения случайных величин $X$ и $Y$.
\begin{center}
\begin{tabular}{lccc}
\toprule
                       & $Y=-1$  & $Y=0$   & $Y=1$   \\ 
 \midrule
$X=0$                 & $0.2$ & $0.1$ & $0.1$ \\
 $X=1$                 & $0.2$ & $0.3$ & $0.1$ \\ 
 \bottomrule
\end{tabular}
\end{center}


\begin{enumerate}
    \item Найдите $F_{X,Y}(0, 0)$;
    \item Найдите $\E(X)$, $\E(X^2)$, $\E(Y)$, $\E(Y^2)$;
    \item Найдите $\Var(X)$, $\Var(Y)$;
    \item Найдите $\Cov(X, Y)$, $\Corr(X, Y)$
\end{enumerate}    
\item Плотность распределения случайного вектора $(X,Y)$ имеет вид
\[
f_{X,Y}(x,y) = 
\begin{cases} 
\frac{6x+4y}{5}, & \text{при } (x,y) \in [0;1] \times [0;1] \\ 
0 , & \text{при } (x,y) \not\in [0;1] \times [0;1] \\
\end{cases}
\]

\begin{enumerate}
\item Найдите $\P(X \leq Y)$;
\item Найдите функцию плотности $f_X(x)$;
\item Найдите $\E(X)$, $\E(Y)$ и $\Cov(X, Y)$;
\item Являются ли случайные величины $X$ и $Y$ независимыми?
\end{enumerate} 


\end{enumerate}

\section{Задачи}

\begin{enumerate}[resume]

\item Статистика авиакомпании «А» за много лет свидетельствует о том, что 10\% людей, купивших билет на самолет, не являются на рейс. Авиакомпания продала 330 билетов на 300 мест.
\begin{enumerate}
\item Какова вероятность, что всем явившимся на рейс пассажирам хватит места?
\item Укажите наибольшее число билетов, которое можно продавать на 300 мест, чтобы случаи переполнения случались не чаще, чем на одном из десяти рейсов.
\end{enumerate}

\item Сегодня акция компании «Ух» стоит 1 рубль. Каждый день акция может с вероятностью 0.7 вырасти на 1\%, с вероятностью 0.2999 упасть на 1\% и с вероятностью 0.0001 обесцениться (упасть на 100\%).
\begin{enumerate}
\item Считая изменение цены акции независимыми, найдите математическое ожидание её стоимости через 20 торговых дней.
\item Найдите предел по вероятности среднего изменения цены акции в процентах на бесконечном промежутке времени (Ответ обоснуйте).
\item Найдите математическое ожидание цены акции на бесконечном промежутке времени.
\item Инвестор вложил все свои средства в акции компании «Ух». Найдите вероятность его разорения на бесконечном промежутке времени.
\end{enumerate}


\end{enumerate}



\newpage
\setcounter{page}{1}
\setcounter{section}{0}
\lhead{Теория вероятностей}
\chead{}
\rhead{2017-12-09, контрольная работа 2}
\lfoot{вариант 43}
\cfoot{}
\rfoot{\thepage/\pageref{LastPage}}

\section{Минимум}
% 2 + 8(экспо и равно) + 10 + 17 

\begin{enumerate}
\item Приведите определение условной вероятности случайного события, формулу Байеса.
\item Сформулируйте определения следующих законов распределений: равномерного и экспоненциального. Укажите математическое ожидание и дисперсию.
\item	Сформулируйте определение и свойства совместной функции плотности двух случайных величин, сформулируйте определение независимости случайных величин.
\item Сформулируйте закон больших чисел в слабой форме.

\item Задана таблица совместного распределения случайных величин $X$ и $Y$.
\begin{center}
\begin{tabular}{lccc}
\toprule
                       & $Y=-1$  & $Y=0$   & $Y=1$   \\ 
 \midrule
$X=0$                 & $0.1$ & $0.1$ & $0.2$ \\
 $X=1$                 & $0.1$ & $0.3$ & $0.2$ \\ 
 \bottomrule
\end{tabular}
\end{center}


\begin{enumerate}
    \item Найдите $F_{X,Y}(0, 0)$;
    \item Найдите $\E(X)$, $\E(X^2)$, $\E(Y)$, $\E(Y^2)$;
    \item Найдите $\Var(X)$, $\Var(Y)$;
    \item Найдите $\Cov(X, Y)$, $\Corr(X, Y)$
\end{enumerate}    
\item Плотность распределения случайного вектора $(X,Y)$ имеет вид
\[
f_{X,Y}(x,y) = 
\begin{cases} 
\frac{4x+6y}{5}, & \text{при } (x,y) \in [0;1] \times [0;1] \\ 
0 , & \text{при } (x,y) \not\in [0;1] \times [0;1] \\
\end{cases}
\]

\begin{enumerate}
\item Найдите $\P(X \leq Y)$;
\item Найдите функцию плотности $f_X(x)$;
\item Найдите $\E(X)$, $\E(Y)$ и $\Cov(X, Y)$;
\item Являются ли случайные величины $X$ и $Y$ независимыми?
\end{enumerate} 


\end{enumerate}

\section{Задачи}

\begin{enumerate}[resume]

\item Статистика авиакомпании «А» за много лет свидетельствует о том, что 10\% людей, купивших билет на самолет, не являются на рейс. Авиакомпания продала 330 билетов на 300 мест.
\begin{enumerate}
\item Какова вероятность, что всем явившимся на рейс пассажирам хватит места?
\item Укажите наибольшее число билетов, которое можно продавать на 300 мест, чтобы случаи переполнения случались не чаще, чем на одном из десяти рейсов.
\end{enumerate}

\item Сегодня акция компании «Ух» стоит 1 рубль. Каждый день акция может с вероятностью 0.7 вырасти на 1\%, с вероятностью 0.2999 упасть на 1\% и с вероятностью 0.0001 обесцениться (упасть на 100\%).
\begin{enumerate}
\item Считая изменение цены акции независимыми, найдите математическое ожидание её стоимости через 20 торговых дней.
\item Найдите предел по вероятности среднего изменения цены акции в процентах на бесконечном промежутке времени (Ответ обоснуйте).
\item Найдите математическое ожидание цены акции на бесконечном промежутке времени.
\item Инвестор вложил все свои средства в акции компании «Ух». Найдите вероятность его разорения на бесконечном промежутке времени.
\end{enumerate}


\end{enumerate}



\end{document}
