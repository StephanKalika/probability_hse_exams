
\element{retprobability1}{
  \begin{questionmult}{1}
Случайным образом выбирается семья с двумя детьми. Событие $A$ — в семье старший ребенок — мальчик,  событие $B$ — в семье только один из детей — мальчик, событие $C$ — в семье хотя бы один из детей — мальчик.  Вероятность $\P(C)$ равна
\begin{multicols}{3}
   \begin{choices}
      \correctchoice{$3/4$}
      \wrongchoice{$1/4$}
      \wrongchoice{$1/2$}
      \wrongchoice{$1$}
      \wrongchoice{$2/3$}

       \end{choices}
  \end{multicols}
  \end{questionmult}
}

\element{retprobability1}{
  \begin{questionmult}{2}
Случайным образом выбирается семья с двумя детьми. Событие $A$ — в семье старший ребенок — мальчик,  событие $B$ — в семье только один из детей — мальчик, событие $C$ — в семье хотя бы один из детей — мальчик.  Вероятность $\P(A \cup C)$ равна
   \begin{multicols}{3}
   \begin{choices}
      \correctchoice{$3/4$}
      \wrongchoice{$3/8$}
      \wrongchoice{$2/3$}
      \wrongchoice{$1$}
      \wrongchoice{$1/2$}

       \end{choices}
  \end{multicols}
  \end{questionmult}
}

\element{retprobability1}{
  \begin{questionmult}{3}
Случайным образом выбирается семья с двумя детьми. Событие $A$ — в семье старший ребенок — мальчик,  событие $B$ — в семье только один из детей — мальчик, событие $C$ — в семье хотя бы один из детей — мальчик.  Вероятность $\P(A | C)$ равна
  \begin{multicols}{3}
   \begin{choices}
      \correctchoice{$2/3$}
      \wrongchoice{$1/4$}
      \wrongchoice{$1/2$}
      \wrongchoice{$1$}
      \wrongchoice{$3/4$}

       \end{choices}
  \end{multicols}
  \end{questionmult}
}

\element{retprobability1}{
  \begin{questionmult}{4}
    Случайным образом выбирается семья с двумя детьми. Событие $A$ — в семье старший ребенок — мальчик,  событие $B$ — в семье только один из детей — мальчик, событие $C$ — в семье хотя бы один из детей — мальчик.
    \begin{choices}
      \correctchoice{$A$ и $B$ — независимы, $A$ и $C$ — зависимы, $B$ и $C$ — зависимы}
      \wrongchoice{События $A$, $B$, $C$ — независимы попарно, но зависимы в совокупности}
      \wrongchoice{Любые два события из $A$, $B$, $C$ — зависимы}
      \wrongchoice{События $A$, $B$, $C$ — независимы в совокупности}
      \wrongchoice{$\P(A\cap B\cap C)=\P(A)\P(B)\P(C)$}

    \end{choices}
  \end{questionmult}
}


\element{retprobability1}{
  \begin{questionmult}{5}
Имеется три монетки. Две «правильных» и одна — с «орлами» по обеим сторонам. Вася выбирает одну монетку наугад и подкидывает ее один раз. Вероятность того, что выпадет орел равна
    \begin{multicols}{3}
   \begin{choices}
      \correctchoice{$2/3$}
      \wrongchoice{$1/2$}
      \wrongchoice{$3/5$}
      \wrongchoice{$1/3$}
      \wrongchoice{$2/5$}

       \end{choices}
  \end{multicols}
  \end{questionmult}
}

\element{retprobability}{
  \begin{questionmult}{6}
Имеется три монетки. Две «правильных» и одна — с «орлами» по обеим сторонам. Вася выбирает одну монетку наугад и подкидывает ее один раз. Вероятность того, что была выбрана неправильная монетка, если выпал орел, равна
  \begin{multicols}{3}
   \begin{choices}
      \correctchoice{$1/2$}
      \wrongchoice{$3/5$}
      \wrongchoice{$2/3$}
      \wrongchoice{$1/3$}
       \wrongchoice{$3/2$}

       \end{choices}
  \end{multicols}
  \end{questionmult}
}


\element{retprobability}{
  \begin{questionmult}{7}
Вася бросает 7 правильных игральных кубиков. Наиболее вероятное количество выпавших шестёрок равно
    \begin{multicols}{3}
   \begin{choices}
      \correctchoice{$1$}
      \wrongchoice{$6/7$}
      \wrongchoice{$7/6$}
      \wrongchoice{$0$}
      \wrongchoice{$2$}

       \end{choices}
  \end{multicols}
  \end{questionmult}
}


\element{retprobability}{
  \begin{questionmult}{8}
Вася бросает 7 правильных игральных кубиков. Вероятность того, что ровно на пяти из кубиков выпадет шестёрка равна
    \begin{multicols}{3}
   \begin{choices}
    \correctchoice{$525\left(\frac{1}{6}\right)^7$}
      \wrongchoice{$\frac{7}{12}\left(\frac{1}{6}\right)^5$}
      \wrongchoice{$\frac{525}{12}\left(\frac{1}{6}\right)^7$}
      \wrongchoice{$\left(\frac{1}{6}\right)^5$}
       \wrongchoice{$\left(\frac{1}{6}\right)^7$}

       \end{choices}
  \end{multicols}
  \end{questionmult}
}


\element{retprobability}{
  \begin{questionmult}{9}
Вася бросает 7 правильных игральных кубиков. Математическое ожидание суммы выпавших очков равно
    \begin{multicols}{3}
   \begin{choices}
      \correctchoice{$24.5$}
      \wrongchoice{$7/6$}
      \wrongchoice{$21$}
      \wrongchoice{$30$}
      \wrongchoice{$42$}

      \end{choices}
  \end{multicols}
  \end{questionmult}
}


\element{retprobability}{
  \begin{questionmult}{10}
Вася бросает 7 правильных игральных кубиков. Дисперсия суммы выпавших очков равна
\begin{multicols}{3}
   \begin{choices}
      \correctchoice{$7\cdot\frac{35}{12}$}
      \wrongchoice{$7$}
      \wrongchoice{$7\cdot \frac{35}{36}$}
      \wrongchoice{$7/6$}
      \wrongchoice{$35/36$}

    \end{choices}
  \end{multicols}
  \end{questionmult}
}

\element{retprobability}{
  \begin{questionmult}{11}
Вася бросает 7 правильных игральных кубиков. Пусть величина  $X$ — сумма очков, выпавших на первых двух кубиках, а величина  $Y$ — сумма очков, выпавших на следующих пяти кубиках. Ковариация $\Cov(X,Y)$ равна
\begin{multicols}{3}
   \begin{choices}
      \correctchoice{$0$}
      \wrongchoice{$1$}
      \wrongchoice{$0.5$}
      \wrongchoice{$2/5$}
      \wrongchoice{$-2/5$}

    \end{choices}
  \end{multicols}
  \end{questionmult}
}


\element{retprobability}{
  \begin{questionmult}{12}
Число изюминок в булочке — случайная величина, имеющая распределение Пуассона. Известно, что в среднем каждая булочка содержит 13 изюминок. Вероятность того, что в случайно выбранной булочке окажется только одна изюминка равна:
\begin{multicols}{3}
   \begin{choices}
      \correctchoice{$13e^{-13}$}
      \wrongchoice{$1/13$}
      \wrongchoice{$e^{-13}$}
      \wrongchoice{$e^{-13}/13$}
      \wrongchoice{$e^{13}/13!$}

    \end{choices}
  \end{multicols}
  \end{questionmult}
}

\element{retcommontext}{
%\newpage
\rule{\textwidth}{1pt}
\textbf{В вопросах 13-16} совместное распределение пары величин $X$ и $Y$ задано таблицей:

\begin{tabular}{c|ccc}
 & $Y=-1$ & $Y=0$ & $Y=1$ \\
\hline
$X=-1$ & $1/4$ & $0$  &  $1/4$\\
$X=1$ & $1/6$ & $1/6$ &  $1/6$ \\
\end{tabular}

\vspace{0.5cm}
}

\element{ret1316}{
  \begin{questionmult}{13}
Математическое ожидание случайной величины $X$ при условии, что $Y=-1$ равно
\begin{multicols}{3}
   \begin{choices}
      \correctchoice{$-1/5$}
      \wrongchoice{$-1/12$}
      \wrongchoice{$0$}
      \wrongchoice{$-1/3$}
      \wrongchoice{$1/10$}

    \end{choices}
  \end{multicols}
  \end{questionmult}
}


\element{ret1316}{
  \begin{questionmult}{14}
Вероятность того, что $X=1$ при условии, что $Y<0$ равна
\begin{multicols}{3}
   \begin{choices}
      \correctchoice{$2/5$}
      \wrongchoice{$1/6$}
      \wrongchoice{$1/12$}
      \wrongchoice{$5/12$}
      \wrongchoice{$1/3$}

    \end{choices}
  \end{multicols}
  \end{questionmult}
}

\element{ret1316}{
  \begin{questionmult}{15}
Дисперсия случайной величины $Y$  равна
\begin{multicols}{3}
   \begin{choices}
      \correctchoice{$5/6$}
      \wrongchoice{$5/12$}
      \wrongchoice{$1/3$}
      \wrongchoice{$1/2$}
      \wrongchoice{$12/5$}

    \end{choices}
  \end{multicols}
  \end{questionmult}
}

\element{ret1316}{
  \begin{questionmult}{16}
Ковариация, $\Cov(X,Y)$, равна
\begin{multicols}{3}
   \begin{choices}
      \correctchoice{$0$}
      \wrongchoice{$1$}
      \wrongchoice{$0.5$}
      \wrongchoice{$-0.5$}
      \wrongchoice{$-1$}

    \end{choices}
  \end{multicols}
  \end{questionmult}
}


\element{retcommontext2}{
%\newpage
\rule{\textwidth}{1pt}
\textbf{В вопросах 17-19} функция распределения случайной величины $X$ имеет вид
\[
F(x)=\begin{cases}
0, \; \text{ если } x<0 \\
cx^2, \; \text{ если } x\in [0;1] \\
1, \; \text{ если } x>1
\end{cases}
\]

\vspace{0.5cm}
}


\element{ret1719}{
  \begin{questionmult}{17}
Константа $c$ равна
\begin{multicols}{3}
   \begin{choices}
      \correctchoice{$1$}
      \wrongchoice{$0.5$}
      \wrongchoice{$1.5$}
      \wrongchoice{$2$}
      \wrongchoice{$2/3$}

    \end{choices}
  \end{multicols}
  \end{questionmult}
}

\element{ret1719}{
  \begin{questionmult}{18}
Вероятность того, что величина $X$ примет значение из интервала  $[0.5, 1.5]$ равна
\begin{multicols}{3}
   \begin{choices}
      \correctchoice{$3/4$}
      \wrongchoice{$1$}
      \wrongchoice{$2/3$}
      \wrongchoice{$1/2$}
      \wrongchoice{$3/2$}

    \end{choices}
  \end{multicols}
  \end{questionmult}
}

\element{ret1719}{
  \begin{questionmult}{19}
Математическое ожидание $\E(X)$ равно
\begin{multicols}{3}
   \begin{choices}
      \correctchoice{$2/3$}
      \wrongchoice{$1/4$}
      \wrongchoice{$1/2$}
      \wrongchoice{$3/4$}
      \wrongchoice{$2$}

    \end{choices}
  \end{multicols}
  \end{questionmult}
}

\element{retcommontext3}{
%\newpage
\rule{\textwidth}{1pt}
\textbf{В вопросах 20-23} совместная функция плотности пары $X$ и $Y$ имеет вид
\[
f(x,y)=\begin{cases}
cx^2y^2, \; \text{ если } x\in[0;1], y\in [0;1] \\
0, \; \text{ иначе}
\end{cases}
\]

\vspace{0.5cm}

}

\element{ret2023}{
  \begin{questionmult}{20}
Константа $c$ равна
\begin{multicols}{3}
   \begin{choices}
      \correctchoice{$9$}
      \wrongchoice{$1$}
      \wrongchoice{$1/2$}
      \wrongchoice{$1/4$}
      \wrongchoice{$2$}

    \end{choices}
  \end{multicols}
  \end{questionmult}
}

\element{ret2023}{
  \begin{questionmult}{21}
Вероятность $\P(X<0.5, Y<0.5)$ равна
\begin{multicols}{3}
   \begin{choices}
      \correctchoice{$1/64$}
      \wrongchoice{$1/8$}
      \wrongchoice{$1/16$}
      \wrongchoice{$1/4$}
      \wrongchoice{$9/16$}

    \end{choices}
  \end{multicols}
  \end{questionmult}
}

\element{ret2023}{
  \begin{questionmult}{22}
Условная функция плотности  $f_{X|Y=2}(x)$ равна
\begin{multicols}{2}
   \begin{choices}
      \correctchoice{не определена}
      \wrongchoice{$f_{X|Y=2}(x)=\begin{cases} 9x^2\, \text{ если } x\in [0;1] \\ 0, \text{ иначе }    \end{cases}$}
      \wrongchoice{$f_{X|Y=2}(x)=\begin{cases} 3x^2\, \text{ если } x\in [0;1] \\ 0, \text{ иначе }    \end{cases}$}
      \wrongchoice{$f_{X|Y=2}(x)=\begin{cases} 36x^2\, \text{ если } x\in [0;1] \\ 0, \text{ иначе }    \end{cases}$}
      \wrongchoice{$f_{X|Y=2}(x)=\begin{cases} x^2\, \text{ если } x\in [0;1] \\ 0, \text{ иначе }    \end{cases}$}

    \end{choices}
  \end{multicols}
  \end{questionmult}
}

\element{ret2023}{
  \begin{questionmult}{23}
Математическое ожидание $\E(X/Y)$ равно
\begin{multicols}{3}
   \begin{choices}
      \correctchoice{$9/8$}
      \wrongchoice{$3$}
      \wrongchoice{$1$}
      \wrongchoice{$1/2$}
      \wrongchoice{$2$}

    \end{choices}
  \end{multicols}
  \end{questionmult}
}

\element{retcommontext4}{
%\newpage
\rule{\textwidth}{1pt}
\textbf{В вопросах 24-25} известно, что $\E(X)=1$, $\Var(X)=1$, $\E(Y)=4$, $\Var(Y)=9$, $\Cov(X,Y)=-3$

\vspace{0.5cm}

}

\element{ret2425}{
  \begin{questionmult}{24}
Ковариация $\Cov(2X-Y,X+3Y)$ равна
\begin{multicols}{3}
   \begin{choices}
      \correctchoice{$-40$}
      \wrongchoice{$-18$}
      \wrongchoice{$22$}
      \wrongchoice{$40$}
      \wrongchoice{$18$}

    \end{choices}
  \end{multicols}
  \end{questionmult}
}

\element{ret2425}{
  \begin{questionmult}{25}
Корреляция $\Corr(2X+3,4Y-5)$ равна
\begin{multicols}{3}
   \begin{choices}
      \correctchoice{$-1$}
      \wrongchoice{$-1/8$}
      \wrongchoice{$1/3$}
      \wrongchoice{$1$}
      \wrongchoice{$1/6$}

    \end{choices}
  \end{multicols}
  \end{questionmult}
}


\element{ret2630}{
  \begin{questionmult}{26}
  \AMCnoCompleteMulti
Пусть случайные величины $X$ и $Y$ — независимы, тогда \textbf{НЕ ВЕРНЫМ} является утверждение
\begin{multicols}{2}
   \begin{choices}
      \correctchoice{$\Var(X-Y)<\Var(X)+\Var(Y)$ }
      \wrongchoice{$\Cov(X,Y) = 0$}
      \wrongchoice{$\E(XY)=\E(X)\E(Y)$}
      \wrongchoice{$\E(X|Y)=\E(X)$}
      \wrongchoice{$\P(X<a, Y<b)=\P(X<a)\P(Y<b)$}
      \wrongchoice{$\P(X<a | Y<b)=\P(X<a)$}
    \end{choices}
  \end{multicols}
  \end{questionmult}
}


\element{ret2630}{
  \begin{questionmult}{27}
Если $\E(X)=0$, то, согласно неравенству Чебышева, $\P(|X| \leq 5 \sqrt{\Var(X)})$ лежит в интервале
\begin{multicols}{3}
   \begin{choices}
      \correctchoice{$[0.96;1]$ }
      \wrongchoice{$[0;0.04]$}
      \wrongchoice{$[0;0.2]$}
      \wrongchoice{$[0.8;1]$}
      \wrongchoice{$[0.5;1]$}

    \end{choices}
  \end{multicols}
  \end{questionmult}
}


\element{ret2630}{
  \begin{questionmult}{28}
Пусть $X_1$, $X_2$, \ldots, $X_n$ — последовательность независимых одинаково распределенных случайных величин, $\E(X_i)=3$ и $\Var(X_i)=9$. Следующая величина имеет асимптотически стандартное нормальное распределение
\begin{multicols}{3}
   \begin{choices}
      \correctchoice{ $\sqrt{n}\frac{\bar{X}-3}{3}$ }
      \wrongchoice{$\frac{\bar{X}_n-3}{3}$}
      \wrongchoice{$\frac{\bar{X}_n-3}{3\sqrt{n}}$}
      \wrongchoice{$\frac{X_n-3}{3}$}

      \wrongchoice{$\sqrt{n}(\bar{X}-3)$}
    \end{choices}
  \end{multicols}
  \end{questionmult}
}

\element{ret2630}{
  \begin{questionmult}{29}
  \AMCnoCompleteMulti
Случайная величина $X$ имеет функцию плотности $f(x)=\frac{1}{3\sqrt{2\pi}} \exp\left(-\frac{(x-1)^2}{18} \right)$. Следующее утверждение \textbf{НЕ ВЕРНО}
\begin{multicols}{3}
   \begin{choices}
      \correctchoice{Случайная величина $X$ дискретна }
      \wrongchoice{$\Var(X)=9$}
      \wrongchoice{$\E(X)=1$}
      \wrongchoice{$\P(X>1)=0.5$}
      \wrongchoice{$\P(X=0)=0$}
      \wrongchoice{$\P(X<0)>0$}
    \end{choices}
  \end{multicols}
  \end{questionmult}
}

\element{ret2630}{
  \begin{questionmult}{30}
  \AMCnoCompleteMulti
Пусть $X_1$, $X_2$, \ldots, $X_n$ — последовательность независимых одинаково распределенных случайных величин, $\E(X_i)=\mu$ и $\Var(X_i)=\sigma^2$. Следующее утверждение в общем случае \textbf{НЕ ВЕРНО}:
%\begin{multicols}{3}
   \begin{choices}
      \correctchoice{$\frac{X_n-\mu}{\sigma} \overset{F}{\to} N(0;1)$ при $n\to\infty$ }
      \wrongchoice{$\lim_{n\to\infty} \Var(\bar{X}_n)=0$}
      \wrongchoice{$\bar{X}_n \overset{P}{\to} \mu$ при $n\to\infty$}
      \wrongchoice{$\frac{\bar{X}_n-\mu}{\sigma /\sqrt{n}} \overset{F}{\to } N(0,1) $ при $n\to\infty$}
       \wrongchoice{$\frac{\bar{X}_n-\mu}{\sqrt{n} \sigma } \overset{P}{\to } 0 $ при $n\to\infty$}
      %\wrongchoice{Центральная предельная теорема не применима к последовательности бернуллиевских случайных величин}
       \wrongchoice{$\bar{X}_n-\mu \overset{F}{\to } 0 $ при $n\to\infty$}
    \end{choices}
  %\end{multicols}
  \end{questionmult}
}

\element{retruler}{
\noindent\rule{\textwidth}{1pt}
}

\element{retnewpage}{
\newpage\null
}
