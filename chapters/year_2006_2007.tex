\subsection{Контрольная работа №1, ??.11.2006}

Вывешенное решение может содержать неумышленные опечатки.

Заметил опечатку? Сообщи преподавателю!

\begin{enumerate}
\item  Из~семей, имеющих троих разновозрастных детей, случайным
образом выбирается одна семья. Пусть событие А заключается в~том,
что в~этой семье
старший ребенок — мальчик, В — в~семье есть хотя бы одна девочка.
\begin{enumerate}
\item Считая вероятности рождения мальчиков и девочек одинаковыми,
выяснить, являются ли события А и В независимыми.
\item Изменится ли результат, если вероятности рождения мальчиков и
девочек различны.
\end{enumerate}
\item Студент решает тест (множественного выбора) проставлением
ответов наугад. В~тесте 10 вопросов, на~каждый из~которых 4
варианта ответов. Зачёт ставится в~том случае, если правильных
ответов будет не~менее 5.
\begin{enumerate}
\item Найти вероятность того, что студент правильно ответит только
на~один вопрос
\item Найти наиболее вероятное число правильных ответов
\item Найти математическое ожидание и дисперсию числа правильных
ответов
\item Найти вероятность того, что студент получит зачёт
\end{enumerate}

\item Вероятность изготовления изделия с~браком на~некотором
предприятии равна 0.04. Перед выпуском изделие подвергается
упрощенной проверке, которая в~случае бездефектного изделия
пропускает его с~вероятностью 0.96, а в~случае изделия с~дефектом
- с~вероятностью 0.05. Определить:
\begin{enumerate}
\item Какая часть изготовленных изделий выходит с~предприятия
\item Какова вероятность того, что изделие, прошедшее упрощенную
проверку, бракованное
\end{enumerate}

\item  Вероятность того, что пассажир, купивший билет, не~придет к~отправлению поезда, равна 0.01. Найти вероятность того, что все
400 пассажиров явятся к~отправлению поезда (использовать
приближение Пуассона).

\item Охотник, имеющий 4 патрона, стреляет по~дичи до~первого
попадания или до~израсходования всех патронов. Вероятность
попадания при~первом выстреле равна 0.6, при~каждом последующем -
уменьшается на~0.1. Найти
\begin{enumerate}
\item Закон распределения числа патронов, израсходованных охотником
\item Математическое ожидание и дисперсию этой случайной величины
\end{enumerate}

\item Поезда метрополитена идут регулярно с~интервалом 2 минуты.
Пассажир приходит на~платформу в~случайный момент времени. Какова
вероятность того, что ждать пассажиру придется не~более полминуты.
Найти математическое ожидание и дисперсию времени ожидания поезда.

\item Время работы телевизора «Best» до~первой поломки является
случайной величиной, распределённой по~показательному закону.
Определить вероятность того, что телевизор проработает более 15
лет, если среднее время безотказной работы телевизора фирмы «Best»
составляет 10 лет. Какова вероятность, что телевизор,
проработавший 10 лет, проработает ещё не~менее 15 лет?


\item Дополнительная задача:

Пусть случайные величины $X_{1}$ и $X_{2}$ независимы и равномерно
распределены на отрезках $[-1;1]$ и $[0;1]$, соответственно. Найти
вероятность того, что $\max\{X_{1},X_{2}\}>0.5$, функцию
распределения случайной величины $Y=\max\{X_{1},X_{2}\}$.
\end{enumerate}

\subsection{Контрольная работа №1, ??.11.2006, решения}

\begin{enumerate}
\item
\begin{enumerate}
\item $\P(A)=0.5$, $\P(B)=1-\P(B^{c})=1-0.5^{3}=\frac{7}{8}$, $\P(A\cap
B)=0.5\cdot (1-0.5^{2})=\frac{3}{8}$, $\P(A\cap B)\neq \P(A)\P(B)$,
события зависимы.
\item $\P(A)=p$, $\P(B)=1-p^{3}$, $\P(A\cap B)=p(1-p^{2})$,
независимость событий возможна только при~$p=0$ или $p=1$
\end{enumerate}
\item  Пусть $X$ — число правильных ответов.
\begin{enumerate}
\item $\P(X=1)=C_{10}^{1}\left(\frac{1}{4}\right)^{1}\left(\frac{3}{4}\right)^{9}$
\item $k_{\P(X=k)\rightarrow \max}=\lfloor p(n+1)\rfloor=\lfloor
\frac{11}{4}\rfloor=2$ (можно, не~зная формулы, просто выбрать
наибольшую вероятность)
\item $\E(X)=10\E(X_{i})=\frac{10}{4}$
$\Var(X)=10\Var(X_{i})=10 \cdot \frac{1}{4} \cdot \frac{3}{4}$
\item $\sum_{i=5}^{10}C_{10}^{i}\left(\frac{1}{4}\right)^{i}\left(\frac{3}{4}\right)^{10-i}$
\end{enumerate}
\item $A$ — изделие браковано, $B$ — изделие признано хорошим
\begin{enumerate}
\item $\P(B)=0.96\cdot 0.96+0.04\cdot 0.05$
\item $\P(A|B)=\frac{0.04\cdot 0.05}{\P(B)}$
\end{enumerate}
\item  $\lambda=np=4$

$\P(X=k)=e^{-\lambda}\frac{\lambda^{k}}{k!}$

$\P(X=0)=e^{-4}$
\item
\begin{enumerate}
\item Распределение имеет вид:

\begin{tabular}{@{}ccccc@{}}
\toprule
$x$       & $1$   & $2$               & $3$                           & $4$             \\ \midrule
$\P(X=x)$ & $0.6$ & $(1-0.6)\cdot0.5$ & $(1-0.6)\cdot(1-0.5)\cdot0.4$ & $1-p_1-p_2-p_3$ \\ \bottomrule
\end{tabular}

Упростим:

\begin{tabular}{@{}ccccc@{}}
\toprule
$x$       & $1$   & $2$   & $3$    & $4$    \\ \midrule
$\P(X=x)$ & $0.6$ & $0.2$ & $0.08$ & $0.12$ \\ \bottomrule
\end{tabular}
\item $\E(X)=1.7$, $\Var(X)\approx 1.08$
\end{enumerate}
\item  $\P(X\le 0.5)=\frac{0.5}{2}=0.25$, $\E(X)=\frac{0+2}{2}=1$ (здравый смысл)

$\Var(X)=\E(X^{2})-(\E(X))^{2}$

$\E(X^{2})=\int_{0}^{2}t^{2}\cdot p(t)dt=\int_{0}^{2}t^{2}\cdot 0.5dt=\frac{4}{3}$
\item $\E(X)=10=\frac{1}{\lambda}$, $\lambda=\frac{1}{10}$, $p(t)=\lambda e^{\lambda t}$ при $t>0$

$\P(X>15)=\int_{15}^{\infty}p(t)dt=...=e^{-\frac{3}{2}}$

$\P(X>25|X>10)=\frac{\P(X>25)}{\P(X>10)}=...=e^{-\frac{3}{2}}$
\item Функция распределения:

$F_{Y}(t)=\P(Y\le t)=\P(\max\{X_{1},X_{2}\}\le t)=\P(X_{1}\le t\cap X_{2}\le t)=\P(X_{1}\le t)\P(X_{2}\le t)=\frac{t+1}{2}\cdot t$ при $t\in [0;1]$.

При $t>1$ получаем, что $F_{Y}(t)=1$ и при $t<0$ получаем, что $F_{Y}(t)=0$.

$\P(\max\{X_{1},X_{2}\}>0.5)=1-\P(\max\{X_{1},X_{2}\}\le 0.5)=1-F(0.5)=\frac{5}{8}$
\end{enumerate}

\subsection{Контрольная работа №2, 27.01.2007}

\subsubsection*{Часть I.}

Обведите верный ответ:

\begin{enumerate}
\item Сумма двух нормальных независимых случайных величин нормальна.
Да.
\item Нормальная случайная величина может принимать отрицательные
значения. Да.
\item Пуассоновская случайная величина является непрерывной. Нет.
\item Дисперсия суммы зависимых величин всегда не~меньше суммы
дисперсий. Нет.
\item Теорема Муавра-Лапласа является частным случаем центральной
предельной. Да.
\item Пусть $X$ — длина наугад выловленного удава в~сантиметрах, а
$Y$ - в~дециметрах. Коэффициент корреляции между этими
величинами равен $\frac{1}{10}$. Нет.
\item Математическое ожидание выборочного среднего не~зависит от~объема
выборки, если $X_{i}$ одинаково распределены. Да.
\item Зная закон распределения $X$ и закон распределения $Y$
можно восстановить совместный закон распределения пары $(X,Y)$. Нет.
\item Если  $X$  — непрерывная случайная величина,  $\E\left(X\right)=6$  и
$\Var\left(X\right)=9$ , то  $Y=\frac{X-6}{3} \sim
\cN\left(0;1\right)$.  Нет.
\item Если ты отвечать на~первые 10 вопросов этого теста наугад, то
число правильных ответов - случайная величина, имеющая
биномиальное распределение. Да.
\item По-моему, сегодня хорошая погода, и вместо контрольной можно
было бы покататься на~лыжах. Да!
\end{enumerate}


Любой ответ на 11 считается правильным.

Тест не является блокирующим.

Обозначения:

$\E(X)$ — математическое ожидание

$\Var(X)$ — дисперсия

\subsubsection*{Часть II.}

Стоимость задач 10 баллов.

\begin{enumerate}
% числа выверены
\item Совместный закон распределения случайных величин  $X$  и  $Y$
задан таблицей:

\begin{center}
\begin{tabular}{@{}cccc@{}}
\toprule
    & $Y=-1$ & $Y=0$ & $Y=2$ \\ \midrule
$X=0$ & $0.2$  & $c$   & $0.2$ \\
$X=1$ & $0.1$  & $0.1$ & $0.1$ \\ \bottomrule
\end{tabular}
\end{center}

Найдите  $c$,  $\P\left(Y>-X\right)$,  $\E\left(X\cdot Y^{2} \right)$,  $\E\left(Y|X>0\right)$

% числа выверены
\item Случайный вектор  $\left(\begin{array}{c}
{X_{1} } \\ {X_{2} }
\end{array}\right)$  имеет нормальное распределение с~математическим ожиданием  $\left(\begin{array}{c} {2} \\ {-1}
\end{array}\right)$  и ковариационной матрицей
$\left(\begin{array}{cc} {9} & {-4.5} \\ {-4.5} & {25}
\end{array}\right)$. Найдите  $\P\left(X_{1} +3X_{2} >20\right)$.

% числа выверены
\item Совместная функция плотности имеет вид
\[
p_{X,Y} \left(x,y\right)=
\begin{cases}
x+y, & \text{ если } x\in \left[0;1\right],\, y\in \left[0;1\right] \\
0, & \text{ иначе}
\end{cases}
\]
Найдите  $\P\left(Y>2X\right)$,  $\E\left(X\right)$

% числа выверены
\item В~супермаркете «Покупан» продаются различные вина:

\begin{center}
\begin{tabular}{@{}lccc@{}}
\toprule
Вина    & Доля & Средняя цена за бутылку (у.е.) & Стандартное отклонение (у.е.) \\ \midrule
Элитные & $0.1$  & $150$                           & $24$                           \\
Дорогие & $0.3$  & $40$                             & $12$                            \\
Дешёвые & $0.6$  & $10$                             & $10$                            \\ \bottomrule
\end{tabular}
\end{center}

Чтобы оценить среднюю стоимость предлагаемого вина производится
случайная выборка 10 бутылок.
\begin{enumerate}
\item Какое количество элитных, дорогих и дешёвых вин должно
присутствовать в~выборке, для того, чтобы выборочное среднее
значение цены имело минимальную дисперсию?
\item Чему равна минимальная дисперсия?
\end{enumerate}

% числа выверены
\item Допустим, что закон распределения $X_{n}$ имеет вид:

\begin{tabular}{@{}cccc@{}}
\toprule
$x$      & $-1$     & $0$             & $2$             \\ \midrule
$\P(X=x)$ & $\theta$ & $2\theta - 0.2$ & $1.2 - 3\theta$ \\ \bottomrule
\end{tabular}


Имеется выборка: $X_{1}=0$, $X_{2}=2$.
\begin{enumerate}
\item Найдите оценку $\hat{\theta}$ методом максимального правдоподобия
\item Найдите оценку $\hat{\theta}$ методом моментов
\end{enumerate}

% числа выверены
\item В~среднем 30\% покупателей супермаркета делают покупку на~сумму
свыше 700 рублей. Какова вероятность того, что из 200 $[$случайно
выбранных$]$ покупателей
более 33\% сделают покупку на сумму свыше 700 рублей?

% числа выверены
\item Пусть $X_{i}$ нормально распределены и
независимы. Имеется выборка
из трех наблюдений: 2, 0, 1.
\begin{enumerate}
\item Найдите несмещенные оценки для~математического ожидания и
дисперсии, $\overline{X}$ и $\hat{\sigma}^{2}$.
\item Найдите вероятность того, что оценка дисперсии превосходит
истинную дисперсию более чем в~3 раза
\end{enumerate}

 % числа выверены
\item Известно, что у случайной величины $X$ есть
математическое
ожидание, $\E(X)=0$, и дисперсия.
\begin{enumerate}
\item Укажите верхнюю границу для $\P(X^{2}>4\Var(X))$?
\item Найдите указанную вероятность, если дополнительно известно, что
$X$ нормально распределена.
\end{enumerate}

% числа выверены
\item Пусть $X_{i}$ независимы и экспоненциально
распределены, то есть имеют функцию плотности вида
$p(t)=\frac{1}{\theta}e^{-\frac{1}{\theta}t}$ при $t>0$.
\begin{enumerate}
\item Постройте оценку математического ожидания методом максимального
правдоподобия
\item Является ли оценка несмещенной?
\item Найдите дисперсию оценки
\item С~помощью неравенства Крамера-Рао проверьте, является ли
оценка эффективной среди несмещенных оценок?
\item Является ли построенная оценка состоятельной?
\end{enumerate}


 % числа выверены
\item Независимые случайные величины $X_{i}$ распределены
равномерно на отрезке $[0;a]$, известно, что $a>10$. Исследователь
хочет оценить
параметр $\theta=\frac{1}{\P(X_{i}<5)}$.
\begin{enumerate}
\item Используя $\overline{X}_{n}$ постройте несмещённую оценку
$\hat{\theta}$ для $\theta$
\item Найдите дисперсию построенной оценки
\item Является ли построенная оценка состоятельной?
\end{enumerate}
\end{enumerate}

\subsubsection*{Часть III.}

Стоимость задачи 20 баллов.

Требуется решить \textbf{\underbar{одну}} из двух 11-х задач по
выбору!

\begin{enumerate}
\item[11-А.] Каждый день Кощей Бессмертный кладет в~сундук случайное количество
копеек (от~одной до~ста, равновероятно). Сколько в~среднем дней нужно Кощею, чтобы набралось не~меньше рубля?

\item[11-B.] Каждый день Петя знакомится с~новыми девушками. С~вероятностью 0.7
ему удаётся познакомиться с~одной девушкой; с~вероятностью 0.2 — с
двумя; с~вероятностью 0.1 — не~удаётся. Дни, когда Пете не удаётся
познакомиться ни~с~одной девушкой, Петя считает неудачными.

Какова вероятность, что до~первого неудачного дня Пете удастся
познакомиться ровно с~30-ю девушками?

\emph{Подсказка}: Думайте!
\end{enumerate}

\subsection{Контрольная работа №2, 27.01.2007, решения}

\begin{enumerate}
\item $c=0.3$, $\P(Y>-X)=0.5$, $\E(XY^{2})=0.5$, $\E(Y|X>0)=\frac{0.1}{0.4}=0.25$
\item $\E(Y)=-1$, $\Var(Y)=207$, $\P(Y>20)=\P(Z>\frac{21}{\sqrt{207}})=\P(Z>1.46)=0.07$
\item $\P(Y>2X)=\int_{0}^{1}\int_{0}^{y/2}(x+y)dxdy=\frac{5}{24}$

$\E(X)=\int_{0}^{1}\int_{0}^{1}x(x+y)dxdy=\frac{7}{12}$
\item Используя метод множителей Лагранжа:

$L=\frac{(0.1\cdot 24)^{2}}{a}+\frac{(0.3\cdot 12)^{2}}{b}+\frac{(0.6\cdot 10)^{2}}{c}+\lambda(10-a-b-c)$

\ldots

$a=2$, $b=3$, $c=5$, можно было использовать готовую формулу
$n_{i}=\frac{w_{i}\sigma_{i}}{\sum w_{j}\sigma_{j}}$

$\Var(\overline{X}^{s})=14.4$
\item
\begin{enumerate}
\item $(2\theta-0.2)(1.2-3\theta)\rightarrow\max$

$\hat{\theta}=0.25$
\item $2.4-7\hat{\theta}=1$, $\hat{\theta}=0.2$
\end{enumerate}
\item $\P(\overline{X}>0.33)=\P\left(\frac{\bar{X}-0.3}{\sqrt{\frac{0.3\cdot
0.7}{200}}}>\frac{0.33-0.3}{\sqrt{\frac{0.3\cdot
0.7}{200}}}\right)=\P(Z>1.03)=0.15$
\item $\bar{X}=1$, $\hat{\sigma}^{2}=1$

$\P(\hat{\sigma}^{2}>3\sigma^{2})=\P\left(2\frac{\hat{\sigma}^{2}}{\sigma^{2}}>6\right)=\P(\chi_{2}^{2}>6)=0.05$
\item
\begin{enumerate}
\item $\P(X^{2}>4\Var(X))=\P(|X-0|>2\sigma)\le
\frac{Var{X}}{4\Var(X)}=\frac{1}{4}$
\item $\P(X^{2}>4\Var(X))=\P(|Z|>2)=0.05$
\end{enumerate}
\item
\begin{enumerate}
\item $\overline{X}$
\item Да
\item $\Var(\overline{X})=\frac{\theta^{2}}{n}$;
\item Да: несмещенность и предел дисперсии равный нулю
\end{enumerate}
\item
\begin{enumerate}
\item $\E(\overline{X})=\frac{a}{2}$
$\theta=\frac{1}{\P(X_{i}<5)}=\frac{1}{5/a}=\frac{1}{5}a$

$\hat{\theta}=\frac{2}{5}\overline{X}$
\item $\Var(\hat{\theta}_{n})=(\frac{2}{5})^{2}\cdot\frac{a^{2}}{12n}$
\item $\lim \Var(\hat{\theta}_{n})=0$, оценка несмещённая,
следовательно, состоятельная.
\end{enumerate}
\item[11-А.] Обозначим $e_{n}$ - сколько дней осталось в~среднем ждать, если
уже набрано $n$ копеек.

Тогда:

$e_{100}=0$

$e_{99}=1$

$e_{98}=\frac{1}{100}e_{99}+\frac{99}{100}e_{100}+1=1+\frac{1}{100}$

$e_{97}=\frac{1}{100}e_{98}+\frac{1}{100}e_{99}+\frac{98}{100}e_{100}+1=(1+\frac{1}{100}))^{2}$

$e_{96}=\frac{1}{100}e_{97}+\frac{1}{100}e_{98}+\frac{1}{100}e_{99}+\frac{97}{100}e_{100}+1=(1+\frac{1}{100})^{3}$

\ldots

По~индукции легко доказать, что $e_{n}=(1+\frac{1}{100})^{99-n}$

Таким образом, $e_{0}=(1+\frac{1}{100})^{99}=2.718 \ldots$

\item[11-Б.]  $p_{0}=0.1$, $p_{1}=0.7\cdot 0.1$;

$p_{n}=\P($в~первый день Петя познакомился с~одной
девушкой$)p_{n-1}+\P($в~первый день Петя познакомился с~двумя
девушками$)p_{n-2}$;

Разностное уравнение: $p_{n}=0.7p_{n-1}+0.2p_{n-2}$
\end{enumerate}

\subsection{Контрольная работа №3, 21.02.2007}

Нужные и ненужные формулы: \\ \\
$T$ — сумма чего-то там. \\
Если $H_{0}$ верна, то $\E(T)=\frac{n}{2}$ и $\Var(T)=\frac{n}{4}$ \\ \\
$T$ — сумма каких-то рангов. \\
Если $H_{0}$ верна, то $\E(T)=\frac{n(n+1)}{4}$ и
$\Var(T)=\frac{n(n+1)(2n+1)}{24}$. \\ \\
$T$ — сумма каких-то рангов. \\
Если $H_{0}$ верна, то $\E(T)=\frac{n_{1}(n_{1}+n_{2}+1)}{2}$,
$\Var(T)=\frac{n_{1}n_{2}(n_{1}+n_{2}+1)}{12}$. \\ \\
$\cos^{2}(x)+\sin^{2}(x)=1$ \\

\textbf{УДАЧИ!}

\subsubsection*{Часть I.}

Обведите нужный ответ

\begin{enumerate}
\item Если $X\sim \cN(0;12)$, $Y\sim \cN(12,24)$, $\Corr(X,Y)=0$, то
$X+Y\sim \cN(12,36)$.
Да. Нет.

$[$любой ответ считался правильным. на самом деле верный ответ -
нет$]$

\item Если закон распределения $X$ задан табличкой

\begin{tabular}{@{}ccc@{}}
\toprule
$x$      & $0$   & $1$   \\ \midrule
$\P(X=x)$ & $0.5$ & $0.5$ \\ \bottomrule
\end{tabular}, то $X$ - нормально распределена. Да. Нет.

\item Непараметрические тесты неприменимы, если выборка имеет
$\chi^{2}$ распределение. Да. Нет.
\item P-значение показывает вероятность отвергнуть нулевую
гипотезу, когда она верна. Да. Нет.
\item Если $t$-статистика равна нулю, то P-значение также равно
нулю. Да. Нет.
\item Если гипотеза отвергает при 5\%-ом уровне значимости, то
она будет отвергаться и при 1\%-ом уровне значимости. Да. Нет.
\item При прочих равных 90\% доверительный интервал шире 95\%-го. Да. Нет.
\item Значение функции плотности может превышать единицу. Да. Нет.
\item Для любой случайной величины  $\E(X^{2} )\ge
(\E(X))^{2}$. Да. Нет.
\item Если $\Corr(X,Y)>0$, то $\E(X)\E(Y)<\E(XY)$. Да. Нет.
\item На экзаменационной работе не шутят! Нет, шутят.
\end{enumerate}

Ответ «да» означает истинное утверждение, ответ «нет» — ложное.

Тест не является блокирующим.


\subsubsection*{Часть II.}

Стоимость задач 10 баллов.

\begin{enumerate}
 % числа выверены
\item Из урны с 5 белыми и 7 черными шарами случайным образом вынимается
2 шара. Случайная величина $X$ принимает значение (-1), если оба
шара - белые; 0, если шары разного цвета и 1, если оба шара
черные.
\begin{enumerate}
\item Найдите $\P(X=-1)$, $\E(X)$, $\Var(X)$
\item Постройте функцию распределения величины $X$ $[$достаточно аккуратно выписать функцию$]$
\end{enumerate}

 % числа выверены
\item Случайная величина $X$ имеет функцию распределения
\[
F_{X}(t)=
\begin{cases}
  0, & t<0 \\
  ct^{2}, & 0\le t <1 \\
  1, & 1\le t \\
\end{cases}
\]
\begin{enumerate}
\item Найдите $c$, $\P(0.5<X<2)$, 25\%-ый квантиль
\item Найдите $\E(X)$, $\Var(X)$, $\Cov(X,-X)$, $\Corr(2X,3X)$
\item Выпишите функцию плотности величины $X$
\end{enumerate}

% числа выверены
\item Доходности акций двух компаний являются случайными величинами $X$
и $Y$ с одинаковым математическим ожиданием и ковариационной
матрицей  $\left(%
\begin{array}{cc}
  4 & -2 \\
  -2 & 9 \\
\end{array}%
\right).$
\begin{enumerate}
\item Найдите $\Corr(X,Y)$
\item В какой пропорции нужно приобрести акции этих двух
компаний, чтобы дисперсия доходности получившегося портфеля была наименьшей?
\item Можно ли утверждать, что величины $X+Y$ и $7X-2Y$ независимы?
\item Изменится ли ответ на пункт «в», если дополнительно
известно, что величины $X$ и $Y$ в совокупности нормально распределены?
\end{enumerate}
Подсказка: Если $R$ - доходность портфеля, то $R=\alpha
X+(1-\alpha)Y$

% числа выверены
\item Проверка 40 случайно выбранных лекций показала, что студент
Халявин присутствовал только на двух из них.
\begin{enumerate}
\item Найдите 90\%-ый доверительный интервал для вероятности
увидеть Халявина на лекции.
\item Укажите минимальный размер выборки, необходимый для того,
чтобы с вероятностью 0.9 выборочная доля посещаемых Халявиным
лекций отличалась от соответствующей вероятности не более, чем на 0.1.
\item Какие предпосылки и теоремы использовались при ответах на предыдущие пункты?
\end{enumerate}

% числа выверены
\item Изучается эффективность нового метода обучения. У группы из 40
студентов, обучавшихся по новой методике, средний бал на экзамене
составил 322.12, а выборочное стандартное отклонение 54.53.
Аналогичные показатели для независимой выборки из 60 студентов
того же курса, обучавшихся по старой методике,
приняли значения 304.61 и 62.61 соответственно.
\begin{enumerate}
\item Проверьте гипотезу о равенстве дисперсий оценок в двух
группах.
\item Какие предпосылки использовались при ответе на «а»?
\item Постройте 90\% доверительный интервал для разницы
математических ожиданий оценок в двух группах
\item Можно ли считать новую методику более эффективной?
\end{enumerate}

% числа выверены
\item В парке отдыха за час 57 человек посетило аттракцион «Чертово
колесо», 48 - «Призрачные гонки» и 54 - «Американские горки». Можно ли на 5\% уровне значимости считать, что посетители
одинаково любят эти три аттракциона?

% числа выверены
\item Можно ли по имеющейся таблице утверждать о независимости пола и
доминирующей руки на 5\% уровне значимости?

\begin{tabular}{@{}ccc@{}}
\toprule
Пол / Рука & Правша & Левша \\ \midrule
Мужчины    & $16$     & $76$    \\
Женщина    & $25$     & $81$ \\ \bottomrule
\end{tabular}

% числа выверены
\item Пусть $X_{i}$ нормально распределены, независимы, $\E(X_{i})=0$,
$\Var(X_{i})=\theta$.
\begin{enumerate}
\item Постройте оценку $\hat{\theta}$ методом максимального
правдоподобия.
\item Проверьте свойства несмещенности, состоятельности,
эффективности у построенной оценки.
\item Какая оценка более предпочтительна: построенная или
привычная
$\hat{\sigma}^{2}=\frac{\sum(X_{i}-\bar{X})^{2}}{n-1}$?
\end{enumerate}

% числа выверены
\item Имеются две конкурирующие гипотезы:
\begin{enumerate}
\item[$H_0$:] Случайная величина X распределена равномерно на (0,100)
\item[$H_a$:] Случайная величина X распределена равномерно на (50,150)
\end{enumerate}
Исследователь выбрал следующий критерий: если $X<c$, принимать гипотезу $H_0$, иначе  $H_a$.
\begin{enumerate}
\item Дайте определение ошибок первого и второго рода.
\item Постройте графики зависимостей ошибок первого и второго рода от $c$.
\end{enumerate}

%числа выверены
\item Вася измерил длину 10 пойманных им рыб. Часть рыб была поймана на
левом берегу реки, а часть - на правом. Бывалые рыбаки говорят,
что на правом берегу реки рыба крупнее.

\begin{tabular}{@{}lccccc@{}}
\toprule
Левый берег  & $25$ & $45$ & $37$ & $47$ & $51$ \\
Правый берег & $49$ & $28$ & $39$ & $46$ & $57$ \\ \bottomrule
\end{tabular}
\begin{enumerate}
\item С помощью теста Манна-Уитни (Mann-Whitney) проверьте
гипотезу о том, что выбор берега реки не влияет на среднюю длину
рыбы против
альтернативной гипотезы, что на правом берегу рыба длиннее.

\emph{Разрешается использование нормальной аппроксимации}
\item{} $[$Не оценивался$]$ Возможно ли в этой задаче использовать
(Wilcoxon Signed Rank Test)?
\end{enumerate}
\end{enumerate}

\subsubsection*{Часть III.}

Стоимость задачи 20 баллов.

Требуется решить \textbf{\underbar{одну}} из двух 11-х задач по
выбору!

\begin{enumerate}
\item[11-А.] Имеются две монетки. Одна правильная, другая - выпадает орлом с
вероятностью $0.45$. Одну из них, неизвестно какую, подкинули $n$
раз и сообщили Вам, сколько раз выпал орел. Ваша задача проверить
гипотезу $H_{0}$: «подбрасывалась правильная монетка» против
$H_{a}$:
«подбрасывалась неправильная монетка».

Каким должно быть наименьшее $n$ и критерий выбора гипотезы, чтобы
вероятность ошибок первого рода не превышала 10\%, а вероятность
ошибки второго рода не превышала 15\%?

\item[11-Б.] Время горения лампочки – экспоненциальная случайная величина с
математическим ожиданием равным $\theta $. Вася включил
одновременно 20 лампочек. Величина  $Y$ обозначает время самого
первого перегорания.
\begin{enumerate}
\item Найдите $\E(Y)$
\item Постройте с помощью  $Y$ несмещенную оценку для  $\theta$
\item Сравните по эффективности оценку построенную в пункте
«б» и
обычное выборочное среднее
\end{enumerate}
\end{enumerate}

\subsection{Контрольная работа №3, 21.02.2007, решения}

\begin{enumerate}
\item
\begin{enumerate}
\item $\P(X=-1) = \frac{5}{12}\cdot\frac{4}{11} = \frac{5}{33}$

$\E(X) = -1 \cdot \frac{5}{33} + 0 \cdot 2 \cdot \frac{5}{12}\cdot\frac{7}{11} + 1 \cdot \frac{7}{12}\cdot\frac{6}{11} = \frac{1}{6}$

$\E(X^2) = (-1)^2 \cdot \frac{5}{33} + 1^2 \cdot \frac{7}{12}\cdot\frac{6}{11} = \frac{31}{66}$

$\Var(X) = \frac{175}{396}$
\item $F(x) = \begin{cases}
0, & x < -1 \\
\frac{5}{33}, & -1 \leq x < 0 \\
\frac{15}{22}, & 0 \leq x < 1 \\
1, & x \geq 1
\end{cases}$
\end{enumerate}
\item
\begin{enumerate}
\item $c=1$, $\P(0.5<X<2) = 0.75$, $0.5$
\item $\E(X) = \frac{2}{3}$, $\Var(X) = \frac{1}{18}$, $\Cov(X,-X) = - \frac{1}{18}$,
$\Corr(2X,3X) = 1$
\item $f(x) = \begin{cases}
0, & t < 0, t \geq 1 \\
2t, & 0 \leq t < 1
\end{cases}$
\end{enumerate}
\item
\begin{enumerate}
\item $\Corr(X,Y)=-\frac{1}{3}$
\item $\alpha=\frac{11}{17}$
\item Нет
\item Да
\end{enumerate}
\item
\begin{enumerate}
\item $\hat{p} = \frac{1}{20}$

$\left[\frac{1}{20} - 1.65 \cdot \sqrt{\frac{\frac{1}{20}\cdot\frac{19}{20}}{40}}; \frac{1}{20} + 1.65 \cdot \sqrt{\frac{\frac{1}{20}\cdot\frac{19}{20}}{40}}  \right]$
\item $\P(\vert \hat{p} - p \vert \leq 0.1) = 0.9 \Rightarrow \P\left(\frac{\vert \hat{p} - p \vert}{\sqrt{\frac{\hat{p}(1-\hat{p})}{n}}} \leq \frac{0.1}{\sqrt{\frac{\hat{p}(1-\hat{p})}{n}}} \right) = 0.9 \Rightarrow \frac{0.1}{\sqrt{\frac{\frac{1\cdot19}{20^2}}{n}}} = 1.65 \Rightarrow n \approx 13$
\end{enumerate}
\item
\begin{enumerate}
\item $\gamma_{obs} = \frac{\hat{\sigma}^2_Y}{\hat{\sigma}^2_X} \approx 1.32$, $\gamma_{crit, 0.95} \approx 1.64$,
оснований отвергать $H_0$ нет
\item $X_1, \ldots, X_n \sim \cN(\mu_X, \sigma^2_X)$, $Y_1, \ldots, Y_n \sim \cN(\mu_Y, \sigma^2_Y)$ — независимые выборки
\item $\left[17.51 - 1.66 \cdot 59.4 \sqrt{\frac{1}{40}+ \frac{1}{60}}; 17.51 + 1.66 \cdot 59.4 \sqrt{\frac{1}{40}+ \frac{1}{60}} \right]$

$[-2.61; 37.64]$

$\hat{\sigma}_0^2 = \frac{\hat{\sigma}^2_X(n_X-1) + \hat{\sigma}^2_Y(n_Y-1)}{n_X+n_Y-2} \approx 59.4$
\item Оснований считать новую методику более эффективной нет, так как $0$ входит в доверительный интервал.
\end{enumerate}
\item $\hat{p}_1 = \frac{57}{159}$, $\hat{p}_2 = \frac{48}{159}$, $\hat{p}_3 = \frac{54}{159}$

$Q_{obs} = \sum_{i=1}^n \frac{(n_i - n \cdot p_i)^2}{n \cdot p_i} = \frac{42}{53}$. $Q_{crit} = 3.84$.
Оснований отвергать нулевую гипотезу нет.
\item $\gamma_{obs} = \sum_{i=1}^s \sum_{j=1}^m \frac{\left(n_{ij} - \frac{n_{i\cdot}n_{\cdot j}}{n}\right)^2}{\frac{n_{i\cdot}n_{\cdot j}}{n}} \approx 1.15$,
$\gamma_{crit} = 3.84$, оснований отвергать $H_0$ нет.
\item
\begin{enumerate}
\item \begin{align*}
L(x_1, \ldots, x_n, \theta) &= \prod_{i=1}^n \frac{1}{\sqrt{2\pi\theta}}e^{-\frac{1}{2}\cdot\frac{x_i^2}{\theta}} = \frac{1}{(\sqrt{2\pi\theta})^n} e^{-\frac{1}{2\theta} \sum_{i=1}^n x_i^2} \\
l(x_1, \ldots, x_n, \theta) &= -\frac{n}{2} \ln (2\pi) - \frac{n}{2} \ln \theta -\frac{1}{2\theta} \sum_{i=1}^n x_i^2 \to \max_{\theta} \\
\frac{\partial l}{\partial \theta} &= - \frac{n}{2 \theta} + \frac{1}{2\theta^2} \sum_{i=1}^n x_i^2 \\
\hat{\theta}_{ML} &= \frac{\sum_{i=1}^n x_i^2}{n}
\end{align*}
\item $\E(\hat{\theta}_{ML}) = \E\left(\frac{\sum_{i=1}^n x_i^2}{n}\right) = \frac{1}{n}\cdot n \E(x_1^2) = \theta$,
оценка несмещённая.

$\Var(\hat{\theta}_{ML}) = \Var\left(\frac{\sum_{i=1}^n x_i^2}{n}\right) = \frac{1}{n^2}\cdot n \Var(x_1^2) = \frac{3\theta^2 - \theta^2}{n}\to_{n\to\infty} 0$,
оценка состоятельная.

$\frac{\partial^2 l}{\partial \theta^2} = \frac{n}{2 \theta^2} - \frac{1}{\theta^3}\sum_{i=1}^n x_i^2$

$-\E\left(\frac{\partial^2 l}{\partial \theta^2}\right) = -\frac{n}{2 \theta^2} + \frac{1}{\theta^3} \cdot n \theta = \frac{n}{2 \theta^2}$

$\Var(\hat{\theta}_{ML}) = \frac{2\theta^2}{n} = \frac{1}{\frac{n}{2 \theta^2}} = I(\theta)$,
оценка эффективная.
\end{enumerate}
\item
\begin{enumerate}
\item O1Р: выбрали $H_a$, но верна $H_0$, то есть $X \sim [0, 100]$, но при этом $X \geq c$.

О2Р: выбрали $H_0$, но верна $H_a$, то есть $X \sim [50, 150]$, но при этом $X > c$.
\item $\alpha = \begin{cases}
1, & c < 0 \\
\frac{100-c}{100}, & 0 \leq c \leq 100 \\
0, & c > 100
\end{cases}$

$\beta = \begin{cases}
0, & c < 50 \\
\frac{c-50}{100}, & 50 \leq c \leq 150 \\
1, & c > 150
\end{cases}$
\end{enumerate}
\item $T = 2 + 4 + 6 + 8 + 10 = 30$

$\E(T) = \frac{1}{2}(n_x(n_x+n_y+1)) = 37.5$

$\Var(T) = \frac{1}{12}(n_x n_y(n_x + n_y)) = \frac{75}{6}$

$\gamma_{obs} = \frac{30-37.5}{\sqrt{\frac{75}{6}}} \approx -2.12$,
$\gamma_{crit, 0.05} = -1.65$, основная гипотеза отвергается.
\end{enumerate}
