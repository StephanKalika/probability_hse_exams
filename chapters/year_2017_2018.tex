\subsection{Теоретический минимум к кр1}

\begin{enumerate}
	\item Классическое определение вероятности
	\item Определение условной вероятности
	\item Определение независимости случайных событий
	\item Формула полной вероятности
	\item Формула Байеса
	\item Функция распределения случайной величины. Определение и свойства.
	\item Функция плотности. Определение и свойства.
	\item Математическое ожидание. Определения для дискретного и абсолютно непрерывного случаев. Свойства.
	\item Дисперсия. Определение и свойства.
	\item Законы распределений. Определение, $\E(X)$, $\Var(X)$:
	\begin{enumerate}
	\item Биномиальное распределение
	\item Распределение Пуассона
	\item Геометрическое распределение
	\item Равномерное распределение
	\item Экспоненциальное распределение
	\end{enumerate}
\end{enumerate}


\subsection{Задачный минимум к кр1}




\subsection{Контрольная работа 1, базовый поток, 24.10.2017}

\subsubsection*{Минимум}

\begin{enumerate}
\item Функция распределения случайной величины: определения и свойства.
\item Экспоненциальное распределение: определение, математическое ожидание и дисперсия.
\item В операционном отделе банка работает 80\% опытных сотрудников и 20\% неопытных. Вероятность совершения ошибки при очередной банковской операции опытным сотрудником равна $0.01$, а неопытным — $0.1$. Известно, что при очередной банковской операции была допущена ошибка. Найдите вероятность того, что ошибку допустил неопытный сотрудник.
\item При работе некоторого устройства время от времени возникают сбои. Количество сбоев за сутки имеет распределение Пуассона. Среднее количество сбоев за сутки равно 3. Найдите вероятность того, что за двое суток не произойдет ни одного сбоя.

\end{enumerate}

\subsubsection*{Задачи}

\begin{enumerate}

\item Правильный кубик подбрасывают один раз. Событие $A$ — выпало чётное число, событие $B$ — выпало число кратное трём, событие $C$ — выпало число, большее трёх.

\begin{enumerate}
\item Сформулируйте определение независимости двух событий;
\item Определите, какие из пар событий $A$, $B$ и $C$ будут независимыми.
\end{enumerate}


\item Теоретический минимум (ТМ) состоит из 10 вопросов, задачный (ЗМ) — из 24 задач.
Каждый вариант контрольной содержит два вопроса из ТМ и две задачи из ЗМ. 
Чтобы получить за контрольную работу оценку 4 и выше, необходимо и достаточно правильно ответить на каждый вопрос ТМ и задачу ЗМ доставшегося варианта. Студент Вася принципиально выучил только $k$ вопросов ТМ и две трети ЗМ.
\begin{enumerate}
\item Сколько всего можно составить вариантов, отличающихся хотя бы одним заданием в ТМ или ЗМ части? Порядок заданий внутри варианта не важен.
\item Найдите вероятность того, что Вася правильно решит задачи ЗМ;
\item Дополнительно известно, что Васина вероятность правильно ответить на вопросы ТМ, составляет $1/15$. Сколько вопросов ТМ выучил Вася?
\end{enumerate}

\newpage

\item Производитель молочных продуктов выпустил новый низкокалорийный йогурт Fit и утверждает, что он вкуснее его более калорийного аналога Fat. 
Четырем независимым экспертам предлагают выбрать наиболее вкусный йогурт из трёх, предлагая им в одинаковых стаканчиках в случайном порядке два Fat и один Fit. 
Предположим, что йогурты одинаково привлекательны. 
Величина $\xi$ — число экспертов, отдавших предпочтение Fit.
\begin{enumerate}
\item Какова вероятность, что большинство экспертов выберут Fit?
\item Постройте функцию распределения величины $\xi$;
\item Каково наиболее вероятное число экспертов, отдавших предпочтение йогорту Fit?
\item Вычислите математическое ожидание и дисперсию $\xi$.
\end{enumerate}

\item Дядя Фёдор каждую субботу закупает в магазине продукты по списку, составленному котом Матроскином. Список не изменяется, и в него всегда входит 1 кг сметаны, цена которого является равномерно распределённой величиной $\alpha$, принимающей значения от 250 до 1000 рублей. Стоимость остальных продуктов из списка в тысячах рублей является случайной величиной $\xi$ с функцией распределения

\[
F(x)=\begin{cases}
1-\exp(-x^2 ), \text{ если } x \geq 0 \\
0, \text{ иначе.}\\
\end{cases}
\]
 
\begin{enumerate}
\item Какую сумму должен выделить кот Матроскин дяде Фёдору, чтобы её достоверно хватало на покупку сметаны?  
\item Какую сумму должен выделить кот Матроскин дяде Фёдору, чтобы Дядя Фёдор с вероятностью 0.9 мог оплатить продукты без сметаны? 
\item Найдите математическое ожидание стоимости продуктов без сметаны;
\item Найдите математическое ожидание стоимости всего списка.
\item Какова вероятность того, что общие расходы будут в точности равны их математическому ожиданию?
\end{enumerate}

Подсказка: $\int_0^{\infty} \exp(-x^2) \, dx = \sqrt{\pi} / 2$. 

\item Эксперт с помощью детектора лжи пытается определить, говорит ли подозреваемый правду. Если подозреваемый говорит правду, то эксперт ошибочно выявляет ложь с вероятностью 0.1. Если подозреваемый обманывает, то эксперт выявляет ложь с вероятностью 0.95. 

В деле об одиночном нападении подозревают десять человек, один из которых виновен и будет лгать, остальные невиновны и говорят правду. 

\begin{enumerate}
\item Какова вероятность того, что детектор покажет, что конкретный подозреваемый лжёт?
\item Какова вероятность того, что подозреваемый невиновен, если детектор показал, что он лжёт?
\item Какова вероятность того, что эксперт точно выявит преступника?
\item Какова вероятность того, что эксперт ошибочно выявит  преступника, то есть покажет, что лжёт невиновный, а все остальные говорят правду?
\end{enumerate}



\end{enumerate}



\subsection{Контрольная работа 1, ИП, 24.10.2017}


Ровно 272 года назад императрица Елизавета повелела завезти во дворцы котов для ловли мышей.


\begin{enumerate}

\item В отсутствии кота Леопольда мыши Белый и Серый подкидывают по очереди игральный додекаэдр
%\footnote{Леопольд подсказывает по случаю праздника, что у додекаэдра 12 граней :)}
.
Сыр достаётся тому, кто первым выкинет число 6. Начинает подкидывать Белый.

\begin{enumerate}
  \item Какова вероятность того, что сыр достанется Белому?
  \item Сколько в среднем бросков продолжается игра?
  \item Какова дисперсия числа бросков?
\end{enumerate}

\item Микки Маус, Белый и Серый решили устроить труэль из любви к мышки Мии. Сначала стреляет Микки, затем Белый, затем Серый, затем снова Микки и так до тех пор, пока в живых не останется только один.

Прошлые данные говорят о том, что Микки попадает с вероятностью $1/3$, Белый — с вероятностью $2/3$, а Серый стреляет без промаха.

Найдите оптимальную стратегию каждого мыша.

\item Микки Маус, Белый и Серый пойманый злобным котом Леопольдом до начала труэли. И теперь Леопольд будет играть с ними в странную игру.

В комнате три закрытых внешне неотличимых коробки: с золотом, серебром и платиной. Общаться после начала игры мыши не могут, но могут заранее договориться о стратегии.

Правила игры таковы. Кот Леопольд будет заводить мышей в комнату по очереди. Каждый из мышей может открыть
две коробки по своему выбору. Перед следующим мышом коробки закрываются.

Если Микки откроет коробку с золотом, Белый
— с серебром, а Серый — с платиной, то они выигрывают. Если
хотя бы один из мышей не найдёт свой металл, то Леопольд их съест.
\begin{enumerate}
\item Какова оптимальная стратегия?
\item Какова вероятность выигрыша при использовании оптимальной стратегии?
\end{enumerate}

\item Накануне войны Жестокий Тиран Мышь очень большой страны издал указ. Отныне за каждого новорождённого мыша-мальчика семья получает денежную премию, но если в семье рождается вторая мышка-девочка, то всю семью убивают. Бедные жители страны запуганы и остро нуждаются в деньгах, поэтому в каждой семье мыши будут появляться до тех пор, пока не родится первая мышка-девочка.

\begin{enumerate}
  \item Каким будет среднее число детей в мышиной семье?
  \item Какой будет доля мышей-мальчиков в стране?
  \item Какой будет средняя доля мышей-мальчиков в случайной семье?
  \item Сколько в среднем мышей-мальчиков в случайно выбираемой семье?
\end{enumerate}

\item Вальяжный кот Василий положил на счёт в банке на Гаити один гурд. Сумма на счету растёт непрерывно с постоянной ставкой в течение очень длительного промежутка времени. В случайный момент этого промежутка кот Василий закрывает свой вклад.

Каков закон распределения первой цифры полученной Василием суммы?

\begin{comment}
\item Начинающий трейдер Афанасий совершает не более одной сделки в день.

Если в какой-то день у трейдера Афанасия есть акция, то за этот день он равновероятно продаёт или не продаёт её. Если в какой-то день у трейдера Афанасия нет акций, то он равновероятно покупает или не покупает одну акцию.

Найдите ожидаемую прибыль Афанасия, если известно, что реализовывал он свою стратегию 100 дней, в начале акции стоили по 50 рублей, в конце — по 80 рублей, максимум составил 120 рублей, а минимум — 30.


\item Страшный Мейн-кун разрубает палочку единичной длины на $10$ частей в случайных и независимых местах, равномерно распределённых по всех длине. Затем Страшный Мейн-кун выбирает случайно один из кусочков и возводит его длину в $24$ степень.

Какое в среднем число он получит?
\end{comment}

\end{enumerate}

