\subsection{Контрольная работа №1, 18.10.2005}

\subsubsection*{Часть I.}

\begin{enumerate}
\item Если  $X$ — случайная величина, то  $\Var\left(X\right)=\Var\left(16-X\right)$
\item Функция распределения случайной величины является неубывающей
\item Дисперсия случайной величины не меньше, чем ее стандартное отклонение
\item Для любой случайной величины  $\E\left(X^{2} \right)\ge \left(\E\left(X\right)\right)^{2} $
\item Если ковариация равна нулю, то случайные величины независимы
\item Значение функции плотности может превышать единицу
\item Если события  $A$  и  $B$  не могут произойти одновременно, то они независимы
\item Для любых событий  $A$  и  $B$  верно, что  $\P\left(A|B\right)\ge \P\left(A\cap B\right)$
\item Функция плотности не может быть периодической
\item Для неотрицательной случайной величины  $\E\left(X\right)\ge \E\left(-X\right)$
\item Я ещё не видел части с задачами, но что-то мне уже домой хочется
\end{enumerate}

\subsubsection*{Часть II.}

Стоимость задач 10 баллов.

\begin{enumerate}
\item Шесть студентов, три юноши и три девушки, стоят в очереди за пирожками в случайном порядке. Какова вероятность того, что юноши и девушки чередуются?

\item Имеется три монетки. Две «правильных» и одна — с «орлами» по обеим сторонам. Петя выбирает одну монетку наугад и подкидывает ее два раза. Оба раза выпадает «орел». Какова вероятность того, что монетка «неправильная»?

\item Вася гоняет на мотоцикле по единичной окружности с центром в начале координат. В случайный момент времени он останавливается. Пусть случайные величины  $X$  и  $Y$  — это Васины абсцисса и ордината в момент остановки. Найдите  $\P\left(X>\frac{1}{2} \right)$,  $\P\left(X>\frac{1}{2} |Y<\frac{1}{2} \right)$. Являются ли события  $A=\left\{X>\frac{1}{2} \right\}$  и  $B=\left\{Y<\frac{1}{2} \right\}$  независимыми?

{\it Подсказка: } $\cos\left(\frac{\pi }{3} \right)=\frac{1}{2} $ {\it , длина окружности } $l=2\pi R$

\item В коробке находится четыре внешне одинаковых лампочки. Две из лампочек исправны, две — нет. Лампочки извлекают из коробки по одной до тех пор, пока не будут извлечены обе исправные.
\begin{enumerate}
\item	Какова вероятность того, что опыт закончится извлечением трех лампочек?
\item Каково ожидаемое количество извлеченных лампочек?
\end{enumerate}

\item Два охотника выстрелили в одну утку. Первый попадает с вероятностью 0.4, второй — с вероятностью 0.7. В утку попала ровно одна пуля. Какова вероятность того, что утка была убита первым охотником?

\item
\begin{enumerate}
\item Известно, что  $\E\left(Z\right)=-3$  и  $\E\left(Z^{2} \right)=15$ . Найдите  $\Var\left(Z\right)$ ,  $\Var\left(4-3Z\right)$  и  $\E\left(5+3Z-Z^{2} \right)$.
\item Известно, что  $\Var\left(X+Y\right)=20$  и  $\Var\left(X-Y\right)=10$ . Найдите  $\Cov\left(X,Y\right)$  и  $\Cov\left(6-X,3Y\right)$.
\end{enumerate}

\item Известно, что случайная величина  $X$  принимает три значения. Также известно, что  $\P\left(X=1\right)=0.3$;  $\P\left(X=2\right)=0.1$  и  $\E\left(X\right)=-0.7$.
Определите чему равно третье значение случайной величины  $X$  и найдите  $\Var\left(X\right)$.

\item Известно, что функция плотности случайной величины  $X$  имеет вид:

\[p\left(x\right)=\left\{\begin{array}{l} {cx^{2} ,\quad x\in [-2;2]} \\ {0,\quad x\notin [-2;2]} \end{array}\right. \]

Найдите значение константы  $c$,  $\P\left(X>1\right)$,  $\E\left(X\right)$,  $\E\left(\frac{1}{X^{3} +10} \right)$  и постройте график функции распределения величины  $X$.

\item Бросают два правильных игральных кубика. Пусть  $X$  — наименьшая из выпавших граней, а  $Y$  — наибольшая.
\begin{enumerate}
\item Рассчитайте  $\P\left(X=3\cap Y=5\right)$
\item Найдите  $\E\left(X\right)$ ,  $\Var\left(X\right)$ ,  $\E\left(3X-2Y\right)$
\end{enumerate}

\item Вася решает тест путем проставления каждого ответа наугад. В тесте 5 вопросов. В каждом вопросе 4 варианта ответа. Пусть  $X$  - число правильных ответов,  $Y$  - число неправильных ответов и  $Z=X-Y$ .
\begin{enumerate}
\item Найдите  $\P\left(X>3\right)$
\item Найдите  $\Var\left(X\right)$  и  $\Cov\left(X,Y\right)$
\item Найдите  $\Corr\left(X,Z\right)$
\end{enumerate}

\end{enumerate}

\subsubsection*{Часть III.}

Стоимость задачи 20 баллов.

Требуется решить {\bf \underbar{одну}} из двух 11-х задач по выбору!

\begin{enumerate}
\item[11-А.] Петя сообщает Васе значение случайной величины, равномерно распределенной на отрезке  $[0;4]$ . С вероятностью  $\frac{1}{4} $  Вася возводит Петино число в квадрат, а с вероятностью  $\frac{3}{4} $  прибавляет к Петиному числу 4. Обозначим результат буквой  $Y$.

Найдите  $\P\left(Y<4\right)$  и функцию плотности случайной величины  $Y$.

Вася выбирает свое действие независимо от Петиного числа.

\item[11-Б.] Вы хотите приобрести некую фирму. Стоимость фирмы для ее нынешних владельцев — случайная величина, равномерно распределенная на отрезке [0;1]. Вы предлагаете владельцам продать ее за называемую Вами сумму. Владельцы либо соглашаются, либо нет. Если владельцы согласны, то Вы платите обещанную сумму и получаете фирму. Когда фирма переходит в Ваши руки, ее стоимость сразу возрастает на 20\%.
\begin{enumerate}
\item Чему равен Ваш ожидаемый выигрыш, если Вы предлагаете цену 0.5?
\item Какова оптимальная предлагаемая цена?
\end{enumerate}
\end{enumerate}

\subsection{Контрольная работа №1, 18.10.2005, решения}

\begin{enumerate}
\item $\P(A)=\frac{2\cdot 3!3!}{6!}=1/10$
\item $\P(A|B)=\frac{1/3}{1/3+2/12}=2/3$
\item $\P\left(X>\frac{1}{2} \right) =1/3$,  $\P\left(X>\frac{1}{2} |Y<\frac{1}{2} \right) = 1/4$
\item
\begin{enumerate}
\item
\begin{tabular}{@{}cccc@{}}
\toprule
$x$         & $2$   & $3$   & $4$   \\ \midrule
$\P(X = x)$ & $1/6$ & $1/3$ & $1/2$ \\ \bottomrule
\end{tabular}
\item $\E(X)=3\frac{1}{3}$
\end{enumerate}
\item  $\P(A|B)=\frac{0.4\cdot 0.3 }{0.4\cdot 0.3+0.6\cdot 0.7}$
\item
\begin{enumerate}
\item $\Var(Z)=6$, $\Var(4-3Z)=54$, $\E(5+3Z-Z^2)=-19$
\item $\Cov(X,Y)=2.5$, $\Cov(6-X,3Y)=-7.5$
\end{enumerate}
\item $x=-2$, $\Var(X)=3.1-0.49=2.61$
\item $c=3/16$, $\P(X>1)=13/16$, $\E(X)=0$, $\E(1/(X^3+10))=\frac{3}{8}\ln(3)$,
$F(x)=\begin{cases}
0, \, x<-2 \\
\frac{x^3+8}{16}, \, x\in [-2;2] \\
1, \, x>2
\end{cases}$
\item $\P(X=3 \cap Y=5)=2/36$, $\E(X)=91/36$, $\Var(X)\approx 2.1$, заметим, что $X+Y=R_1+R_2$, поэтому $\E(X)+\E(Y)=7$, и $\E(3X-2Y)=3\E(X)-2\E(Y)=3\E(X)-2(7-\E(X))=5\E(X)-14$
\item $\P(X>3)=61/1024$, $\Var(X)=15/16$, $\Cov(X,Y)=\Cov(X,5-X)=-\Cov(X,X)=-15/16$, $\Corr(X,Z)=\Corr(X,2X-5)=1$


\end{enumerate}

\subsection{Контрольная работа №2, 21.12.2005}


%Отметьте знаком "+" утверждения, которые Вы считаете истинными.

%Отметьте знаком "-" утверждения, которые Вы считаете ложными.

\subsubsection*{Часть I.}

\begin{enumerate}
\item Сумма двух нормальных независимых случайных величин нормальна
\item Сумма любых двух непрерывных случайных величин непрерывна
\item Нормальная случайная величина не может принимать отрицательные значения
\item Пуассоновская случайная величина является непрерывной
\item Сумма двух независимых равномерно распределенных величин равномерна
\item Дисперсия суммы зависимых величин всегда больше суммы дисперсий
\item Дисперсия пуассоновской случайной величины равна ее математическому ожиданию
\item Если  $X$ — непрерывная случайная величина,  $\E\left(X\right)=6$  и  $\Var\left(X\right)=9$ , то  $Y=\frac{X-6}{3} \sim \cN\left(0;1\right)$
\item Теорема Муавра-Лапласа является частным случаем центральной предельной
\item Для любой случайной величины  $\E\left(X|X>0\right)\ge \E\left(X\right)$
\end{enumerate}

\subsubsection*{Часть II.}

Стоимость задач 10 баллов.
\begin{enumerate}
\item Вася, владелец крупного Интернет-портала, вывесил на главной странице рекламный баннер. Ежедневно его страницу посещают 1000 человек. Вероятность того, что посетитель портала кликнет по баннеру равна 0.003. С помощью пуассоновского приближения оцените вероятность того, что за один день не будет ни одного клика по баннеру.
\item Совместный закон распределения случайных величин  $X$  и  $Y$  задан таблицей:

\begin{center}
\begin{tabular}{@{}lccc@{}}
\toprule
    & $Y=-1$ & $Y=0$ & $Y=2$ \\ \midrule
$X=0$ & $0.2$  & $c$   & $0.2$ \\
$X=1$ & $0.1$  & $0.1$ & $0.1$ \\ \bottomrule
\end{tabular}
\end{center}

Найдите  $c$,  $\P\left(Y>-X\right)$,  $\E\left(X\cdot Y^{2} \right)$,  $\E\left(Y|X>0\right)$
\item Случайный вектор  $\left(\begin{array}{cc} {X_{1} } & {X_{2} } \end{array}\right)$  имеет нормальное распределение с математическим ожиданием  $\left(\begin{array}{cc} {2} & {-1} \end{array}\right)$  и ковариационной матрицей  $\left(\begin{array}{cc} {9} & {-4.5} \\ {-4.5} & {25} \end{array}\right)$.
Найдите  $\P\left(X_{1} +3X_{2} >20\right)$.
\item Совместная функция плотности имеет вид
\[
p_{X,Y} \left(x,y\right)=\left\{\begin{array}{l} {x+y,\quad \text{если}\, x\in \left[0;1\right],\, y\in \left[0;1\right]} \\ {0,\quad \text{иначе}} \end{array}\right.
\]
Найдите  $\P\left(Y>X\right)$,  $\E\left(X\right)$,  $\E\left(X|Y>X\right)$.
\item В среднем 20\% покупателей супермаркета делают покупку на сумму свыше 500 рублей. Какова вероятность того, что из 200 покупателей менее 21\% сделают покупку на сумму менее 500 рублей?
\item Вася и Петя метают дротики по мишени. Каждый из них сделал по 100 попыток. Вася оказался метче Пети в 59 попытках. На уровне значимости 5\% проверьте гипотезу о том, что меткость Васи и Пети одинаковая, против альтернативной гипотезы о том, что Вася метче Пети.
\item Найдите  $\P\left(X\in \left[16;23\right]\right)$, если
\begin{enumerate}
\item $X$  нормально распределена,  $\E\left(X\right)=20$,  $\Var\left(X\right)=25$
\item $X$  равномерно распределена на отрезке  $\left[0;30\right]$
\item $X$  распределена экспоненциально и  $\E\left(X\right)=20$
\end{enumerate}
\item Каждый день цена акции равновероятно поднимается или опускается на один рубль. Сейчас акция стоит 1000 рублей. Введем случайную величину  $X_{i} $ , обозначающую изменение курса акции за  $i$-ый день. Найдите  $\E\left(X_{i} \right)$  и  $\Var\left(X_{i} \right)$ . С помощью центральной предельной теоремы найдите вероятность того, что через сто дней акция будет стоить больше 1030 рублей.
\item Определите математическое ожидание и дисперсию случайной величины, если ее функция плотности имеет вид  $p\left(t\right)=c\cdot \exp \left(-2\cdot \left(t+1\right)^{2} \right)$.
\item Пусть случайные величины  $X$  и  $Y$  независимы и распределены по Пуассону с параметрами  $\lambda _{X} =5$  и  $\lambda _{Y} =15$  соответственно. Найдите условное распределение случайной величины  $X$ , если известно, что  $X+Y=50$.
\end{enumerate}

\subsubsection*{Часть III.}

Стоимость задачи 20 баллов.
Требуется решить {\bf \underbar{одну}} из двух 11-х задач по выбору!
\begin{enumerate}
\item[11-А.] Допустим, что оценка  $X$  за экзамен распределена равномерно на отрезке  $\left[0;100\right]$. Итоговая оценка  $Y$  рассчитывается по формуле  $Y=\left\{\begin{array}{l} {0,\quad if\quad X<30} \\ {X,\quad if\quad X\in \left[30;80\right]} \\ {100,\quad if\quad X>80} \end{array}\right. $.

Найдите  $\E\left(Y\right)$,  $\E\left(X\cdot Y\right)$,  $\E\left(Y^{2} \right)$,  $\E\left(Y|Y>0\right)$.

\item[11-Б.] Вася играет в компьютерную игру — «стрелялку-бродилку». По сюжету ему нужно убить 60 монстров. На один выстрел уходит ровно 1 минута. Вероятность убить монстра с одного выстрела равна 0.25. Количество выстрелов не ограничено. Сколько времени в среднем Вася тратит на одного монстра? Найдите дисперсию этого времени? Какова вероятность того, что Вася закончит игру меньше, чем за 3 часа?
\end{enumerate}


\subsection{Контрольная работа №2, 21.12.2005, решения}

\begin{enumerate}
\item $\P(X=0) = e^{-0.003} \approx 0.997$
\item $c=0.3$, $\P(Y > -X) = 0.4$, $\E(X\cdot Y^2) = 0.5$, $\E(Y|X>0) = 1/3$
\item $\P(X_1 +3X_2 > 20) = \P\left(\frac{X_1 +3X_2 +1}{\sqrt{207}}  > \frac{20+1}{\sqrt{207}}\right) = \P(\cN(0,1) > 1.46) = 0.0721$

$\E(X_1 +3X_2) = -1$, $\Var(X_1 +3X_2) = 9 + 9\cdot 25 +6 \cdot(-4.5) = 207$
\item $\P(Y > X) = \int_0^1 \int_x^1 (x+y) dy dx = 0.5$

$\E(X) = \int_0^1 (x+0.5)x dx = 7/12$

% f_{X| Y>X} = \int_x^1 (x+y) dy  = -1.5 x^2 + x + 0.5

% \E(X|Y>X) = \int_0^1 (-1.5 x^2 + x + 0.5)x dx  = 5/24
\item $\P(\hat{p} < 0.21) = \P\left(\frac{\hat{p} - 0.8}{\sqrt{\frac{0.8(1-0.8)}{200}}} < \frac{0.21 - 0.8}{\sqrt{\frac{0.8(1-0.8)}{200}}} \right) \approx 0$ % проверить!
\item
\item
\begin{enumerate}
\item $0.5118$
\item $7/30$
\item $-e^{-\frac{1}{20\cdot 23}} + e^{-\frac{1}{20\cdot 16}} \approx 0.13$
\end{enumerate}
\item Если $S$ — финальная стоимость акции, то $S=1000+X_1+X_2+\ldots+X_{100}$. Тогда по ЦПТ $S\sim \cN(1000,100)$ и $\P(S>1030)\approx 0.001$.
\item $\E(X) = -c \sqrt{\frac{\pi}{2}}$, $\Var(X) = c \frac{5}{4} \frac{\pi}{2} - c^2 \frac{\pi}{2}$
\item
\begin{multline*}
\P(X=k|X+Y=50) = \frac{\P(X+Y =50 | X=k)\P(X=k)}{\P(X+Y=50)} = \frac{\P(Y=50-k)\P(X=k)}{\P(X+Y=50)} = \\
\frac{\left(\frac{e^{-15} 15^{50-k}}{(50-k)!}\right) \left(\frac{e^{-5} 5^k}{k!}\right)}{\frac{e^{-(5+15)}(5+15)^{50}}{50!}} = C_{50}^k \frac{5^k \cdot 15^{50-k}}{(5+15)^{50}} = C_{50}^k \left(\frac{5}{5+15}\right)^k \left(\frac{15}{5+15}\right)^{50-k}
\end{multline*}
Получили биномиальное распределение с параметрами $p=1/4$, $n=50$.
\end{enumerate}


\subsection{Контрольная работа №3, 04.03.2006}

Solution!

Просто из сил выбьешься, пока вдруг как-то само не уладится;
что-то надо подчеркнуть, что-то — выбросить, не договорить, а
где-то — ошибиться, без ошибки такая пакость, что глядеть тошно. \\
В.А. Серов \\

\subsubsection*{Часть I.}

Обведите верный ответ:

\begin{enumerate}
\item Если $X\sim \chi_{n}^{2}$ и $Y\sim \chi_{n+1}^{2}$, $X$ и $Y$ -
независимы, то  $X$ не превосходит $Y$. Нет.
\item В тесте Манна-Уитни предполагается нормальность хотя бы одной
из сравниваемых выборок. Нет.
\item График функции плотности случайной величины, имеющей
$t$-распределение симметричен относительно 0. Да.
\item Мощность больше у того теста, у которого вероятность ошибки
2-го рода меньше. Да.
\item Если $X\sim t_{n}$, то $X^{2}\sim F_{1,n}$. Да.
\item При прочих равных 90\% доверительный интервал шире 95\%-го.  Нет.
\item Несмещенная выборочная оценка дисперсии не превосходит квадрата
выборочного среднего. Нет.
\item Если гипотеза отвергает при 5\%-ом уровне значимости, то она
будет отвергаться и при 1\%-ом уровне значимости. Нет.
\item У t-распределения более толстые «хвосты», чем у стандартного
нормального. Да.
\item P-значение показывает вероятность отвергнуть нулевую гипотезу,
когда она верна. Нет.
\item Если t-статистика равна нулю, то P-значение также равно нулю.
 Нет.
\item Если $X\sim \cN(0;1)$, то $X^{2}\sim \chi_{1}^{2}$. Да.
\item Пусть $X_{i}$ — длина $i$-го удава в сантиметрах, а $Y_{i}$ —
в дециметрах. Выборочный коэффициент корреляции между этими
наборами данных равен $\frac{1}{10}$. Нет.
\item Математическое ожидание выборочного среднего не зависит от
объема выборки, если $X_{i}$ одинаково распределены. Да.
\item Зная закон распределения $X$ и закон распределения $Y$
можно восстановить совместный закон распределения пары $(X,Y)$. Нет.
\item Если ты отвечаешь на вопросы этого теста наугад, то число
правильных ответов — случайная величина, имеющая биномиальное
распределение с дисперсией $4$. Да.
\end{enumerate}

\subsubsection*{Часть II.}

Стоимость задач 10 баллов.

\begin{enumerate}
\item Пусть случайная величина  $X$  распределена
равномерно на отрезке $\left[0;a\right]$, где  $a>3$ .
Исследователь хочет оценить параметр  $\theta =\P\left(X<3\right)$. Рассмотрим следующую оценку

$\hat{\theta
}=\left\{\begin{array}{l} {1,\; X<3} \\ {0,\; X\ge 3}
\end{array}\right. $
\begin{enumerate}
\item Объясните, что означают термины «несмещенность»,
«состоятельность», «эффективность».
\item Верно ли, что оценка $\hat{\theta}$ является несмещенной?
\item Найдите $\E\left(\left(\hat{\theta }-\theta \right)^{2} \right)$.
\end{enumerate}

\item Пусть $X_{1}$, $X_{2}$, ..., $X_{n}$ независимы и их функции
плотности имеет вид:
\[
f(x)=
\begin{cases}
    (k+1)x^{k}, & x \in [0;1]; \\
    0, & x \notin [0;1].
\end{cases}
\]
Найдите оценки параметра $k$:

\begin{enumerate}
\item Методом максимального правдоподобия
\item Методом моментов
\end{enumerate}

\item У 200 человек записали цвет глаз и волос. На уровне значимости
10\% проверьте гипотезу о независимости этих признаков.

\begin{center}
\begin{tabular}{@{}cccc@{}}
\toprule
Цвет глаз / волос & Светлые & Тёмные & Итого \\ \midrule
Зелёные           & 49      & 25     & 74    \\
Другие            & 30      & 96     & 126   \\ \midrule
Итого             & 79      & 121    & 200   \\ \bottomrule
\end{tabular}
\end{center}

\item На курсе два потока, на первом потоке учатся 40 человек, на втором
потоке 50 человек. Средний балл за контрольную на первом потоке
равен 78 при (выборочном) стандартном отклонении в 7 баллов. На
втором потоке средний балл равен 74 при (выборочном) стандартном
отклонении в 8 баллов.
\begin{enumerate}
\item Постройте 90\% доверительный интервал для разницы баллов между двумя потоками
\item На 10\%-ом уровне значимости проверьте гипотезу о том, что результаты контрольной между потоками не отличаются.
\item Рассчитайте точное P-значение (P-value) теста в пункте «б»
\end{enumerate}

\item Предположим, что время жизни лампочки распределено нормально. По
10 лампочкам оценка стандартного отклонения времени жизни
оказалась равной 120 часам.
\begin{enumerate}
\item Найдите 80\%-ый (двусторонний)
доверительный интервал для истинного стандартного отклонения.
\item Допустим, что выборку увеличат до 20 лампочек. Какова
вероятность того, что выборочная оценка дисперсии будет отличаться
от истинной дисперсии меньше, чем на 40\%?
\end{enumerate}

\item Из 10 опрошенных студентов часть предпочитала готовиться по синему
учебнику, а часть — по зеленому. В таблице представлены их
итоговые баллы.

\begin{tabular}{@{}lcccccc@{}}
\toprule
Синий   & 76 & 45 & 57 & 65 &    &    \\
Зелёный & 49 & 59 & 66 & 81 & 38 & 88 \\ \bottomrule
\end{tabular}
\begin{enumerate}
\item С помощью теста Манна-Уитни (Mann-Whitney) проверьте
гипотезу о том, что выбор учебника не меняет закона распределения оценки.

\emph{Разрешается использование нормальной аппроксимации}

\item Возможно ли в этой задаче использовать (Wilcoxon Signed Rank Test)?
\end{enumerate}

\item Вася очень любит играть в преферанс. Предположим, что Васин
выигрыш распределен нормально. За последние 5 партий средний
выигрыш составил 1560 рублей, при оценке стандартного отклонения
равной 670 рублям. Постройте 90\%-ый доверительный интервал для
математического ожидания Васиного выигрыша.

\item Имеется две конкурирующие гипотезы:
\begin{enumerate}
\item[$H_{0}$:] Величина $X$ распределена равномерно на отрезке $[0;100]$
\item[$H_{a}$:] Величина $X$ распределена равномерно на отрезке $[50;150]$
\end{enumerate}
Исследователь выбрал такой критерей: если $X<c$, то использовать $H_{0}$, иначе использовать $H_{a}$.
\begin{enumerate}
\item Что такое «ошибка первого
рода», «ошибка второго рода»,
«мощность теста»?
\item Постройте графики зависимостей ошибок первого и второго рода от $c$.
\end{enumerate}

\item На плоскости выбирается точка со случайными координатами. Абсцисса
и ордината независимы и распределены $\cN(0;1)$. Какова вероятность
того, что расстояние от точки до начала координат будет больше
2.45?

\item С вероятностью 0.3 Вася оставил конспект в одной из 10 посещенных
им сегодня аудиторий. Вася осмотрел 7 из 10 аудиторий и конспекта в них не нашел.
\begin{enumerate}
\item Какова вероятность того, что конспект будет найден в
следующей осматриваемой им аудитории?
\item Какова (условная) вероятность того, что конспект оставлен
где-то в другом месте?
\end{enumerate}
\end{enumerate}

\subsubsection*{Часть III.}

Стоимость задачи 20 баллов.

Требуется решить {\bf \underbar{одну}} из двух 11-х задач по
выбору!

\begin{enumerate}
\item[11-А.] [Hardy-Weinberg theorem]

У диплоидных организмов наследственные характеристики определяются
парой генов. Вспомним знакомые нам с 9-го класса горошины чешского
монаха Менделя. Ген, определяющий форму горошины, имеет две
аллели:  'А' (гладкая) и 'а' (морщинистая). 'А' доминирует 'а'. В
популяции бесконечное количество организмов. Родители каждого
потомка определяются случайным образом, согласно имеющемуся
распределению генотипов. Одна аллель потомка выбирается наугад из
аллелей матери, другая - из аллелей отца. Начальное распределение
генотипов имеет вид: 'АА' - 30\%, 'Аа' - 60\%, 'аа' - 10\%.
\begin{enumerate}
\item Каким будет распределение генотипов в $n$-ом поколении?
\item Заметив закономерность, сформулируйте и докажите теорему
Харди-Вайнберга для произвольного начального распределения
генотипов.
\end{enumerate}

\item[11-Б.] В киосках продается «открытка-подарок». На открытке есть
прямоугольник размером 2 на 7. В каждом столбце в случайном
порядке находятся очередная буква слова «подарок» и звёздочка.
Например, вот так:

\begin{tabular}{|c|c|c|c|c|c|c|}
  \hline
  П & * & * & А & * & О & К \\
  \hline
  * & О & Д & * & Р & * & * \\
  \hline
\end{tabular}

Прямоугольник закрыт защитным слоем, и покупатель не видит, где
буква, а где звёздочка. Следует стереть защитный слой в одном
квадратике в каждом столбце. Можно попытаться угадать любое число
букв. Если открыто $n$ букв слова «подарок» и не открыто ни одной
звёздочки, то открытку можно обменять на $50\cdot 2^{n-1}$ рублей.
Если открыта хотя бы одна звёздочка, то открытка
остается просто открыткой.
\begin{enumerate}
\item Какой стратегии следует придерживаться покупателю, чтобы
максимизировать ожидаемый выигрыш?
\item Чему равен максимальный ожидаемый выигрыш?
\end{enumerate}
\emph{Подсказка}: Думайте!
\end{enumerate}


\subsection{Контрольная работа №3, 04.03.2006, решения}

\begin{enumerate}

\item
\begin{enumerate}
\item
\item $\E(\hat{\theta})=1\cdot \P(X<3)+0\cdot \P(X \ge 3)=\theta$, да является
\item $\E\left(\left(\hat{\theta }-\theta \right)^{2} \right)=\E\left(\hat{\theta}^{2}-2\theta\hat{\theta}+\theta^{2}\right) \stackrel{\hat{\theta}^{2}=\hat{\theta}}{=} \theta-2\theta^{2}+\theta^{2}=\theta-\theta^{2}$
\end{enumerate}
\item
\begin{enumerate}
\item  $L=(k+1)^{n}(x_{1}\cdot x_{2} \cdot...\cdot x_{n})^{k}$

$l=\ln{L}=n\ln(k+1)+k(\sum \ln{x_{i}})$

$\frac{\partial l}{\partial k}=\frac{n}{k+1}+\sum \ln{x_{i}}$

$\frac{n}{\hat{k}+1}+\sum \ln{x_{i}}=0$

$\hat{k}=-\left(1+\frac{n}{\sum \ln{x_{i}}} \right)$
\item  $\E(X_{i})=\int t\cdot p(t)dt=\int_{0}^{1} (k+1)t^{k+1}=\frac{k+1}{k+2}$

$\frac{\hat{k}+1}{\hat{k}+2}=\bar{X}$

$\hat{k}=\frac{2\bar{X}-1}{1-\bar{X}}$
\end{enumerate}
\item $C=\sum \frac{(X_{i,j}-n \hat{p}_{i,j})^{2}}{n\hat{p}_{i,j}}\sim \chi_{(r-1)(c-1)}^{2}$

$C\sim \chi_{1}^{2}$

$C=35$

Если $\alpha=0.1$, то $C_{crit}=2.706$.

Вывод: $H_{0}$ (гипотеза о независимости признаков) отвергается.
\item
\begin{enumerate}
\item Число наблюдений велико, используем нормальное распределение.

$\P\left(-1,65<\frac{\bar{X}-\bar{Y}-\triangle}{\sqrt{\frac{\hat{\sigma}_{x}^{2}}{40}+\frac{\hat{\sigma}_{y}^{2}}{50}}}<1,65\right)=0.9$

$\triangle \in 4 \pm 1.65\sqrt{\frac{49}{40}+\frac{64}{50}}$

$\triangle \in [1.4;6.6]$
\item Используем результат предыдущего пункта: $H_{0}$ отвергается, так как число 0 не входит в доверительный интервал.
\item $Z=2.505$ и $P_{value}=0.0114$
\end{enumerate}
\item
\begin{enumerate}
\item $\chi_{9}^{2}=\frac{9\hat{\sigma}^{2}}{\sigma^{2}} \in [4.17; 14.69]$

$\sigma^{2} \in [8822.3; 31080]$

$\sigma \in [93.9; 176.3] $
\item  $\P(|\hat{\sigma}^{2}-\sigma^{2}|<0.4\sigma^{2})=\P(0.6<\frac{\hat{\sigma}^{2}}{\sigma^{2}}<1.4)=\P(11.4<\chi_{19}^{2}<26.6)\approx 0.8$
\end{enumerate}
\item
\begin{enumerate}
\item $W_{1}=2+4+6+8=20$ или $W_{2}=1+3+5+7+9+10=35$

$U_{1}=10$ или $U_{2}=14$

$Z_{1}=-0.43=-Z_{2}$

Вывод: $H_{0}$ (гипотеза об отсутствии сдвига между законами распределения) не отвергается
\item Нет, т.к. наблюдения не являются парными.
\end{enumerate}
\item  $\P(-2.13<t_{4}<2.13)=0.9$

$\mu \in 1560 \pm 2.13\cdot \sqrt{\frac{670^{2}}{5}}$

$\mu \in [921.8;2198.2]$
\item
\begin{enumerate}
\item
\item  $\P(\text{1 type error})=\P(X>c|X\sim U[0;100])= \left\{
\begin{array}{ll}
  1, & c<0 \\
  1-\frac{c}{100}, & c \in [0;100] \\
  0, & c>100 \\
\end{array}
\right.$

$\P(\text{2 type error})=\P(X<c|X\sim U[50;150])= \left\{
\begin{array}{ll}
  0, & c<50 \\
  \frac{c-50}{100}, & c \in [50;150] \\
  1, & c>150 \\
\end{array}
\right.$

Построение оставлено читателю в качестве самостоятельного
упражнения :)
\end{enumerate}
\item $\P(\sqrt{X^{2}+Y^{2}}>2.45)=\P(X^{2}+Y^{2}>2.45^{2})=\P(\chi_{2}^{2}>6)=0.05$
\item
\begin{enumerate}
\item  $A$ = конспект забыт в 8-ой аудитории

$B$ = конспект был забыт в другом месте (не в аудиториях)

$C$ = конспект не был найден в первых 7-и

$\P(A|C)=\frac{\P(A)}{\P(C)}=\frac{0.3\cdot 0.1}{0.3\cdot 0.3+0.7}=\frac{3}{79}$
\item $\P(B|C)=\frac{\P(B)}{\P(C)}=\frac{0.7}{0.79}=\frac{70}{79}$
\end{enumerate}

\item[11-А.] О чем молчал учебник биологии 9 класса...

Если:

а) ген имеет всего две аллели;

б) в популяции бесконечное число организмов;

в) одна аллель потомка выбирается наугад из аллелей матери, другая
— из аллелей отца;

То распределение генотипов стабилизируется уже в первом поколении (!!!).

То есть $AA_{1}=AA_{2}=...$ и $Aa_{1}=Aa_{2}=...$.

Вероятность получить 'A' от родителя для рождающихся в поколении 1
равна: $p_{1}=0.3\cdot 1+0.6\cdot 0.5 + 0,1\cdot 0=0.6$

В общем виде: $p_{1}=AA_{0}+0.5\cdot Aa_{0}$

$AA_{1}=p_{1}^{2}=0.36$, $Aa_{1}=2p_{1}(1-p_{1})=0.48$.

$p_{2}=AA_{1}+0.5\cdot Aa_{1}=p_{1}^{2}+p_{1}(1-p_{1})=p_{1}$
\item[11-Б.]
\begin{enumerate}
\item Безразлично.
Если я решил попробовать угадать $n$ букв, то выигрыш вырастает, а
вероятность падает в 2 раза по сравнению c попыткой угадать $(n-1)$-у букву.
\item В силу предыдущего пункта: $\E(X)=\frac{1}{2}\cdot 50=25$
\end{enumerate}
\end{enumerate}
