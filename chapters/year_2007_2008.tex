\subsection{Контрольная работа №1, 03.11.2007}

\textbf{Quote}\\
The 50-50-90 rule: Anytime you have a 50-50 chance of getting something right, there's a 90\% probability you'll get it wrong. \\
Andy Rooney\\

\subsubsection*{Часть I.}

Обведите верный ответ:

\begin{enumerate}
\item Для любой случайной величины $\P(X>0)\ge \P(X+1>0)$. Нет.
\item Для любой случайной величины с $\E(X)<2$, выполняется условие $\P(X<2)=1$. Нет.
\item Если $A\subset B$, то $\P(A|B)\le \P(B|A)$. Да.
\item Если  $X$  — случайная величина, то $\E(X)+1=\E(X+1)$. Да.
\item Функция распределения случайной величины является неубывающей. Да.
\item Для любых событий $A$ и $B$, выполняется $\P(A|B)+\P(A|B^{c})=1$. Нет.
\item Для любых событий  $A$  и  $B$  верно, что $\P(A|B)\ge \P(A\cap
B)$, если обе вероятности существуют. Да.
\item Функция плотности может быть периодической. Нет.
\item Если случайная величина $X$ имеет функцию плотности, то $\P(X=0)=0$. Да.
\item Для неотрицательной случайной величины $\E(X)\ge \E(-X)$. Да.
\item Вероятность бывает отрицательной. Даже не знаю, что и сказать...
\end{enumerate}

\subsubsection*{Часть II.}

Стоимость задач 10 баллов.

\begin{enumerate}
% числа выверены
\item На день рождения к Васе пришли две Маши, два Саши, Петя и Коля. Все вместе с Васей сели за круглый стол. Какова вероятность, что Вася окажется между двумя тезками?

% числа выверены
\item Поезда метро идут регулярно с интервалом 3 минуты. Пассажир
приходит на платформу в случайный момент времени. Пусть $X$ —
время ожидания поезда в минутах.

Найдите $\P(X<1)$, $\E(X)$.

%\textbf{Задача 2} \\ % числа выверены
%На десяти карточках написаны числа от 1 до 9. Число 8 фигурирует
%два раза, остальные числа - по одному разу. Карточки извлекают в
%случайном порядке. \\
%Какова вероятность того, что девятка появится позже обеих
%восьмерок? \\

 % числа выверены
\item Вы играете две партии в шахматы против незнакомца. Равновероятно
незнакомец может оказаться новичком, любителем или профессионалом.
Вероятности вашего выигрыша в отдельной партии, соответственно,
будут равны: 0.9; 0.5; 0.3.
\begin{enumerate}
\item Какова вероятность выиграть первую партию?
\item Какова вероятность выиграть вторую партию, если вы выиграли
первую?
\end{enumerate}

 % числа выверены
\item Время устного ответа на экзамене распределено по экспоненциальному закону, то есть имеет функцию плотности $p(t)=c\cdot e^{-0.1t}$ при $t>0$.
\begin{enumerate}
\item Найдите значение параметра $c$
\item Какова вероятность того, что Иванов будет отвечать более получаса?
\item Какова вероятность того, что Иванов будет отвечать еще более получаса, если он уже отвечает 15 минут?
\item Сколько времени в среднем длится ответ одного студента?
\end{enumerate}


%\textbf{Задача 5} \\ % числа выверены
%Допустим, что вероятности рождения мальчика и девочки одинаковы. Сколько детей должно быть в семье, чтобы вероятность того, что имеется по крайней мере один ребенок каждого пола была больше
%0,95? \\


%\textbf{Задача 6} \\ % числа выверены
%Жители уездного города N независимо друг от друга говорят правду с вероятностью $\frac{1}{3}$. Вчера мэр города заявил, что в 2014 году в городе будет проведен межпланетный шахматный турнир. Затем заместитель мэра подтвердил эту информацию. \\
%Какова вероятность того, что шахматный турнир действительно будет проведен? \\

%\textbf{Задача 7} \\ % числа выверены
%Известно, что предварительно зарезервированный билет на автобус
%дальнего следования выкупается с вероятностью 0,9. В обычном
%автобусе 18 мест, в микроавтобусе 9 мест. Компания «Микро»,
%перевозящая людей в микроавтобусах, допускает резервирование 10
%билетов на один микроавтобус. Компания «Макро», перевозящая
%людей в обычных автобусах допускает резервирование 20
%мест на один автобус. \\
%У какой компании больше вероятность оказаться в ситуации нехватки
%мест? \\

 % числа выверены
\item Годовой договор страховой компании со спортсменом-теннисистом, предусматривает выплату страхового возмещения  в случае травмы специального вида. Из предыдущей практики известно, что вероятность получения теннисистом такой травмы  в любой фиксированный день равна 0.00037. Для периода действия договора вычислите
\begin{enumerate}
\item Наиболее вероятное число страховых случаев
\item Математическое ожидание числа страховых случаев
\item Вероятность того, что не произойдет ни одного страхового случая
\item Вероятность того, что произойдет ровно 2 страховых случая
\end{enumerate}
P.S. Указанные вероятности вычислите двумя способам: используя биномиальное распределение и распределение Пуассона.

 % числа выверены
\item Допустим, что закон распределения $X$ имеет вид:

\begin{tabular}{@{}cccc@{}}
\toprule
$x$    & $1$      & $2$       & $3$         \\ \midrule
$\P(X=x)$ & $\theta$ & $2\theta$ & $1-3\theta$ \\ \bottomrule
\end{tabular}
\begin{enumerate}
\item Найдите $\E(X)$ %, $\Var(X)$ \\
\item При каких $\theta$ среднее будет наибольшим? При каких — наименьшим?
\end{enumerate}
%в) При каких $\theta$ дисперсия будет наибольшей? При каких - наименьшей? \\

% числа выверены
\item Вася пригласил трех друзей навестить его. Каждый из них появится
независимо от другого с вероятностью $0.9$, $0.7$ и $0.5$
соответственно. Пусть $N$ — количество пришедших гостей. Найдите $\E(N)$.

% числа выверены
\item У спелестолога в каменоломнях сели батарейки в налобном фонаре, и он оказался в абсолютной темноте. В рюкзаке у него 6 батареек, 4 новых и 2 старых. Для работы фонаря требуется две новых батарейки. Спелестолог вытаскивает из рюкзака две батарейки наугад и вставляет их в фонарь. Если фонарь не начинает работать, то спелестолог откладывает эти две батарейки и пробует следующие две и так далее.
\begin{enumerate}
\item Найдите закон распределения числа попыток
\item Сколько попыток в среднем потребуется?
\item Какая попытка скорее всего будет первой удачной?
\end{enumerate}
\end{enumerate}

\subsubsection*{Часть III.}

Стоимость задачи 20 баллов.

Требуется решить \textbf{\underbar{одну}} из двух задач (9-А или 9-Б) по
выбору!

\begin{enumerate}
\item[9-А.] По краю идеально круглой столешницы отмечается наугад $n$ точек. В этих точках к столешнице прикручиваются ножки. Какова вероятность того, что полученный столик с $n$ ножками будет устойчивым?

\item[9-Б.] На окружности  с центром $O$ (не внутри окружности!) сидят три муравья, их
координаты независимы и равномерно распределены по окружности. Два
муравья $A$ и $B$ могут общаться друг с другом, если $\angle AOB<\pi/2$.

Какова вероятность того, что все три муравья смогут не перемещаясь
общаться друг с другом (возможно через посредника)?
\end{enumerate}



\subsection{Контрольная работа №1, 03.11.2007, решения}

\begin{enumerate}
\item Слева должен сесть тот, у кого есть тезка. $p_{1}=4/6$

Справа должен сесть его парный. $p_{2}=1/5$

Итого: $p=p_{1}\cdot p_{2}=2/15$
\item $p=1/3$, $\E(X)=1.5$
\item $p_{a}=\frac{1}{3}(0.9+0.5+0.3)=\frac{17}{30}$

$p_{b}=\frac{1}{3}(0.9^{2}+0.5^{2}+0.3^{2})/p_{a}=\frac{115}{170}$
\item
\begin{enumerate}
\item либо взятие интеграла, либо готовый ответ: $c=0.1$
\item $\int_{30}^{+\infty}p(t)dt=e^{-3}\approx 0.05$
\item Такой же результат, как в «б»
\item $1/\lambda=10$
\end{enumerate}
\item
\begin{enumerate}
\item[б)] $365\cdot 0.00037=0.13505$

Следовательно, «a», ближайшее целое равно 0.

Для Пуассоновского распределения: $\lambda=0.13505$
\item[в)] $\P(N=0)=0.99963^{365}\approx e^{-\lambda}$
\item[г)] $\P(N=2)=C_{365}^{2}0.99963^{363}0.00037^{2}\approx e^{-\lambda}\lambda^{2}/2$
\end{enumerate}
\item $\E(X)=3-4\theta$, $\theta\in[0;1/3]$, $\theta_{max}=0$, $\theta_{min}=1/3$
\item $N=X_{1}+X_{2}+X_{3}$, где $X_{i}$ равно 1 или 0 в зависимости от того, пришел ли друг. Значит $\E(N)=\E(X_{1})+\E(X_{2})+\E(X_{3})=0.9+0.7+0.5=2.1$
\item $\P(N=1)=\frac{C_{4}^{2}}{C_{6}^{2}}=6/15$

$\P(N=3)=\frac{4\cdot 2}{C_{6}^{2}}\frac{3\cdot 1}{C_{5}^{2}}=4/15$

$\P(N=2)=5/15$

$\E(N)=28/15$, первая.
\item[9-А.] Имеется $n$ способов выбрать левую точку. Оставшиеся $(n-1)$ точка должны попасть в правую полуокружность относительно выбранной левой точки.

Получаем $p=n\cdot (0.5)^{n-1}$
\item[9-Б.] Будем считать координату одного за точку отсчета. На квадрате $[0;1]\times[0;1]$ нетрудно нарисовать нужное множество.

$p=3/8$
\end{enumerate}



\subsection{Контрольная работа №2, демо-версия, 21.01.2008}

Демо-версия.

В контрольной на этом месте будет 10 тестовых вопросов!

\subsubsection*{Часть II.}

Стоимость задач 10 баллов.

\begin{enumerate}

% числа выверены
\item Совместный закон распределения случайных величин  $X$  и  $Y$
задан таблицей:

\begin{center}
\begin{tabular}{@{}cccc@{}}
\toprule
       & $Y=-1$ & $Y=1$ & $Y=2$ \\ \midrule
$X=-1$ & $0.1$  & $c$   & $0.2$ \\
$X=1$  & $0.1$  & $0.1$ & $0.1$ \\ \bottomrule
\end{tabular}
\end{center}

Найдите  $c$,  $\P\left(Y>X\right)$,  $\E\left(X\cdot Y \right)$,  $\E\left(X|Y>0\right)$. Являются ли величины $X$ и $Y$ независимыми?

% числа выверены
\item Случайный вектор  $\left(\begin{array}{c}
{X_{1} } \\ {X_{2} }
\end{array}\right)$  имеет нормальное распределение с
математическим ожиданием  $\left(\begin{array}{c} {-2} \\ {1}
\end{array}\right)$  и ковариационной матрицей
$\left(\begin{array}{cc} {9} & {-4} \\ {-4} & {36}
\end{array}\right)$.
\begin{enumerate}
\item Найдите  $\P\left(X_{1} +X_{2} >0\right)$.
\item Какое условное распределение имеет $X_{1}$ при условии, что $X_{2}=-1$?
\end{enumerate}

 % числа выверены
\item Совместная функция плотности имеет вид
\[
p_{X,Y} \left(x,y\right)=
\begin{cases}
  c(x-y), & \text{ если } x\in \left[0;1\right],\, y\in \left[0;1\right], x>y \\
  0, & \text{ иначе}
\end{cases}
\]
Найдите  $c$, $\P\left(3Y>X\right)$,  $\E\left(X\right)$, $\E(X|Y>0.5)$.

\item Вероятность дождя в субботу 0.5, вероятность дождя в воскресенье 0.3. Корреляция между наличием дождя в субботу и наличием дождя в воскресенье равна $r$.
Какова вероятность того, что в выходные вообще не будет дождя?

% числа выверены
\item Автор книги получает 50 тыс. рублей сразу после заключения
контракта и 5 рублей за каждую проданную книгу. Автор
предполагает, что количество книг, которые будут проданы — это
случайная величина с ожиданием в 10 тыс. книг и стандартным
отклонением в 1 тыс. книг. Чему равен ожидаемый доход автора? Чему
равна дисперсия дохода автора?

% числа выверены
\item Сейчас акция стоит 1000 рублей. Каждый день цена может равновероятно либо возрасти на~3 рубля, либо упасть на~5 рублей.
\begin{enumerate}
\item Чему равно ожидаемое значение цены через 60 дней? Дисперсия?
\item Какова вероятность того, что через 60 дней цена будет больше 900 рублей?
\end{enumerate}


 % числа выверены
\item В данном регионе кандидата в парламент Обещаева И.И.
поддерживает 60\% населения. Сколько нужно опросить человек, чтобы
с вероятностью 0.99 доля  опрошенных избирателей, поддерживающих
Обещаева~И.И.,  отличалась от 0.6 (истинной доли) менее, чем на
0.01?

% числа выверены
\item С помощью неравенства Чебышева, укажите границы, в~которых
находятся величины; рассчитайте также их точное значение
\begin{enumerate}
\item  $\P(-2\sigma<X-\mu<2\sigma)$, $X\sim \cN(\mu;\sigma^{2})$
\item  $\P(8<X<12)$, $X\sim U[0;20]$
\item $\P(-2<X-\E(X)<2)$, $X$ имеет экспоненциальное распределение с
$\lambda=1$
\end{enumerate}
\end{enumerate}

\subsubsection*{Часть III.}

Стоимость задачи 20 баллов.

Требуется решить \textbf{\underbar{одну}} из двух 9-х задач по
выбору!

\begin{enumerate}
\item[9-А.] Вы приехали в уездный город $N$. В городе кроме Вас живут $M$ мирных граждан и $U$ убийц. Каждый день на улице случайным образом встречаются два человека. Если встречаются два мирных гражданина, то они пожимают друг другу руки. Если встречаются мирный гражданин и убийца, то убийца убивает мирного гражданина. Если встречаются двое убийц, то оба погибают.

Каковы Ваши шансы выжить в этом городе? Зависят ли они от Вашей стратегии?

\item[9-Б.] Дед Мороз развешивает новогодние гирлянды. Аллея состоит из 2008 елок. Каждой гирляндой Дед Мороз соединяет две елки (не обязательно соседние). В результате Дед Мороз повесил 1004 гирлянды и все елки оказались украшенными. Какова вероятность того, что существует хотя бы одна гирлянда, пересекающаяся с каждой из других?

Например, гирлянда 5-123 (гирлянда соединяющая 5-ую и 123-ю елки) пересекает гирлянду 37-78 и гирлянду 110-318.

\emph{Подсказка}: Думайте!
\end{enumerate}


\subsection{Контрольная работа №2, демо-версия, 21.01.2008, решения}

\begin{enumerate}
\item $c=0.4$

$\P(Y>X) = \P(Y=1, X=-1) + \P(Y=2, X=-1) + \P(Y=2, X=1) = 0.7$

$\E(XY) = 0.1 -0.4 - 0.4 -0.1 + 0.1 +0.2 =-0.5$

$\E(X|Y>0) = -1\cdot\frac{0.6}{0.8} + 1\cdot \frac{0.2}{0.8} = -0.5$

Случайны величины $X$ и $Y$ не являются независимыми.
\item
\begin{enumerate}
\item Найдём распредление случайной величины $Z = X_1 + X_2$:
\[
\E(Z) = -1, \Var(Z) = \Var(X_1) + \Var(X_2) + 2 \Cov(X_1, X_2) = 37
\]
Получили, что $Z \sim \cN(-1, 37)$, тогда
\[
\P(Z>0) = \P\left(\frac{Z+1}{\sqrt{37}} > \frac{0+1}{\sqrt{37}}\right) = 0.4364
\]
\item $\E(X_1 | X_2 = -1) = -2 -4 \cdot \frac{1}{36} \cdot (-1 -1) = -\frac{16}{9}$

$\Var(X_1 | X_2 = -1) = 9 - (-4) \cdot \frac{1}{36} \cdot (-4) = \frac{77}{9}$

$X_1 | X_2 = -1 \sim \cN(-\frac{16}{9}, \frac{77}{9})$
\end{enumerate}
\item $c=6$

$\P(3Y>X) = \int_{0}^{1} \int_{0}^{3y} 6(x-y) dx dy = 3$

$f_{X} (x) = \int_{0}^{x} 6(x-y) dy = 3x^2 \Rightarrow \E(X) = \int_{0}^{1} 3x^3 dx = 0.75$
\item Введём следующие случайные величины:

$
X = \begin{cases}
1 & \text{в субботу не будет дождя}, p=0.5 \\
0 & \text{иначе}, p=0.5
\end{cases}
$
\hspace{0.5cm}
$
Y = \begin{cases}
1 & \text{в воскресенье не будет дождя }, p=0.7 \\
0 & \text{иначе }, p=0.3
\end{cases}
$

Найдем их математические ожидания и дисперсии: $\E(X)=0.5$, $\Var(X)=0.25$, $\E(Y)=0.3$, $\Var(Y)=0.21$.

В условии дана корееляция $X$ и $Y$, найдём ковариацию: $\Cov(X, Y) = r \cdot 0.5 \sqrt{0.21}$.
По определению, $\Cov(X, Y) = \E(XY)-\E(X)\E(Y)$, откуда можно найти $\E(XY)$: $\E(XY) = r \cdot 0.5 \sqrt{0.21} + 0.5\cdot0.7$.

Заметим, что $\E(XY)$ — это и есть искомая вероятность, потому что при подсчёте совместного математического ожидания в~сумме
будет только одно слагаемое, в~котором $X=1$ и $Y=1$, остальные же будут равны нулю.
\item Пусть $X$ — случайная величина, обозначающая количество проданных книг. Будем считать, что продажи каждой книги —
независимые события.

$\E(50 + 5X) = 100$, $\Var(50 + 5X) = 25$
\item Пусть $X$ — случайная величина, обозначающая изменение цены акции за день, a $S$ —  финальную стоимость акции.
\begin{enumerate}
\item $\E(S) = \E(1000 + 60X) = 1000 + 60 (0.5 \cdot 3 + 0.5 \cdot 5) = 1240$

$\Var(S) = \Var(1000 + 60X) = 3600(0.5 \cdot 9 + 0.5 \cdot 25 - 16) = 3600$
\item $\P(S > 900) = \P\left(\frac{S-1240}{60} > \frac{900-1240}{60} \right) = 1 - \P(\cN(0,1)< -\frac{17}{3}) \approx 1$
\end{enumerate}
\item $\P\left(|\hat{p} - 0.6| <0.01\right) = 0.99 \Rightarrow \P\left(\frac{|\hat{p} - 0.6|}{\sqrt{\frac{0.6\cdot0.4}{n}}} < \frac{0.01}{\sqrt{\frac{0.6\cdot0.4}{n}}} \right)=0.99 $

По таблице: $\frac{0.01}{\sqrt{\frac{0.6\cdot0.4}{n}}}  = 2.57 \Rightarrow n=62 $
\item
\begin{enumerate}
\item $\P(-2 < \cN(0,1) < 2) = 0.9544$, $1 - \frac{1}{4} < \P(-2\sigma<X-\mu<2\sigma) < 1$
\item $\P(8<X<12) = 0.2$, $1 - \frac{20^2}{12} < \P(-2 < X - \E(X) < 2)< 1$
\item $\P(-1 < X < 3) = \int_{-1}^{3} e^{-x} dx \stackrel{x>0}{=} 1 - e^{-3}$, $1- \frac{1}{4} < \P(-2<X-\E(X)<2) < 1$

\item[9-Б.]
Подразумевая под точками концы гирлянды, докажем следующее утверждение.

Бросим $2n \geq 4$ точек $X_1, X_2, \ldots, X_{2n}$ случайным образом на отрезок $[0;1]$. Пусть для $1 \leq i \leq n$ $J_i$ — это отрезок с концами $X_{2i-1}$ и $X_{2i}$.
Тогда вероятность того, что найдётся такой отрезок $J_i$, который пересекает все другие отрезки, равна $2/3$ и не зависит от $n$.

Доказательство. Бросим $2n+1$ точек на окружность, тогда $2n$ точек образуют пары, а оставшуюся обозначим $X$ и будем считать её и началом, и концом отрезка.
Каждому получившемуся отрезку присвоим уникальное имя.
Далее, будем двигаться от точки $X$ по часовой стрелке до тех пор, пока не встретим одно и то же имя дважды, например «а».
После этого пойдём в обратную сторону, и будем идти, пока не встретим какое-нибудь другое имя дважды, например, «б».
Теперь посмотрим на получившуюся последовательность между «б» и «а» на концах пути, читая её по часовой стрелке от «б» до «а».
Нас интересует взаимное расположение $X$, второй «а» и второй «б».
Зная, что «а» стоит после $X$, выпишем все возможные случаи, где может стоять «б»:
\begin{enumerate}
\item перед $X$
\item между $X$ и «а»
\item после «а»
\end{enumerate}
Покажем, что во втором и третьем случае отрезок «б» пересекает все остальные, а в первом такого отрезка вообще нет. Попутно заметим, что появление каждого и случаев равновероятно.

Действительно, если «б» стоит после $X$, и отрезок соответствующий этому имени, не пересекает какой-нибудь другой отрезок «в», то последовательность выглядела бы как «бвв$X$б» или «б$X$ввб», что противоречит описанному построению.
Если «б» стоит перед $X$ и отрезок «в» пересекает оба отрезка «а» и «б», то мы снова приходим в противоречие с построением.
В итоге, получаем, что искомая вероятность равна $2/3$.
\end{enumerate}
\end{enumerate}




\subsection{Контрольная работа №2, 21.01.2008}

\subsubsection*{Часть I.}

Обведите верный ответ:

\begin{enumerate}
\item Сумма двух нормальных независимых случайных величин нормальна.
Да.
\item Нормальная случайная величина может принимать отрицательные
значения. Да.
\item Пуассоновская случайная величина является непрерывной. Нет.
\item Дисперсия суммы зависимых величин всегда не меньше суммы
дисперсий. Нет.
\item Теорема Муавра-Лапласа является частным случаем центральной
предельной. Да.
\item Пусть $X$ — длина наугад выловленного удава в сантиметрах, а
$Y$ — в дециметрах. Коэффициент корреляции между этими
величинами равен $\frac{1}{10}$. Нет.
\item Математическое ожидание выборочного среднего не зависит от
объема выборки, если $X_{i}$ одинаково распределены. Да.
\item Зная закон распределения $X$ и закон распределения $Y$
можно восстановить совместный закон распределения пары $(X,Y)$. Нет.
\item Если  $X$  — непрерывная случайная величиа,  $\E\left(X\right)=6$  и
$\Var\left(X\right)=9$ , то  $Y=\frac{X-6}{3} \sim
\cN\left(0;1\right)$.  Нет.
\item Если ты отвечать на первые 10 вопросов этого теста наугад, то
число правильных ответов — случайная величина, имеющая
биномиальное распределение. Да.
\item Раз уж выпал свежий снег, то вместо контрольной можно
было бы покататься на лыжах! Да.
\end{enumerate}

Любой ответ на 11 считается правильным. \\
Тест не является блокирующим. \\
Обозначения: \\
$\E(X)$ — математическое ожидание \\
$\Var(X)$ — дисперсия

\subsubsection*{Часть II.}

Стоимость задач 10 баллов.

\begin{enumerate}
% числа выверены
\item Совместный закон распределения случайных величин  $X$  и  $Y$
задан таблицей:

\begin{tabular}{@{}cccc@{}}
\toprule
    & $Y=-1$ & $Y=0$ & $Y=2$ \\ \midrule
$X=0$ & $0.2$  & $c$   & $0.2$ \\
$X=1$ & $0.1$  & $0.1$ & $0.1$ \\ \bottomrule
\end{tabular}

Найдите  $c$,  $\P\left(Y>-X\right)$,  $\E\left(X\cdot Y \right)$, $\Corr(X,Y)$, $\E\left(Y|X>0\right)$.

% числа выверены
\item Случайный вектор  $\left(\begin{array}{c}
{X_{1} } \\ {X_{2} }
\end{array}\right)$  имеет нормальное распределение с
математическим ожиданием  $\left(\begin{array}{c} {2} \\ {-1}
\end{array}\right)$  и ковариационной матрицей
$\left(\begin{array}{cc} {9} & {-4.5} \\ {-4.5} & {25}
\end{array}\right)$.
\begin{enumerate}
\item Найдите  $\P\left(X_{1} +3X_{2} >20\right)$.
\item Какое условное распределение имеет $X_{1}$ при условии, что $X_{2}=0$?
\end{enumerate}

% числа выверены
\item Совместная функция плотности имеет вид
\[
p_{X,Y} \left(x,y\right)=
\begin{cases}
x+y, & \text{ если } x\in \left[0;1\right],\, y\in \left[0;1\right] \\
0, & \text{ иначе}
\end{cases}
\]
Найдите  $\P\left(Y>2X\right)$, $\E\left(X\right)$. Являются ли величины $X$ и $Y$ независимыми?

\item Вася может получить за экзамен равновероятно либо 8 баллов, либо 7 баллов. Петя может получить за экзамен либо 7 баллов — с вероятностью 1/3; либо 6 баллов — с вероятностью 2/3. Известно, что корреляция их результатов равна 0.7.

Какова вероятность того, что Петя и Вася покажут одинаковый результат?

% числа выверены
\item В городе Туме проводят демографическое исследование семейных пар. Стандартное отклонение возраста мужа оказалось равным 5 годам, а стандартное отклонение возраста жены — 4 годам. Найдите корреляцию возраста жены и возраста мужа, если стандартное отклонение разности возрастов оказалось равным 2 годам.

 % числа выверены
\item Сейчас акция стоит 100 рублей. Каждый день цена может равновероятно либо возрасти на 8\%, либо упасть на 5\%.
\begin{enumerate}
\item Какова вероятность того, что через 64 дня цена будет больше 110 рублей?
\item Чему равно ожидаемое значение логарифма цены через 100 дней?
\end{enumerate}
Подсказка: $\ln(1.08)=0.07696$, $\ln(0.95)=-0.05129$, $\ln(1.1)=0.09531$

\item Допустим, что срок службы пылесоса имеет экспоненциальное распределение. В среднем один пылесос бесперебойно работает 7 лет. Завод предоставляет гарантию 5 лет на свои изделия. Предположим также, что примерно 80\% потребителей аккуратно хранят все бумаги, необходимые, чтобы воспользоваться гарантией.
\begin{enumerate}
\item Какой процент потребителей в среднем обращается за гарантийным ремонтом?
\item Какова вероятность того, что из 1000 потребителей за гарантийным ремонтом обратится более 35\% покупателей?
\end{enumerate}
Подсказка: $\exp(5/7)=2.0427$

% числа выверены
\item Известно, что у случайной величины $X$ есть
математическое ожидание, $\E(X)=0$, и дисперсия.
\begin{enumerate}
\item Укажите верхнюю границу для $\P(X^{2}>2.56\cdot \Var(X))$? %$[5]$\\
\item Найдите указанную вероятность, если дополнительно известно, что
$X$ нормально распределена. %$[5]$\\
\end{enumerate}
\end{enumerate}

\subsubsection*{Часть III.}

Стоимость задачи 20 баллов.

Требуется решить \textbf{\underbar{одну}} из двух 9-х задач по
выбору!

\begin{enumerate}
\item[9-А.] Cлучайная величина $X$ распределена равномерно на отрезке $[0;1]$. Вася изготавливает неправильную монетку, которая выпадает «орлом» с вероятностью  $x$ и передает ее Пете.
Петя, не зная $x$, и подкидывает монетку один раз. Она выпала
«орлом».
\begin{enumerate}
\item Какова вероятность того, что она снова выпадет
«орлом»?
\item Как выглядит ответ, если Пете известно, что монетка при
$n$ подбрасываниях  $k$  раз выпала орлом?
\end{enumerate}

\item[9-Б.] В семье $n$ детей. Предположим, что вероятности рождения мальчика и девочки равны. Дед Мороз спросил каждого мальчика «Сколько у тебя сестер?» и сложив эти ответы получил $X$. Затем Дед Мороз спросил каждую девочку «Сколько у тебя сестер?» и cложив эти ответы получил $Y$. Например, если в семье 2 мальчика и 2 девочки, то каждая девочка скажет, что у нее одна сестра, а каждый мальчик скажет, что у него 2 сестры, $X=4$, $Y=2$
\begin{enumerate}
\item Найдите $\E(X)$ и $\E(Y)$
\item Найдите $\Var(X)$, $\Var(Y)$
\end{enumerate}
\emph{Подсказка}: Думайте!
\end{enumerate}



\subsection{Контрольная работа №2, 21.01.2008, решения}

\begin{enumerate}
\item $c=0.2$, далее $\P\left(Y>-X\right)=0.5$  и $\E\left(X\cdot Y\right)=0.1$

$\Corr(X,Y)=\frac{-0.02}{\sqrt{0.24\cdot 1.41}}$

$\E\left(Y|X>0\right)=0.25$
\item
\begin{enumerate}
\item $\E(S)=-1$, $\Var(S)=207$, $\P(Z>1.47)=1-0.9292=0.0708$
\item $p(x_{1}|0)\sim \exp\left(-\frac{1}{2}\left(\begin{array}{cc} {x_{1}-2} & {0+1} \end{array}\right) \left(\begin{array}{cc} {9} & {-4.5} \\ {-4.5} & {25}
\end{array}\right)^{-1}\left(\begin{array}{c} {x_{1}-2} \\ {0+1}
\end{array}\right)\right)$

$p(x_{1}|0)\sim \exp\left(-\frac{1}{2det}(25(x_{1}-2)^{2}+9(x_{1}-2)+9)\right)$

$p(x_{1}|0)\sim \exp\left(-\frac{1}{2\cdot 8.19}(x_{1}-1.82)^{2}\right)$

$\Var(X_{1}|X_{2}=0)=8.19$, $\E(X_{1}|X_{2}=0)=1.82$

Есть страшные люди, которые наизусть помнят, что:

$\Var(X_{1}|X_{2}=x_{2})=(1-\rho^{2})\sigma_{1}^{2}$

$\E(X_{1}|X_{2}=x_{2})=\mu_{1} + \rho\frac{\sigma_{1}}{\sigma_{2}}(x_{2}-\mu_{2})$
\end{enumerate}
\item $\P(Y>2X)=\int_{0}^{1}\int_{0}^{y/2}(x+y)dxdy=\frac{5}{24}$

$\E(X)=\int_{0}^{1}\int_{0}^{1}x(x+y)dxdy=\frac{7}{12}$

Зависимы
\item Рассмотрим $X=8-($Васин бал$)$ и $Y=($Петин бал$)-6$

$\Corr(X,Y)=-0.7$ (т.к. при линейном преобразовании может поменяться только знак корреляции)

$\Var(X)=\frac{1}{2}\left(1-\frac{1}{2}\right)$

$\Var(Y)=\frac{1}{3}\left(1-\frac{1}{3}\right)$

Интересующая нас величина - это $\P(X=1\cap Y=1)=\E(XY)=\Cov(X,Y)+\E(X)\E(Y)$

answer: $\frac{10-7\sqrt{2}}{60}\approx 0.001675$

key point: $\Cov=-\frac{7\sqrt{2}}{60}$
\item $\frac{37}{40}=0.925$
\item Частая ошибка в «а» — решение другой задачи, где проценты заменены на копейки.

Пусть $N$ — число подъемов акции.
\begin{enumerate}
\item
\begin{multline*}
\P(100\cdot 1.08^N\cdot 0.95^{64-N}>110)= \P(N \ln(1.08)+(64-N) \ln(0.95)> \ln(1.1))= \\
= \P\left(N>\frac{\ln(1.1)-64\ln(0.95)}{\ln(1.08)-\ln(0.95)}\right)
\end{multline*}
Заметим, что $N$ - биномиально распределена, примерно $N\left(64\cdot\frac{1}{2},64\cdot\frac{1}{4}\right)$

$Z=\frac{N-32}{4}$ - стандартная нормальная и $\P(Z>-1,42)=0.92$
\item $\E(N\ln(1.08)+(100-N)\ln(0.95))$

На этот раз $\E(N)=50$ и $\E(\ln(P_{100}))=1.28$
\end{enumerate}
\item $p_{break}=1-\exp(-5/7)=0.51=\int_{0}^{5}\frac{1}{7}e^{-\frac{t}{7}}dt$

$p=0.8\cdot 0.51\approx 0.4$

$\E(S)=1000p=400$, $\Var(S)=1000p(1-p)=240$

$\P(S>350)=\P(Z>-3.23)\approx 1$
\item
\begin{enumerate}
\item $\P(X^{2}>2.56\Var(X))=\P(|X-0|>1.6\sigma)\le
\frac{Var{X}}{2.56\Var(X)}=\frac{100}{256}\approx 0.4$
\item $\P(X^{2}>2.56\Var(X))=\P(|Z|>1.6)=0.11$
\end{enumerate}
\item[9-А.] б) Искомая вероятность равна $Prob=f(k+1,n-k)/f(k,n-k)$, где

$f(a,b)=\int_{0}^{1}x^{a}(1-x)^{b}dx$

Проинтегрировав по частям, видим, что $f(a,b)=f(a+1,b-1)\frac{b}{a+1}$

Отсюда $f(a,b)=\frac{a!b!}{(a+b+1)!}$

Подставляем, и получаем: $Prob=\frac{k+1}{n+2}$

Если кто получит этот ответ другим (более интуитивным) образом - тому большой дополнительный балл (!) - обращайтесь на \href{mailto:boris.demeshev@gmail.com}{boris.demeshev@gmail.com}
\item[9-Б.] Занумеруем детей в порядке появления на свет. Обозначим $M_{i}$ — индикатор того, что $i$-ый ребенок — мальчик, и $F_{i}$ — индикатор того, что $i$-ый ребенок — девочка. Конечно, $F_{i}+M_{i}=1$ и $F_{i}M_{i}=0$.
$M$, $F$ — общее число мальчиков и девочек соответственно.

Запасаемся простыми фактами:

$\E(F_{i})=\E(M_{i})=\E(F_{i}^{2})=\E(M_{i}^{2})=\frac{1}{2}$

$\E(F)=\E(M)=\frac{n}{2}$

$\Var(F_{i})=\Var(M_{i})=\frac{1}{4}$

$\Var(F)=\Var(M)=\frac{n}{4}$

$\E(F^{2})=\E(M^{2})=\Var(F)+\E(F)^{2}=\frac{n(n+1)}{4}$

$\E(FF_{i})=\frac{n+1}{4}$

Заметим, что $X_{i}=X_{i}+M_{i}F_{i}=M_{i}F$. Таким образом,

$X=MF=nF-F^{2}$

$Y_{i}=F-F_{i}-X_{i}$

$Y=(n-1)F-MF=(n-1)F-nF+F^{2}=F^{2}-F$

Далее берем матожидание (легко) и дисперсию (громоздко):  $\E(X)=\E(Y)=\frac{n(n-1)}{4}$

... (если кто решил до сих пор, то наверняка, он смог и дальше решить) ...
\end{enumerate}



\subsection{Контрольная работа №3, демо-версия, 01.03.2008}

Демо-версия кр3!
\subsubsection*{Часть I.} Здесь будет тест!
\subsubsection*{Часть II.}

Стоимость задач 10 баллов.

\begin{enumerate}
\item Вася и Петя метают дротики по мишени. Каждый из них сделал
по 100 попыток. Вася оказался метче Пети в 59 попытках.
\begin{enumerate}
\item На уровне
значимости 5\% проверьте гипотезу о том, что меткость Васи и Пети
одинаковая, против альтернативной гипотезы о том, что Вася метче
Пети.
\item Чему равно точное $P$-значение при проверке гипотезы в п. «а»?
\end{enumerate}

% числа выверены
\item Из 10 опрошенных студентов часть предпочитала готовиться по
синему учебнику, а часть - по зеленому. В таблице представлены их
итоговые баллы.

\begin{tabular}{@{}lcccccc@{}}
\toprule
Синий   & 76 & 45 & 57 & 65 &    &    \\
Зелёный & 49 & 59 & 66 & 81 & 38 & 88 \\ \bottomrule
\end{tabular}


С помощью теста Манна-Уитни (Mann-Whitney) проверьте гипотезу о
том, что выбор учебника не меняет закона распределения оценки.

% числа выверены
\item Имеется случайная выборка $X_{1}$, $X_{2}$, ..., $X_{n}$, где все $X_{i}$ имеют распределение, задаваемое табличкой:

\begin{tabular}{@{}lccc@{}}
\toprule
$x$         & $1$ & $2$   & $5$     \\ \midrule
$\P(X=x)$ & $a$ & $0.1$ & $0.9-1$ \\ \bottomrule
\end{tabular}
\begin{enumerate}
\item Постройте оценку неизвестного $a$ методом моментов
\item Является ли построенная оценка состоятельной?
\end{enumerate}

% числа выверены
\item Имеется случайная выборка $X_{1}$, $X_{2}$, ..., $X_{n}$, где все $X_{i}$ имеют $\cN(27,a)$ распределение.
Найдите оценку неизвестного $a$ методом максимального правдоподобия.

Напоминалка: не забудьте проверить условия второго порядка

% числа выверены
\item На курсе два потока, на первом потоке учатся 40 человек, на втором
потоке 50 человек. Средний балл за контрольную на первом потоке
равен 78 при (выборочном) стандартном отклонении в 7 баллов. На
втором потоке средний балл равен 74 при (выборочном) стандартном
отклонении в 8 баллов.
\begin{enumerate}
\item Постройте 90\% доверительный интервал для разницы баллов между
двумя потоками
\item На 10\%-ом уровне значимости проверьте гипотезу о том, что
результаты контрольной между потоками не отличаются.
\end{enumerate}


% числа выверены
\item Проверьте независимость пола респондента и предпочитаемого
им сока:

\begin{tabular}{@{}cccc@{}}
\toprule
  & Апельсиновый & Томатный & Вишнёвый \\ \midrule
М & $69$         & $40$     & $23$     \\
Ж & $74$         & $62$     & $34$     \\ \bottomrule
\end{tabular}

% числа выверены
\item На Древе познания Добра и Зла растет 6 плодов познания Добра и 5 плодов познания Зла. Адам и Ева съели по 2 плода. Какова вероятность того, что Ева познала Зло, если Адам познал Добро?

 % числа выверены
\item Пусть $X_{i}$ — независимы и имеют функцию плотности $p(t)=e^{a-t}$ при $t>a$, где $a$ - неизвестный параметр. В качестве оценки неизвестного $a$ используется $\hat{a}_{n}=\min\{X_{1},X_{2},...,X_{n}\}$.
\begin{enumerate}
\item Является ли предлагаемая оценка состоятельной?
\item Является ли предлагаемая оценка несмещенной?
\end{enumerate}

Решение:

Заметим, что $\hat{a}_{n}\geq a$.

$\P(|\hat{a}_{n}-a|>\varepsilon)=\P(\hat{a}_{n}-a>\varepsilon)=\P(\hat{a}_{n}>a+\varepsilon)=\P(\min\{X_{1},X_{2},...,X_{n}\}>a+\varepsilon)= \\
=\P(X_{1}>a+\varepsilon \cap X_{2}> a+\varepsilon\cap ...)=
\P(X_{1}>a+\varepsilon)\cdot \P(X_{2}>a+\varepsilon)\cdot ...=
\left(\int_{a+\varepsilon}^{\infty}e^{a-t}dt\right)^{n}=\left(e^{-\varepsilon}\right)^{n}=e^{-n\varepsilon}$

$\lim_{n\to\infty} e^{-n\varepsilon} =0$

б) нет, не является ни при каких $n$, хотя смещение с ростом $n$ убывает
\end{enumerate}

\subsubsection*{Часть III.}

Стоимость задачи 20 баллов.

Требуется решить \textbf{\underbar{одну}} из двух 9-х задач по
выбору!

\begin{enumerate}
\item[9-A.] Имеются две монетки. Одна правильная, другая — выпадает орлом с
вероятностью $0.45$. Одну из них, неизвестно какую, подкинули $n$
раз и сообщили Вам, сколько раз выпал орел. Ваша задача проверить
гипотезу $H_{0}$: «подбрасывалась правильная монетка» против
$H_{a}$:
«подбрасывалась неправильная монетка».

Каким должно быть наименьшее $n$ и критерий выбора гипотезы, чтобы
вероятность ошибок первого рода не превышала 10\%, а вероятность
ошибки второго рода не превышала 15\%?

\item[9-B.] Пусть $X_{i}$ — iid, $U[-b;b]$. Имеется выборка из 2-х наблюдений. Вася строит оценку для $b$ по формуле $\hat{b}=c\cdot (|X_{1}|+|X_{2}|)$.
\begin{enumerate}
\item При каком $c$ оценка будет несмещенной?
\item При каком $c$ оценка будет минимизировать средне-квадратичную ошибку, $MSE=\E((\hat{b}-b)^{2})$?
\end{enumerate}
\end{enumerate}

\subsection{Контрольная работа №3, 01.03.2008}

\subsubsection*{Часть I.}

Обведите верный ответ:

\begin{enumerate}
\item Мощность теста можно рассчитать заранее, до проведения теста. Да.
\item Точное $P$-значение можно рассчитать заранее, до проведения теста. Нет.
\item Если гипотеза отвергает при 5\%-ом уровне значимости, то
она обязательно будет отвергаться и при 10\%-ом уровне значимости. Да.
\item Мощность больше у того теста, у которого вероятность ошибки
1-го рода меньше.  Нет.
\item Функция плотности $F$-распределения $p(t)$ не определена при $t<0$.  Нет.
\item При большом $k$ случайную величину, имеющую $\chi_{k}^{2}$ распределение, можно считать нормально распределенной. Да.
\item Оценки метода моментов всегда несмещенные.  Нет.
\item Оценки метода максимального правдоподобия асимптотически несмещенные. Да.
\item Непараметрические тесты можно использовать, даже если закон распределения выборки неизвестен. Да.
\item Неравенство Крамера-Рао применимо только к оценкам метода максимального правдоподобия. Нет.
\end{enumerate}

Да — истинное утверждение, Нет - ложное \\
Тест не является блокирующим. \\
Обозначения: \\
$\E(X)$ — математическое ожидание \\
$\Var(X)$ — дисперсия

\subsubsection*{Часть II.}

Стоимость задач 10 баллов.

\begin{enumerate}
\item Школьник Вася аккуратно замерял время, которое ему требовалось, чтобы добраться от школы до дома. По результатам 90 наблюдений, среднее выборочное оказалось равным 14 мин, а несмещенная оценка дисперсии - 5 мин$^{2}$.
\begin{enumerate}
\item Постройте 90\% доверительный интервал для среднего времени на дорогу
\item На уровне значимости 10\% проверьте гипотезу о том, что среднее время равно 14,5 мин, против альтернативной гипотезы о меньшем времени.
\item Чему равно точное $P$-значение при проверке гипотезы в п. «б»?
\end{enumerate}

% числа выверены
\item Садовник осматривал розовые кусты и записывал число цветков. Всего в саду растет 25 розовых кустов. Предположим, что количество цветков на разных кустах независимы и одинаково распределены. \\
Вот заметки садовника:

12, 17, 21, 14, 15; 21, 16, 24, 11, 14; 22, 17, 21, 14, 15; 12, 26, 14, 21, 14; 11, 31, 18, 13, 18.

Проверьте гипотезу о том, что медиана количества цветков равна 19.

% числа выверены
\item Имеется случайная выборка $X_{1}$, $X_{2}$, ..., $X_{n}$, где все $X_{i}$ имеют распределение, задаваемое табличкой:

\begin{tabular}{@{}cccc@{}}
\toprule
$x$         & $1$ & $2$  & $5$    \\ \midrule
$\P(X=x)$ & $a$ & $2a$ & $1-3a$ \\ \bottomrule
\end{tabular}
\begin{enumerate}
\item Постройте оценку неизвестного $a$ методом моментов
\item Является ли построенная оценка несмещенной?
\end{enumerate}

% числа выверены
\item Имеется случайная выборка $X_{1}$, $X_{2}$, ..., $X_{n}$, где все $X_{i}$ имеют $N(a,4a)$ распределение.
Найдите оценку неизвестного $a$ методом максимального правдоподобия.

Напоминалка: не забудьте проверить условия второго порядка

 % числа выверены
\item Допустим, что логарифм дохода семьи имеет нормальное распределение. В городе А была проведена случайная выборка 40 семей, показавшая выборочную дисперсию 20 (тыс.р.)$^{2}$. В городе Б по 30 семьям выборочная дисперсия оказалась равной 32 (тыс.р.)$^{2}$.
На уровне значимости 5\% проверьте гипотезу о том, что дисперсия (логарифма дохода) одинакова, против альтернативной гипотезы о том, что город А более однородный.

 % числа выверены
\item Учебная часть утверждает, что все три факультатива («Вязание крючком для экономистов», «Экономика вышивания крестиком» и «Статистические методы в макраме») одинаково популярны. В этом году на эти факультативы соответственно записалось 35, 31 и 40 человек. Правдоподобно ли заявление учебной части?

% может изменить одно из 0.7 на 0.6?
% числа выверены
\item Снайпер попадает в «яблочко» с вероятностью 0.8, если в предыдущий раз он попал в «яблочко» и с вероятностью 0.7, если в предыдущий раз он не попал в «яблочко» или если это был первый выстрел. Снайпер стрелял по мишени 3 раза.
\begin{enumerate}
\item Какова вероятность попадания в «яблочко» при втором выстреле?
\item Какова вероятность попадания в «яблочко» при втором выстреле, если при первом снайпер попал, а при третьем - промазал?
\end{enumerate}

% числа выверены
\item Пусть $X_{i}$ — независимы и распределены равномерно на $[a-1;a]$, где $a$ — неизвестный параметр. В качестве оценки неизвестного $a$ используется $\hat{a}_{n}=\max\{X_{1},X_{2},...,X_{n}\}$.
\begin{enumerate}
\item Является ли предлагаемая оценка состоятельной?
\item Является ли предлагаемая оценка несмещенной?
\end{enumerate}
\end{enumerate}

\subsubsection*{Часть III.}

Стоимость задачи 20 баллов.

Требуется решить \textbf{\underbar{одну}} из двух 9-х задач по
выбору!

\begin{enumerate}
\item[9-A.] Два лекарства испытывали на мужчинах и женщинах. Каждый
человек принимал только одно лекарство. Общий процент людей,
почувствовавших улучшение, больше среди принимавших лекарство А.
Процент мужчин, почувствовавших улучшение, больше среди принимавших лекарство В. Процент женщин, почувствовавших улучшение, больше среди принимавших лекарство В. Возможно ли это?

\item[9-B.] Есть два золотых слитка, разных по весу. Сначала взвесили первый слиток и получили результат $X$. Затем взвесили второй слиток и получили результат $Y$. Затем взвесили оба слитка и получили результат $Z$. Допустим, что ошибка каждого взвешивания — это случайная величина с нулевым средним и дисперсией $\sigma^{2}$.
\begin{enumerate}
\item Придумайте наилучшую оценку веса первого слитка.
\item Сравните придуманную Вами оценку с оценкой, получаемой путем усреднения двух взвешиваний первого слитка.
\end{enumerate}
\end{enumerate}


\subsection{Контрольная работа №3, 01.03.2008, решения}

\begin{enumerate}
\item
\begin{enumerate}
\item $[13.61;14.39]$
\item Отвергается ($Z_{observed}=-2.12$, $Z_{critical}=-1.28$)
\item $P_{value}=0.017$
\end{enumerate}
\item Заменяем числа на цифры 0 и 1 (0 - меньше 19 цветков), (1 - больше)

$\hat{p}=\frac{8}{25}=0.32$

$H_{0}$: $p=0.5$

$H_{a}$: $p\neq 0.5$

$Z=\frac{0.32-0.5}{\sqrt{\frac{0.5\cdot 0.5}{25}}}=-1.8$

При уровне значимости 5\%, $Z_{critical}=1.96$. Значит, $H_{0}$ - не отвергается.
\item
\begin{enumerate}
\item $\hat{a}=\frac{5-\bar{X}}{10}$
\item Да, является
\end{enumerate}
\item $L=-\frac{n}{2}\ln(a)-\frac{na}{8}-\frac{\sum X_{i}^{2}}{8a}+c$

$L'=0$ равносильно $\hat{a}^{2}+4\hat{a}+4=4+\frac{\sum X_{i}^{2}}{n}$

$\hat{a}=-2+\sqrt{4+\frac{\sum X_{i}^{2}}{n}}$
\item $F_{29.39}=\frac{32}{20}=1.6$

$F_{critical}=1.74$

Гипотеза о том, что дисперсия одинакова не отвергается.
\item $\chi^{2}_{observed}=1.15$

$\chi^{2}_{2,5\%}=5.99$

Правдоподобно.
\item
\begin{enumerate}
\item $p=0.7\cdot 0.8+ 0.3\cdot 0.7=0.77$
\item $p=\frac{0.7\cdot0.8\cdot0.2}{0.7\cdot 0.8\cdot 0.2 + 0.7\cdot 0.2 \cdot 0.3}=\frac{8}{11}$
\end{enumerate}
\item
\begin{enumerate}
\item Заметим, что $\hat{a}_{n}\leq a$.
\begin{multline*}
\P(|\hat{a}_{n}-a|>\varepsilon)=\P(-(\hat{a}_{n}-a)>\varepsilon)=\P(\hat{a}_{n}<a-\varepsilon)=\P(\max\{X_{1},X_{2},...,X_{n}\}<a-\varepsilon)= \\
=\P(X_{1}<a-\varepsilon \cap X_{2}< a-\varepsilon\cap ...)=\P(X_{1}<a-\varepsilon)\cdot \P(X_{2}<a-\varepsilon)\cdot ...=(1-\varepsilon)^{n}
\end{multline*}
$\lim_{n\to\infty} (1-\varepsilon)^{n} =0$
\item Нет, не является ни при каких $n$, хотя смещение с ростом $n$ убывает
\end{enumerate}
\item[9-А.] Да, \url{http://en.wikipedia.org/wiki/Simpson's_paradox}
\item[9-Б.]
\begin{enumerate}
\item Пусть истинные веса слитков равны $x$, $y$ и $z$.

Назовем оценку буквой $\hat{x}$

$\hat{x}=aX+bY+cZ$

Несмещенность: $\E(\hat{x})=a\E(X)+b\E(Y)+c\E(Z)=ax+by+c(x+y)=x$

$a+c=1$, $b+c=0$

$\hat{x}=(1-c)X+(-c)Y+cZ$

Эффективность: $\Var(\hat{x})=((1-c)^{2}+c^{2}+c^{2})\cdot \sigma^{2}=(3c^{2}-2c+1)\sigma^{2}$

Чтобы минимизировать дисперсию нужно выбрать $c=1/3$

Т.е. $\hat{x}=\frac{2}{3}X-\frac{1}{3}Y+\frac{1}{3}Z$
\item $\Var(\hat{x})=\frac{2}{3}\sigma^{2}$

$\Var\left(\frac{X_{1}+X_{2}}{2}\right)=\frac{1}{2}\sigma^{2}$

Усреднение двух взвешиваний первого слитка лучше.
\end{enumerate}
\end{enumerate}
